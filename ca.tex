https://www.vatican.va/content/john-paul-ii/en/encyclicals/documents/hf_jp-ii_enc_01051991_centesimus-annus.html


\begin{quote}
  All of this can be summed up by repeating once more that economic freedom is only one element of human freedom. When it becomes autonomous, when man is seen more as a producer or consumer of goods than as a subject who produces and consumes in order to live, then economic freedom loses its necessary relationship to the human person and ends up by alienating and oppressing him.
  \attrib{\textit{CA} 39, JPII 1991}
\end{quote}

\begin{quote}
  Marxism criticized capitalist bourgeois societies, blaming them for the commercialization and alienation of human existence. This rebuke is of course based on a mistaken and inadequate idea of alienation, derived solely from the sphere of relationships of production and ownership, that is, giving them a materialistic foundation and moreover denying the legitimacy and positive value of market relationships even in their own sphere. Marxism thus ends up by affirming that only in a collective society can alienation be eliminated. However, the historical experience of socialist countries has sadly demonstrated that collectivism does not do away with alienation but rather increases it, adding to it a lack of basic necessities and economic inefficiency.

  The historical experience of the West, for its part, shows that even if the Marxist analysis and its foundation of alienation are false, nevertheless alienation — and the loss of the authentic meaning of life — is a reality in Western societies too. This happens in consumerism, when people are ensnared in a web of false and superficial gratifications rather than being helped to experience their personhood in an authentic and concrete way. Alienation is found also in work, when it is organized so as to ensure maximum returns and profits with no concern whether the worker, through his own labour, grows or diminishes as a person, either through increased sharing in a genuinely supportive community or through increased isolation in a maze of relationships marked by destructive competitiveness and estrangement, in which he is considered only a means and not an end.

  The concept of alienation needs to be led back to the Christian vision of reality, by recognizing in alienation a reversal of means and ends. When man does not recognize in himself and in others the value and grandeur of the human person, he effectively deprives himself of the possibility of benefitting from his humanity and of entering into that relationship of solidarity and communion with others for which God created him. Indeed, it is through the free gift of self that man truly finds himself. This gift is made possible by the human person's essential ``capacity for transcendence". Man cannot give himself to a purely human plan for reality, to an abstract ideal or to a false utopia. As a person, he can give himself to another person or to other persons, and ultimately to God, who is the author of his being and who alone can fully accept his gift. A man is alienated if he refuses to transcend himself and to live the experience of self-giving and of the formation of an authentic human community oriented towards his final destiny, which is God. A society is alienated if its forms of social organization, production and consumption make it more difficult to offer this gift of self and to establish this solidarity between people.
  \attrib{\textit{CA} 41, JPII 1991}
\end{quote}