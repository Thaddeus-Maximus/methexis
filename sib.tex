
E.F. Schumacher, author of /textit{Small Is Beautiful}, was statistician, economist, a dedicated atheist, but converted to Catholicism. He also helped found the Intermediate Technology Development Group which today operates as Practical Action. \href{https://www.religion-online.org/article/small-is-beautiful-and-so-is-rome-surprising-faith-of-e-f-schumacher/}{Charles Fager writes}:

\begin{quote}
... one of the most frequently cited sections [of \textit{Small is Beautiful}], ``Buddhist Economics", almost made it appear as if he were deeply involved in Eastern religions. But wasn't this section, I inquired, really more informed by the Catholic writings and thinkers he mentioned so frequently elsewhere in the book - the papal encyclicals, Newman, Gilson and, above all, Thomas Aquinas?
\end{quote}

Indeed, Schumacher even admits immediately before the iconic section, "The choice of Buddhism for this purpose is purely incidental; the techings of Christianity, Islam, or Judaism could have been used just as well." [p. 52]

%% Side takeaway from this article, worth thinking more about: Buddhism for Schumacher was an entry point back into Christianity from Atheism insofar as it let him cast aside his modern western antichristian biases.
%% And if I were to go around England passing myself off as a Buddhist, then I would also be thinking that everyone else around me was stupid, because they'd all got the wrong religion. They're all unenlightened, while I'm the one who has the truth. And there are many people in the West these days going around acting like quasi-Orientals, with dreadful results.

Schumacher emphasizes that the Christian message is a very profound and quite specific one rather than some vague feel-good notion which could manifest in a broad variety of teachings.

\begin{quote}
I foudn that in England almost any old nonsense was being written and passed off as Christianity, even by bishops. And so I finally decided that the Catholic tradition was the one where I felt most at home, and where the essentials of Christianity were best preserved.
\end{quote}

Schumacher is quite adamant that Catholicism must be a \textit{social} affair, that it must manifest in community and continuity. Dr. John Coleman, professor of religion and society at the Jesuit School of Theology in Berkeley, stated,

\begin{quote}
By this I mean the stream of Catholic thought that build on Thomistic principles, as particularly reapplied in the work of Jacques Martain. Its adherents stressed that human institutions ought to be manageable in size, respectful of the human scale, and sanely run so that they did not damage the people involved in them.
\end{quote}

% Schumacher extends the approach that the Family and Church exist as mediators and outside of the government- to technology. Would like to find direct exemplats of this.

Coleman continues:

\begin{quote}
The problem with social Catholicism, is that it has been mainly enunciated rather than acted upon.
\end{quote}






\begin{quote}
The Buddhist point of view takes the function of work to be at least threefold: to give man a chance to utilise and develop his faculties; to enable him to overcome his ego-centredness by joining with other people in a common task; and to bring forth the goods and services needed for a becoming existence.
\end{quote} [SIB, p.54-55]

\begin{quote}
The craftsman himself can always, if allowed to, draw the delicate distinction between the machine and the tool. The carpet loom is a tool, a contrivance for holding warp threads at a stretch for the pile to be woven roudn them by the craftsmen's fingers; but the power loom is a machine, and its significance as a destroyer of culture lies in the fact that it does the essentially human part of the work.
\end{quote} [Ananda Coomaraswamy via SIB, p.55]


\begin{quote}
  The men who conceived the idea that "morality is bunk" did so with a mind well-stocked with moral ideas. But the minds of the third and fourth generations are no longer well-stocked with such ideas: they are well-stocked with ideas conceived in the nineteenth century, namely, that "morality is bunk," that everything that appears to be "higher" is really nothing but something quite mean and vulgar.
\end{quote}[SIB]

\begin{quote}
What do I miss, as a human being, if I have never heard of the Second Law of Thermodynamics? The answer is: Nothing. And what do I miss by not knowing Shakespeare? Unless I get my understanding from another source, I simply miss my life.
\end{quote}[SIB, p. 87]

\begin{quote}
  Education cannot help us along as it accords no place to metaphysics. Whether the subjects taught are subjects of science or of the humanities, if the teaching does not lead to a clarification of metaphysics, that is to say, of our fundamental convictions, it cannot educate a man, and consequently, is of no value to society.
\end{quote}[SIB, p.93]

\begin{quote}
  Life is kept going by divergent problems which have to be "lived" and are solved only in death. Convergent problems on the other hand are man's most useful invention; they do not, as such, exist in reality, but are created by a process of abstraction. When they have been solved, the solution can be written down and passed on to others, who can apply it without needing to reproduce the mental effort necessary to find it. If this were the case with human relations - in family life, economics, politics, education, and so forth - well, I am a loss at how to finish the sentence. There would be no more human relations but only mechanical reactions; life would be a living death. Divergent problems, as it were, force man to strain himself to a level above himself; they demand, and thus provoke the supply of, forces from a higher level, thus bringing love, beauty, goodness and truth into our lives. It is only with the help of these higher forces that the opposites can be reconciled in the living situation.
\end{quote} [SIB, p.97-98]


\begin{quote}
The very start of Buddhist economic planning would be a planning for full employment, and the primary purpose of this would in fact be employment for everyone who needs an "outside" job: it would not be the maximisation of employment nor the maximisation of production. Women, on the whole, do not need an "outside" job, and the large-scale employment of women in offices or factories would be considered a sign of serious economic failure. In particular, to let mothers of young children work in factories while the children run wild would be as uneconomic in the eyes of a Buddhist economist as the employment of a skilled worker as a soldier in the eyes of a modern economist.
\end{quote}

This phrase about women may sound misogynist (bear in mind, this work was written in 1973). But the underlying sentiment is anything but. Schumacher has claimed women to be the most skilled labor of all and that their utilzation in a factory environment is downright inefficient- no, insulting, to women, children, and society.

To the true-libertarian-types out there, Schumacher makes it clear:

\begin{quote}
While the materialist is mainly interested in goods, the Buddhist is mainly interested in liberation... For the modern economist this is very difficult to understand. He is used to measuring the "standard of living" by the amoutn of annual consumption, assuming all the time athat a man who consumes more is "better off" than a man who consumes less. A Buddhist economist would consider this approach excessively irrational: since consumption is merely a means to human well-being, the aim should be to obtain the maximum of well-being with the minimum of consumption.
\end{quote} [p.57]

\begin{quote}
From teh point of view of Buddhist economics, therefore, production from local resources for local needs is the most rational way of economic life, while dependence on imports from afar and the consequent need for produce for export to unknown and distanct peoples is highly uneconomic and justifiable only in eceptional cases and on a small scale.
\end{quote} [p.59]

\begin{quote}
Is it not true that the great prosperity of Germany became possible only through this unification? All the same, the German-speaking Swiss and the German-speaking Austrians, who did not join, did just as well economically, and if we make a list of all the most properous countries in the world, we find that most of them are very small; whereas a list of all the biggest countries in the world shows most of them to be very poor indeed.
\end{quote} [p.64]

\begin{quote}
there always appears to be a need for at least two things simultaneously, which on the face of it, seem to be incompatible and to exclude one another. We always need both freedom and order. We need the freedom of lots and lots of small, autonomous units, and, at the same time, the orderliness of large-scale, possibly global, unity and coordination. When it comes to action, we obviously need small units, because action is a highly personal affair, and one cannot be in touch with more than a very limited number of persons at any one time. But when it comes to the world of ideas, to principles or to ethics, to the indivisibility of peace and also of ecology, we need to recognise the unity of mankind and base our actions upon this recognition.
\end{quote} [p.65]

\begin{quote}
A highly developed transport and communications system has one immensely powerful effect: it makes people footloose. Millions of people start moving about, deserting the rural areas and the smaller towns to follow the city lights, to go to the big city, causing a patholigical growth. Take the country in which all this is perhaps most exemplified - the United States... They freely talk abotu the polarisation of the population of the United States into three immense megalopolitan areas: [Boston to Washington, Chicago, and San Fransisco to San Diego]; the rest of the country being left practically empty; deserted provincial towns, and the land cultivated with vast tractors, combine harvesters, and immense amounts of chemicals. If this is somebody's conception of the future of the United States, it is hardly a future worth having.
\end{quote}[p.68]

Again the date of this publishing proves prophetic: in the US, Far Western states are the fastest growing- in particular, Nevada and Idaho, while California and most of the eastern seaboard stagnates in growth, and Illinois and New York continue to shed population. This trend predates the Coronavirus pandemic and many of these more populous states' lockdown policies. The reasons cited by many are a sense of urban crowding and decreased quality of life with a vague notion that reconnection to nature and community is better served by a more rural, or at least suburban, lifestyle.

\begin{quote}
If so much reliance is today being placed in the power of education to enable ordinary people to cope with the problems thrown up by scientific and technological progress, then there must be something more to education than Lord Snow suggests [that all men must be educated in technical matters]. Science and engineering produce "know-how"; but "know-how" is nothing by itself; it is a means without an end, a mere potentiality, an unfinished sentence. "Know-how" is no more a culture than a piano is music. Can education help us to finish the sentence, to turn the potentiality into a reality to the benefit o man? To do so, the task of education would be, first and foremost, the transmission of ideas of value, of what to do with our lives. There is no doubt also the need to transmit know-how but this must take second place, for it is obviously somewhat foolhardy to put great powers into the hands of people without making sure that they have a reasonable idea of what to do with them.
\end{quote}[p.81-82]

It's worth noting that these words were written in 1977, and Snow said the Russians were doing much better than the Western world and will have a clear edge. We know how that turned out.

\begin{quote}
When people ask for education, they normally mean something more than mere training, something more than mere knowledge of facts, and something more than mere diversion. Maybe they cannot themselves formulate precisely what they are looking for; but I think that what they are looking for is ideas that would make the world, and thair whole lives, intelligible to them. When a thing is intelligible, you have a sense of participation; when a thing is unintelligible you have a sense of estrangement.
\end{quote}[p.84]

\begin{quote}
  The most striking thing about modern industry is that it requires so much and accomplishes so little. Modern industry seems to be inefficient to a degree that surpasses one's ordinary powers of imagination. Its inefficiency therefore remains unnoticed.
\end{quote} [SIB, p.118]

\begin{quote}
  An industrial system that uses forty percent of the world's primary resources to supply less than six percent of the world's population could be called efficient only if it obtained strikingly successful results in terms of human happiness, well-being, culture, peace, and harmony. I do not need to dwell on the fact that the American system fails to do this.
\end{quote}[SIB, p.119]

The operative word may be strikingly- because supplying 6 percent with 40 is a factor of ten times less efficient than 94 with 60.

\begin{quote}
  The religion of economics promotes an idolatry of rapid change, unaffected by the elementary truism that a change is which is not an unquestionable improvement is a doubtful blessing. The burden of proof is placed on those who take the 'ecological viewpoint': unless \textit{they} can produce evidence of marked injury to man, the change will proceed. Common sense, on the contrary, would suggest that the burden of proof should lie on the man who wants to introduce a change; \textit{he} has to demonstrate that there \textit{cannot} be any damaging consequences. But this would take too much time, and would therefore be uneconomic.
\end{quote}[SIB, p.134]

\begin{quote}
  All changes in a complex mechanism involve some risk and should be undertaken only after careful study of all the facts available. Changes should be made on a small scale first so as to provide a test before they are widely applied. When information is incomplete, changes should stay close to the natural processes which have in their favour the indisputable evidence of having supported life for a very long time.
\end{quote}[SIB, p.135]

While some accuse the Amish of being stuck-up and altogether rejecting of technology, this idea of waiting on evidence is really what they are employing rather than some paleolithic rejection.

\begin{quote}
  Rather less than one-half of this coutnry is, as they say, gainfully occupied, and about one-third of these are actual producers in agriculture, mining, construction, and industry. I do mean \textit{actual production}, not people who tell other people what to do, or account for the past, or plan for the future, or distribute what other people have produced. In other words, rather less than one-sixth of the total population is engaged in actual production; on average, each of them supports five others beside himself, of which two are gainfully employed on things other than real production and three are not gainfully employed. Now, a fully employed person, allowing for holidays, sickness, and other absence, spends about one-fifth of his total time on his job. It follows that the proportion of "total social time" spent on actual production - in the narrow sense in which I am using the term - is roughly, one-fifth of one-third of one-half, i.e. 3 1/2 percent. The other 96 1/2 percent of "total social time" is spent in other ways, including sleeping, eating, watching television, doing jobs that are not directly productive, or just killing time more or less humanely.
\end{quote} [SIB, p.149-150]

\begin{quote}
  [these calculations] quote adequately serve to show what technology has enabled us to do: namely, to reuce the amount of time actually spent on production in its most elementary sense to such a tiny percentage of total social time that it pales into insignificance, that it carries no real weight, let along prestige. When you look at industrial society in this way, you cannot be suprised to find that prestige is carried by those who help fill the other 96 1/2 percent of total social time, primarily the entertainers but also the executors of Parkinson's Law. Infact, one might put the following proposition to students of sociology: "The prestige carried by people in modern industrial society variesi n inverse proportion to their closeness to actual production"
\end{quote} [SIB, p.150]

\begin{quote}
  The whole drift of modern technological development is to reduce [directly productive time] further, asymptotically to zero. Imagine we set ourselves a goal in the opposite direction: to increase it sixfold, to about twenty percent, so that twenty percent of total social time would be used for actually producing things, employing hands and brains and, naturally, excellent tools. An incredible thought! Even children would be allowed to make themselves useful, even old people. At one-sixth of present-day productivity, we should be producing as much as at present. There would be six times as much time for any piece of work we chose to undertake- enough to make really good job of it, to enjoy oneself, to produce real quality, even to make things beautiful. Think of the therapeutic value of real work; think of its educational value. No one would then want to raise the school-leaving age or to lower the retirement age, so as to keep people off the labour market. Everybody would be welcome to lend a hand. Everybody would be admitted to what is now the rarest privilege, the opportunity of working usefully, creatively, with his own hands and brains, in his own time, at his own pace- and with excellent tools. Would this mean an enormous extension of working hours? No, people who work in this way do not know the difference between work and leisure. Unless they sleep or eat or occasionally choose to do nothing at all, they are always agreeable, productively engagaed. Many of the "on-cost jobs" would simply disappear; I leave it to the reader's imagination to identify them. There would be little need for mindless entertainment or other drugs, and unquestionably much less illness.
\end{quote}[SIB, p.151-152]

\begin{quote}
  It is almost like a providential blessing that we, the rich countries, have found it in our heart at least to consider the Third World and to try to mitigate its poverty In spite of the mixture of motives and the persisstence of exploitative practices, I think that this fairly recent development in the outlook of the rich is an honourable one. And it could save us; for the poverty of the poor makes it in any case impossible for them successfully to adopt our technology. Of course they often try to do so, and then have to bearth e most dire consequences in terms of mass unemployment, mass migration into cities, rural decay, and intolerable social tensions. They need in fact, the very kind of thing I am talking about, which we also need: a \textit{different} kind of technology, a technology with a human face, which, instead of making human hands and brains redundant, helps them to become far more productive than they ever have before.
\end{quote}[SIB, p.153]

\begin{quote}
  It is my experiene that it is rather more difficult to racpture directness and simplicity than to advance in the direction of ever more sophistication and complexity. Any third-rate engineer or researcher can increase complexity; but it takes a certain flair of real insight to make things simple again. And this insight does not come easily to people who have allowed themselves to become alienated from real, productive work and from the self-balancing system of nature, which never fails to recognise measure and limitation. Any activity which fails to recognise a self-limiting principle is of the devil.
\end{quote}[SIB, p.154]

\begin{quote}
  Let us admit that the people of the forward stampede, like the devil, have all the best tunes or at least the most popular and familiar tunes. You cannot stand still, they say; standing still means going down; you must go forward; there is nothing wrong with modern technology except that it is as yet incomplete; let us complete it.

  ...

  This is the authentic voice of the forward stampede, which talks in much the same tone as Dostoyevsky's Grand Inquisitor: "Why  have you come to hinder us?"
\end{quote}[SIB, p.155]

\begin{quote}
  Strange to say, the sermon on the mount gives pretty precise instructions on how to construct and outlook that could lead to an Economics of Survival.

  - How blessed are those who know they are poor: the Kingdom of Heaven is theirs.
  - How blessed are the sorrowful; they shall find consolation.
  - How blessed are those of a gentle spirit; they shall have the earth for their possession.
  - How blessed are those who hunger and thirst to see right prevail; they shall be satisfied;
  - How blessed are the peacemakers; God shall call them his sons.

  It may seem daring to connect these beatitudes with matters of technology and economics. But may it not be that we are in trouble precisely because we have failed for so long to make this connection? It is not difficult to discern what these beatitudes may mean for us today:

  - We are poor, not demigods.
  - We have plenty to be sorrowful about, and are not emerging into a golden age.
  - We need a gentle approach, a non-violent spirit, and small is beautiful.
  - We must concern ourselves with justice and see right prevail.
  - And all this, only this, can enable us to become peacemakers.
\end{quote}[SIB, p.156-157]

\begin{quote}
  The home-comers base themselves upon a different picture of man from that which motivates the people of the forward stampede. It would be very superficial to say that the latter believe in "growth" while the former do not. In a sense, everybody believes in growth, and rightly so, because growth is an essential feature of life. The whole point, however, is to give to the idea of growth a qualitative determination; for there are always many things that ought to be growing and many things that ought to be diminishing.
\end{quote}[SIB, p.157]

\begin{quote}
  I think it was the Chinese, before WWII, who calculated that it tooks the work of thirty peasants to keep one man or woman at a university. If that person at the university took a five-year course, by the time he had finished he would have consumed 150 peasant-work-years. How can this be justified? Who has the right to appropriate 150 years of peasant work to keep one person at university for five years, and what do the peasants get back for it? These questions lead us to the parting of the ways: is education to be a "passport to privilege" or is it something which people take upon themselves almost like a monastic vow, a sacred obligation to serve the people? The first road takes the educated young man into the fashionable district of Bombay, which a lot of other highly educated people have already gone and where he can join a mutual admiration society, a "trade union of the privileged," to see to it that his privileges are not reoded by the great messes of his contemporaries who have not been educated. This is one way. The other way would be emarked upon in a different spirit and would lead to a different destination. It would take him back to the people who, after all, directly or indirectly, have paid for his education by 150 peasant-work-years; having consumed the fruits of their work, he would feel in honour bound to return something to them.
\end{quote}[SIB, p.207]

\begin{quote}
  The problem is not new. Leo Tolstoy referred to it when he wrote: "I sit on a man's back, choking him, and making him carry me, and yet assure myself and others that I am very sorry for him and wish to ease his lot be any means possible, except getting off his back."
\end{quote}[SIB, p.207-208]

\begin{quote}
  In those sixty years, a vast increase of inflexibility. Galbraith comments: "Had Ford and his associates [in 1903] decided at any point to shift from gasoline to steam power, the machine shop could have accommodated itself to the change in a few hours." Now, try to change even one screw, it takes that many months.
\end{quote}[SIB, p.211-212]

\begin{quote}
  The greatest deprivation anyone can suffer is to have no chance of looking after himself and making a livelihood.
\end{quote}[SIB, p.219]

\begin{quote}
  Modern industrial society, typified by large-scale organizations, gives far too little thought to it. Managements assume that people work simply for money, for the pay-packet at the end of the week. No doubt, this is true up to a point, but when a worker, asked why he worked only four shifts last week, answers: "Because I couldn't make ends meet on three shifts' wages," everyone is stunned and feels check-mated.
\end{quote}[SIB, p.249]

Such a worker disobeys a typical supply curve which suggests that he would work more if his wages were higher: clearly, if his wages were higher, he would work less!

% What were luxuries to our fathers have become necessities for us.

\begin{quote}
  I would very much prefer to take any interested person on a tour of our forty-five acre, ancient Manor House Estate, interspersed with chemical plants and laboratories, than to laboriously write [an] article which is bound to raise as many questions as it answers.
\end{quote}[Ernst Bader, via SIB, p.280]

% Matthew 6:33










This sounds good at first- but there is something amiss about this. I don't think drawing a perfect line is important, but understanding the principle at play- and how the line might not be between material goods- is. CNC Mills and 3D printers, for example, certainly can be either a \textit{tool} or \textit{machine}. If they are employed to pump out parts en masse, they are a \textit{machine} to those tending them. If they are used by a designer to make their designs manifest (some of which would not be feasible with the use of manual machines), they are most certainly a \textit{tool} as the human aspect of design is still in play. The step from one-off to mass production- from an expression of creativity to brainless toil- this is what can turn a \textit{tool} into a \textit{machine}.