\documentclass[10pt,letterpaper,openany]{book}
\usepackage[dvipsnames]{xcolor}
\usepackage{outlines}
\usepackage{amsmath}
\usepackage{tikz}
\usepackage{circuitikz}
\usepackage{svg}

\usepackage{hyperref}
\usepackage{enumitem}
\usepackage{caption}
\usepackage{subcaption}
\DeclareCaptionOptionNoValue{centering}{\centering} % Make sure everything is centered in subs
\captionsetup[sub]{centering}

\usepackage{multirow}
\usepackage{cancel}
\usepackage{float}

\usepackage{parskip}

\usepackage{slantsc,lmodern}

\usepackage{pgfplotstable,booktabs}
\usepackage{textcomp}
\usepackage{gensymb}

\usepackage{paralist}

\usepackage[paper=letterpaper,margin=1in]{geometry}

\usepackage{etoolbox}

\usepackage{imakeidx}
\makeindex

\usepackage{mdframed}
\mdfdefinestyle{qbox}{
  frametitlebackgroundcolor   =black!15,
    frametitlerule            =false,
    backgroundcolor     = lightgray,
    innertopmargin      =\topskip,
    innerbottommargin   =\topskip,
}
\newmdenv[style=qbox]{qbox}

%\addtolength{\oddsidemargin}{-.875in}
%\addtolength{\evensidemargin}{-.875in}
%\addtolength{\textwidth}{1.75in}

%\addtolength{\topmargin}{-.875in}
%\addtolength{\textheight}{1.75in}

% Centers all the floats
\makeatletter
\g@addto@macro\@floatboxreset\centering
\makeatother

\newcommand{\volume}{{\ooalign{\hfil$V$\hfil\cr\kern0.08em--\hfil\cr}}}

\newtheorem{theorem}{Theorem}

\begin{document}

\clearpage
%% temporary titles
% command to provide stretchy vertical space in proportion
\newcommand\nbvspace[1][3]{\vspace*{\stretch{#1}}}
% allow some slack to avoid under/overfull boxes
\newcommand\nbstretchyspace{\spaceskip0.5em plus 0.25em minus 0.25em}
% To improve spacing on titlepages
\newcommand{\nbtitlestretch}{\spaceskip0.6em}
\pagestyle{plain}
\begin{center}
  \bfseries
  \nbvspace[1]
  \Huge
  {\nbtitlestretch\huge
    METHEXIS}

  \nbvspace[1]
  \normalsize
  A PROPOSAL FOR A NEW APPROACH TO TECHNOLOGY\\
  
  \nbvspace[1]
  \small BY\\
  \Large THADDEUS HUGHES\\

  \nbvspace[2]

  \nbvspace[3]
  \normalsize

  \large
  FLOATING IN THE WILD \\
  \small JULY 2021 \\
\end{center}

\newgeometry{top=0.75in,bottom=0.75in,right=0.75in,left=1.25in}
\raggedbottom
%\tableofcontents

\section{Some Critiques}

Bear in mind that these critiques are cries for help, rediscovery, and further analysis. Critiques are not prescriptive- they are flags for the doctor to come in. Yelling in pain is important, but is by no means curative.

%% I'm not sure I feel comfortable giving this a platform. I should probably scrub this.
%% "In order to get our message before the public with some chance of making a lasting impression, we've had to kill people."

\subsection{Selections from ISAIF}

The source of these first critiques is a most unfortunate one. I do not uphold these writings as from a genius, but as from someone who was abused by society yet intelligent enough to articulate why and how. The lack of constructive action that has come about from his actions and words speaks to their lack of serious substance, and perhaps, their hyperbolic nature. I believe he most certainly has overplayed his hand. I pray that God have mercy on his soul. But I would be amiss if I did not present his empassioned words as exemplars of the thoughts that are racing through many people's minds- even if on an unconscious level, or in different words.

\begin{quote}
Human beings have a need for something that we will call the \textit{power process}... which has four elements. The three most clear-cut of these we call goal, effort and attainment of goal... consider the hypothetical case of a man who can have anything he wants just by wishing for it. Such a man has power, but he will develop serious psychological problems. At first he will have a lot of fun, by by and by he will become acutely boared and demoralized... History shows that leisured aristocracies tend to become decadent. This is not true of fighting aristocracies that have to struggle to maintain their power... Power is not enough. One must have goals toward which to exercise one's power.
\end{quote}

Kacynski also puts forth the important distinction of \textit{surrogate} activities.

\begin{quote}
When people do not have to exert themselves to satisfy their physical needs they often set up artificial goals for themselves. In many cases they pursue these goals with the same energy and emotional involvement that they otherwise would have put in to the search for physical necessities... We use the term "surrogate activity" to designate an activity that is direceted toward an artificial goal that people set up for themselves merely in order to have some goal to work toward, or let us say, merely for the qake of the fulfillment that they get from pursuing the goal.

...

In modern industrial society only minimal effort is necessary to satisfy one's physical needs. It is enough to go through a training program to acquire some petty technical skill, then come to work on time and exert the very modest effort needed to hold a job. The only requirements area moderate amount of intelligence and, most of all, simple obedience. If one has those, society takes care of one from cradle to grave.

...

For many if not most people, surrogate activities are less satisfying than the pursuit of real goals (that is, goals that people would want to attain even if their need for the power process were already fulfilled). One indication of this is the fact that, in many or most cases, people who are deeply involved in surrogate activities are never satisfied, never at rest. Thus the money-maker constantly strives for more and more wealth. The scientist no sooner solves one problem than he moves onto the next. The long-distance runner drives himself to run always farther and faster.
\end{quote}

\section{A Survey of Teachings}

\subsection{E.F. Schumacher's Small is Beautiful}

E.F. Schumacher, author of /textit{Small Is Beautiful}, was statistician, economist, a dedicated atheist, but converted to Catholicism. He also helped found the Intermediate Technology Development Group which today operates as Practical Action. \href{https://www.religion-online.org/article/small-is-beautiful-and-so-is-rome-surprising-faith-of-e-f-schumacher/}{Charles Fager writes}:

\begin{quote}
... one of the most frequently cited chapters [of \textit{Small is Beautiful}], ``Buddhist Economics", almost made it appear as if he were deeply involved in Eastern religions. But wasn't this chapter, I inquired, really more informed by the Catholic writings and thinkers he mentioned so frequently elsewhere in the book - the papal encyclicals, Newman, Gilson and, above all, Thomas Aquinas?
\end{quote}

Indeed, Schumacher even admits immediately before the iconic chapter, "The choice of Buddhism for this purpose is purely incidental; the techings of Christianity, Islam, or Judaism could have been used just as well." [p. 52]

%% Side takeaway from this article, worth thinking more about: Buddhism for Schumacher was an entry point back into Christianity from Atheism insofar as it let him cast aside his modern western antichristian biases.
%% And if I were to go around England passing myself off as a Buddhist, then I would also be thinking that everyone else around me was stupid, because they'd all got the wrong religion. They're all unenlightened, while I'm the one who has the truth. And there are many people in the West these days going around acting like quasi-Orientals, with dreadful results.

Schumacher emphasizes that the Christian message is a very profound and quite specific one rather than some vague feel-good notion which could manifest in a broad variety of teachings.

\begin{quote}
I foudn that in England almost any old nonsense was being written and passed off as Christianity, even by bishops. And so I finally decided that the Catholic tradition was the one where I felt most at home, and where the essentials of Christianity were best preserved.
\end{quote}

Schumacher is quite adamant that Catholicism must be a \textit{social} affair, that it must manifest in community and continuity. Dr. John Coleman, professor of religion and society at the Jesuit School of Theology in Berkeley, stated,

\begin{quote}
By this I mean the stream of Catholic thought that build on Thomistic principles, as particularly reapplied in the work of Jacques Martain. Its adherents stressed that human institutions ought to be manageable in size, respectful of the human scale, and sanely run so that they did not damage the people involved in them.
\end{quote}

% Schumacher extends the approach that the Family and Church exist as mediators and outside of the government- to technology. Would like to find direct exemplats of this.

Coleman continues:

\begin{quote}
The problem with social Catholicism, is that it has been mainly enunciated rather than acted upon.
\end{quote}









%%%

\begin{quote}
The Buddhist point of view takes the function of work to be at least threefold: to give man a chance to utilise and develop his faculties; to enable him to overcome his ego-centredness by joining with other people in a common task; and to bring forth the goods and services needed for a becoming existence.
\end{quote} [p.54-55]

\begin{quote}
The craftsman himself can always, if allowed to, draw the delicate distinction between the machine and the tool. The carpet loom is a tool, a contrivance for holding warp threads at a stretch for the pile to be woven roudn them by the craftsmen's fingers; but the power loom is a machine, and its significance as a destroyer of culture lies in the fact that it does the essentially human part of the work.
\end{quote} - Ananda Coomaraswamy [p.55]

This sounds good at first- but there is something amiss about this. I don't think drawing a perfect line is important, but understanding the principle at play- and how the line might not be between material goods- is. CNC Mills and 3D printers, for example, certainly can be either a \textit{tool} or \textit{machine}. If they are employed to pump out parts en masse, they are a \textit{machine} to those tending them. If they are used by a designer to make their designs manifest (some of which would not be feasible with the use of manual machines), they are most certainly a \textit{tool} as the human aspect of design is still in play. The step from one-off to mass production- from an expression of creativity to brainless toil- this is what can turn a \textit{tool} into a \textit{machine}.

%%% OK, maybe you should have signed up for that icon-painting class.

\begin{quote}
The very start of Buddhist economic planning would be a planning for full employment, and the primary purpose of this would in fact be employment for everyone who needs an "outside" job: it would not be the maximisation of employment nor the maximisation of production. Women, on the whole, do not need an "outside" job, and the large-scale employment of women in offices or factories would be considered a sign of serious economic failure. In particular, to let mothers of young children work in factories while the children run wild would be as uneconomic in the eyes of a Buddhist economist as the employment of a skilled worker as a soldier in the eyes of a modern economist.
\end{quote}

This phrase about women may sound misogynist (bear in mind, this work was written in 1973). But the underlying sentiment is anything but. Schumacher has claimed women to be the most skilled labor of all and that their utilzation in a factory environment is downright inefficient- no, insulting, to women, children, and society.

To the true-libertarian-types out there, Schumacher makes it clear:

\begin{quote}
While the materialist is mainly interested in goods, the Buddhist is mainly interested in liberation... For the modern economist this is very difficult to understand. He is used to measuring the "standard of living" by the amoutn of annual consumption, assuming all the time athat a man who consumes more is "better off" than a man who consumes less. A Buddhist economist would consider this approach excessively irrational: since consumption is merely a means to human well-being, the aim should be to obtain the maximum of well-being with the minimum of consumption.
\end{quote} [p.57]

\begin{quote}
From teh point of view of Buddhist economics, therefore, production from local resources for local needs is the most rational way of economic life, while dependence on imports from afar and the consequent need for produce for export to unknown and distanct peoples is highly uneconomic and justifiable only in eceptional cases and on a small scale.
\end{quote} [p.59]

\begin{quote}
Is it not true that the great prosperity of Germany became possible only through this unification? All the same, the German-speaking Swiss and the German-speaking Austrians, who did not join, did just as well economically, and if we make a list of all the most properous countries in the world, we find that most of them are very small; whereas a list of all the biggest countries in the world shows most of them to be very poor indeed.
\end{quote} [p.64]

\begin{quote}
there always appears to be a need for at least two things simultaneously, which on the face of it, seem to be incompatible and to exclude one another. We always need both freedom and order. We need the freedom of lots and lots of small, autonomous units, and, at the same time, the orderliness of large-scale, possibly global, unity and coordination. When it comes to action, we obviously need small units, because action is a highly personal affair, and one cannot be in touch with more than a very limited number of persons at any one time. But when it comes to the world of ideas, to principles or to ethics, to the indivisibility of peace and also of ecology, we need to recognise the unity of mankind and base our actions upon this recognition.
\end{quote} [p.65]

\begin{quote}
A highly developed transport and communications system has one immensely powerful effect: it makes people footloose. Millions of people start moving about, deserting the rural areas and the smaller towns to follow the city lights, to go to the big city, causing a patholigical growth. Take the country in which all this is perhaps most exemplified - the United States... They freely talk abotu the polarisation of the population of the United States into three immense megalopolitan areas: [Boston to Washington, Chicago, and San Fransisco to San Diego]; the rest of the country being left practically empty; deserted provincial towns, and the land cultivated with vast tractors, combine harvesters, and immense amounts of chemicals. If this is somebody's conception of the future of the United States, it is hardly a future worth having.
\end{quote}[p.68]

The recent boom of work-from-home and sprawl out of cities seems to have debunked this prediction... but it is quite arguable that this is only due to a cultural realization that this future isn't tenable.


This text is not aimed at pontificating or producing particularly new profound ideas. The aim of this text is to grapple with the existing ideas and begin putting them into practice.

\section{Principles}

When we consider principles, we must consider the various stages that a technology goes through: development, proliferation, usage, and obsolecense. For a given technology, its particular units may go through these phases at different rates- some being thrown out before new regions recieve the technology.

\subsection{Subsidiarity}

\subsection{Keeping Humanity in the Picture}

\section{A Survey of the Modern Condition}

\section{Projects}



\end{document}
