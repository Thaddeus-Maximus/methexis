\documentclass[letterpaper]{article}

\usepackage{attrib}

\usepackage{fontspec}
\setmainfont{CMU Serif}
%\setsansfont{CMU Sans Serif}
\newfontfamily{\greekfont}{CMU Serif}
%\newfontfamily{\greekfontsf}{CMU Sans Serif}

\usepackage{polyglossia}
\setmainlanguage{english}
\setotherlanguage{greek}

\usepackage{hyperref}

%\usepackage{slantsc,lmodern}

\usepackage[paper=letterpaper,margin=1in]{geometry}

\begin{document}

\clearpage
%% temporary titles
% command to provide stretchy vertical space in proportion
\newcommand\nbvspace[1][3]{\vspace*{\stretch{#1}}}
% allow some slack to avoid under/overfull boxes
\newcommand\nbstretchyspace{\spaceskip0.5em plus 0.25em minus 0.25em}
% To improve spacing on titlepages
\newcommand{\nbtitlestretch}{\spaceskip0.6em}
\pagestyle{plain}
\begin{center}
  \bfseries
  \nbvspace[1]
  \Huge
  {\nbtitlestretch\huge
    METHEXIS | ΜΕΘΕΞΙΣ}

  \nbvspace[1]
  \normalsize
  REDISCOVERING AND APPLYING THE ANCIENT AND CATHOLIC APPROACH TO TECHNOLOGY\\

  \nbvspace[1]
  \small BY\\
  \Large THADDEUS HUGHES\\

  \nbvspace[2]

  \nbvspace[3]
  \normalsize

  \large
  FLOATING IN THE WILD \\
  \small JULY 2021 \\
\end{center}

\newgeometry{top=0.75in,bottom=0.75in,right=0.75in,left=1.25in}
\raggedbottom
\tableofcontents

\section{Motivation}

The modern (western) world works predominantly on enlightenment principles. That individuals exist as atomic without any sort of rooting in their pasts that gives them values on which to stand. That morality can and ought to be continuously rediscovered, rather than passed down. That `innovation' and `scientific thinking' consistently leads in a profitable path.

These ideas are not rooted in charity and humility, the cornerstones of true wisdom. Rather they are rooted in a presumption that what (we think) we know now is the best- the tip of the spear, the `cutting edge'. One needs only a cursory glance at history and the decline of civilizations to realise this is nowhere near a guaruntee, and that true wisdom is hard to come by. I think even a number of enlightenment thinkers, such as the influential John Locke, understood this. One cannot read the works of Locke without still seeing a great deal of ancient wisdom that hadn't yet shaken off- indeed many regard him to be quite inconsistent because of this, and consistency is key within an enlightened thinker's framework.

\begin{quote}
  The men who conceived the idea that ``morality is bunk" did so with a mind well-stocked with moral ideas. But the minds of the third and fourth generations are no longer well-stocked with such ideas: they are well-stocked with ideas conceived in the nineteenth century, namely, that ``morality is bunk," that everything that appears to be ``higher" is really nothing but something quite mean and vulgar.
  \attrib{\textit{SIB}, Schumacher 1973} % TODO: PAGE #
\end{quote}[SIB]

In the present age there is a longing to rediscover this discarded Western heritage, which is at its core a Christian one. When one comprehends that the true underlying value of a society is its moral nature, and that its development of technology reflects and reinforces it, one can see with clear and fresh eyes first that ancient civilizations had virtue, then appreciate their understanding of virtue, and finally to take this eternally valid paradigm and apply it to ancient life.

\begin{quote}
  All this lyrical stuff about entering the Aquarian Age and reaching a new level of consciousness and takign the next step in evolution is nonsense. Much of it is a sort of delusion of grandeur, the kind of thing you hear from people in the loony bin. what I'm struggling to do is help recapture something our ancestors had. If we can just regain the consciousness the West had before the Cartesian Revolution, which I call the Second Fall of Man, then we'll be getting somewhere.

  \attrib{\textit{SIB+SIR}, Schumacher 1977}
\end{quote}

As Leo XIII writes in \textit{Rerum Novarum},

\begin{quote}
  From contemplation of this divine Model [of Jesus Christ], it is more easy to understand that the true worth and nobility of man lie in his moral qualities, that is, in virtue.
  \attrib{\textit{RN} 24, Leo XIII 1891}
\end{quote}

\begin{quote}
  If human society is to be healed now, in no other way can it be healed save by a return to Christian life and Christian institutions. When a society is perishing, the wholesome advice to give to those who would restore it is to call it to the principles from which it sprang; for the purpose and perfection of an association is to aim at and to attain that for which it is formed, and its efforts should be put in motion and inspired by the end and object which originally gave it being. Hence, to fall away from its primal constitution implies disease; to go back to it, recovery.
  \attrib{\textit{RN} 27, Leo XIII 1891}
\end{quote}

\section{Metaphysics}

\subsection{The Tree of Knowledge}

There is an endless drumbeat from institutions of learning small and large that we need more technologists (engineers, scientists, the like), with little stopping to ask what they should make. Lip service is paid that they ought to `better humanity', but at the end of the day, this seems to boil down to following market forces. While some might say `holistic' education is important, the reality is more severe: \textit{integrated} education is important, along with a recognition that all technologies are built \textit{from} the created firmament, and ought to exist \textit{for} created life.

\begin{quote}
What do I miss, as a human being, if I have never heard of the Second Law of Thermodynamics? The answer is: Nothing. And what do I miss by not knowing Shakespeare? Unless I get my understanding from another source, I simply miss my life.
\attrib{\textit{SIB} 87, Schumacher 1973}
\end{quote}

Or, as the stoics might say, ``the unexamined life is not worth living". The Christian would simply point to the essential teaching uttered over and over that a life not lived through Jesus Christ is no life at all. If we build wonderful technologies that reduce material sufferring, but we do it at the cost of even one less soul for our Lord, it is for naught.

Indeed to harken back to Eden,

\begin{quote}
  Sorrow is knowledge; they who know the most

  Must mourn the deepest o'er the fatal truth,

  The Tree of Knowledge is not that of Life.
  \attrib{\textit{Manfred}, George Gordon Byron} % TODO: year, page #
\end{quote}

\begin{quote}
  Education cannot help us along as it accords no place to metaphysics. Whether the subjects taught are subjects of science or of the humanities, if the teaching does not lead to a clarification of metaphysics, that is to say, of our fundamental convictions, it cannot educate a man, and consequently, is of no value to society.
  \attrib{\textit{SIB} 93, Schumacher 1973}
\end{quote}

This might suggest that we need to have endless `philosophy of <discipline>' courses- this is absolutely not so! It merely means that the learned skills must constantly be pointing back to underlying truths. This will have a secondary benefit in that knowledge will be linked together, thus being easier to remember and recall.

As Joel Barstad puts it,

\begin{quote}
  I added an insistent thirst to overcome certain conventional and traditional oppositions, among them the separation of intelligence from sanctity, of study from worship, of the liberal arts from the arts of subsistence, of the speculative from the practical and creative. How could I hope to know the Word without bowing before Him in worship and them giving him a birth in the materiality of my life?
\end{quote} [CHP, p. 3]

\subsection{Soul and Body}

There are two main heresies as regards the relationship between soul and body: one is to accept the body, and to reject the soul. This is \textit{materialism}- all that matters (or even, is) is the material realm we can experience with our five senses. Over the past century or more, this heresy has gained significant traction, and is the basis of many influential ideologies, most notably Communism. The word itself has a negative connotation in our society, associated with accumulation of goods, but this is not quite broad enough- accumulation of mere sensory experiences- another form of hedonism: this too is materialist.

The other heresy is in our time is not as prevalent, although in the early centuries A.D. was quite the rage, is the \textit{gnostic} heresy: that material things are flawed, evil, or are in some sense inauthentic or a distraction.

The orthodox Christian understanding has always been one of dualism: there exists both a spiritual realm and a physical realm- they interplay, and both are important places not to be neglected. The field of \textit{Theology of the Body} has borne great fruit towards understanding that the soul and body are not doomed to conflict.

\subsection{The Aim of Economics (οίκονέμoμαι) is the Household (οίκος)}

``Economics" comes from the greek word ``οίκονέμoμαι" which means ``household management". And even in this, the household still has an aim: for the benefit of its members both materially and spiritually. The essential difference between managing a household and managing oneself is the \textit{social} aspect of it.

Some have confused the aim of economics to be that of acquiring money. Not so! In fact, in one's reading of Aristotle, we find this original meaning juxtaposed to other means of wealth-getting.

\begin{quote}
  There are two sorts of wealth-getting as I have said; one is part of household management, the other is retail trade: the former necessary and honourable, which that which consists in exchange is justly censured; for it is unnatural, and a mode by which men gain from one another. The most hated sort, and with the greatest reason, is usury, which makes a gain out of money itself, and not from the natural object of it. For money was to be used in exchange, but not to increase at interest. And this term interest [τοκος], which means the birth of money from money, is applied to the breeding of money because the offspring resembles the parent. Wherefore of all modes of getting wealth this is the most unnatural.
  \attrib{\textit{Politics} Bk.1 Ch.10, Aristotle}
\end{quote}

\section{Principles}

\subsection{Subsidiarity}

\begin{quote}
  Neither must it be supposed that the solicitude of the Church is so preoccupied with the spiritual concerns of her children as to neglect their temporal and earthly interests. Her dsire is that the poor, for example, should rise above poverty and wretchedness, and better their condition in life; and for this she makes a strong endeavor.
\end{quote} [para. 28, Rerum Novarum]

Even in this, Leo XIII uses very nuanced language. It would be one thing to say "the poor should not be wretched". It is another to say that the "poor should rise above poverty... and better their condition.".  It is quite clear that the betterment of their condition is not to be imposed upon them externally (even if it is their desire) but is to be derived from their own works.

\subsection{Non-alienation}

\begin{quote}
  It is my experiene that it is rather more difficult to racpture directness and simplicity than to advance in the direction of ever more sophistication and complexity. \textbf{Any third-rate engineer or researcher can increase complexity; but it takes a certain flair of real insight to make things simple again. And this insight does not come easily to people who have allowed themselves to become alienated from real, productive work} and from the self-balancing system of nature, which never fails to recognise measure and limitation. Any activity which fails to recognise a self-limiting principle is of the devil.
  \attrib{\textit{SIB} 154, Schumacher 1973}
\end{quote}

\subsection{Temperance}

The developed world is technologically \textit{gluttonous}. This term has connotations with food, the dominant source of energy for the ancients, but there is no reason we cannot apply it to the ever-growing consumption of disposable goods, electricity, and the like. In fact, since now our predominant source of energy (which powers our predominant source of computational power) is not from food, it makes good sense to apply the principle of gluttony versus temperance to these created and refined resources.

A typical (not all) environmentalist response to the growing demand for electricity, which is predominantly provided by fossil fuels, is predominantly that we should seek to accomplish the same ends and levels of consumption with less inputs (i.e. increased efficiency) and forms of energy generation which do not pollute as much. Many keen will advise us to cease the usage of disposable one-use products in favor of durable (or at least compostable) products.

But this is not a problem with technological roots; in many cases it is a manifestation of a glut. We must ask ourselves: is our consumption (at least in the non-destitute of the developed world) to a point of nourishment, or a point of obesity?

If an obese person wishes to obtain for themselves the virtue of temperance, how much more would it be gained by eliminating sweetened drinks altogether from their diet rather than merely switching to `sugar-free' sodas? These `diet' options in reality, disrupt the body's metabolism and do not help the person shake their sweet tooth. If one does not obtain virtue by using replacement products which are more difficult to produce, how will switching from `dirty' energy sources to `clean' energy sources grant us the virtue of temperance?

Return to pre-industrial levels of energy consumption (a sort of \textit{technological asceticism} may not be necessary, but reconsideration of our lives to see that we are not gluttonous in our usage of energy and technology is. Many have remarked on the need for and benefits of \textit{fasting} from technology. Like a fast from food, such fasting would allow us to bring our passions and desires for technology in line with what is truly necessary for our subsistence, and to make clear how these things place stumbling blocks for the wellness of our souls. 

For one unfamiliar with the practice of fasting, \href{http://rutgersnb.occministries.org/wp-content/uploads/2015/07/St.-Basil-the-Great%E2%80%99s-First-Homily-on-Fasting.pdf}{St. Basil the Great's First Homily on Fasting} is an excellent introduction to this powerful practice. 

\subsection{Fraternity}

\begin{quote}
  But the Church, with Jesus Christ as her Master and Guide, aims higher still. She lays down precepts yet more perfect, and tries to bind class to class in friendliness and good feeling.
  \attrib{\textit{RN} 21, Leo XIII 1891}
\end{quote}

\begin{quote}
  There is no intermediary more powerful than religion in drawing the rich an the working class together, by reminding each of its duties to the other, and especially of the obligations of justice.
  \attrib{\textit{RN} 19, Leo XIII 1891}
\end{quote}

\section{Projects}

\subsection{META-STEM}

Many might be familiar with the modern acronym of ``STEM" (Science, Technology, Engineering, Mathematics). In some regards, this is a useful collection of loosely related disciplines. In other regards, drawing a line about these and packaging them together suggests that these fields have more to do with each other than they do with arts or philosophy. In practice, this is of course folly- many engineers pull loosely from these other disciplines while working closely with graphic design artists. Are `social sciences' sciences? Aren't they better connected to theology, morality, and ethics?

There's been a movement to ``put the arts in STEM" and turn the acronym to STEAM.

While this is, in some regards, a noble effort and perhaps a slight remedy to the underlying metaphysical mindset (as Dotstoevsky quoth, ``Beauty will save the world"), it is still too shortsighted.

I'd propose a new acronym, or a prefix: META-STEM. The META standing for Manual labor, Ethics, Theology, and Arts.

If your stomach churns at the word ``Theology" you could replace it with ``Teleology". These are, essentially, the same thing: the study of the divine is the study of our end goals.

\subsection{Proliferation of Manufacturing Technologies}

The essential problem that must be wrestled with is this: the short-run monetary costs of utilizing distributed, ethical manufacturing will always be higher than centralized, unethical manufacturing.

\subsection{Technological Asceticism}

\end{document}
