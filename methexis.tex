\documentclass[10pt,letterpaper,openany]{book}
\usepackage[dvipsnames]{xcolor}
\usepackage{outlines}
\usepackage{amsmath}
\usepackage{tikz}
\usepackage{circuitikz}
\usepackage{svg}

\usepackage{hyperref}
\usepackage{enumitem}
\usepackage{caption}
\usepackage{subcaption}
\DeclareCaptionOptionNoValue{centering}{\centering} % Make sure everything is centered in subs
\captionsetup[sub]{centering}

\usepackage{multirow}
\usepackage{cancel}
\usepackage{float}

\usepackage{parskip}

\usepackage{slantsc,lmodern}

\usepackage{pgfplotstable,booktabs}
\usepackage{textcomp}
\usepackage{gensymb}

\usepackage{paralist}

\usepackage[paper=letterpaper,margin=1in]{geometry}

\usepackage{etoolbox}

\usepackage{imakeidx}
\makeindex

\usepackage{mdframed}
\mdfdefinestyle{qbox}{
  frametitlebackgroundcolor   =black!15,
    frametitlerule            =false,
    backgroundcolor     = lightgray,
    innertopmargin      =\topskip,
    innerbottommargin   =\topskip,
}
\newmdenv[style=qbox]{qbox}

%\addtolength{\oddsidemargin}{-.875in}
%\addtolength{\evensidemargin}{-.875in}
%\addtolength{\textwidth}{1.75in}

%\addtolength{\topmargin}{-.875in}
%\addtolength{\textheight}{1.75in}

% Centers all the floats
\makeatletter
\g@addto@macro\@floatboxreset\centering
\makeatother

\newcommand{\volume}{{\ooalign{\hfil$V$\hfil\cr\kern0.08em--\hfil\cr}}}

\newtheorem{theorem}{Theorem}

\begin{document}

\clearpage
%% temporary titles
% command to provide stretchy vertical space in proportion
\newcommand\nbvspace[1][3]{\vspace*{\stretch{#1}}}
% allow some slack to avoid under/overfull boxes
\newcommand\nbstretchyspace{\spaceskip0.5em plus 0.25em minus 0.25em}
% To improve spacing on titlepages
\newcommand{\nbtitlestretch}{\spaceskip0.6em}
\pagestyle{plain}
\begin{center}
  \bfseries
  \nbvspace[1]
  \Huge
  {\nbtitlestretch\huge
    METHEXIS}

  \nbvspace[1]
  \normalsize
  A PROPOSAL FOR A NEW APPROACH TO TECHNOLOGY\\
  
  \nbvspace[1]
  \small BY\\
  \Large THADDEUS HUGHES\\

  \nbvspace[2]

  \nbvspace[3]
  \normalsize

  \large
  FLOATING IN THE WILD \\
  \small JULY 2021 \\
\end{center}

\newgeometry{top=0.75in,bottom=0.75in,right=0.75in,left=1.25in}
\raggedbottom
%\tableofcontents

\section{A Survey of the Modern Condition}

\section{A Survey of Teachings}

\subsection{E.F. Schumacher's Small is Beautiful}

E.F. Schumacher, author of /textit{Small Is Beautiful}, was statistician, economist, a dedicated atheist, but converted to Catholicism. He also helped found the Intermediate Technology Development Group which today operates as Practical Action. \href{https://www.religion-online.org/article/small-is-beautiful-and-so-is-rome-surprising-faith-of-e-f-schumacher/}{Charles Fager writes}:

\begin{quote}
... one of the most frequently cited chapters [of \textit{Small is Beautiful}], ``Buddhist Economics", almost made it appear as if he were deeply involved in Eastern religions. But wasn't this chapter, I inquired, really more informed by the Catholic writings and thinkers he mentioned so frequently elsewhere in the book - the papal encyclicals, Newman, Gilson and, above all, Thomas Aquinas?
\end{quote}

%% Side takeaway from this article, worth thinking more about: Buddhism for Schumacher was an entry point back into Christianity from Atheism insofar as it let him cast aside his modern western antichristian biases.
%% And if I were to go around England passing myself off as a Buddhist, then I would also be thinking that everyone else around me was stupid, because they'd all got the wrong religion. They're all unenlightened, while I'm the one who has the truth. And there are many people in the West these days going around acting like quasi-Orientals, with dreadful results.

Schumacher emphasizes that the Christian message is a very profound and quite specific one rather than some vague feel-good notion which could manifest in a broad variety of teachings.

\begin{quote}
I foudn that in England almost any old nonsense was being written and passed off as Christianity, even by bishops. And so I finally decided that the Catholic tradition was the one where I felt most at home, and where the essentials of Christianity were best preserved.
\end{quote}

Schumacher is quite adamant that Catholicism must be a \textit{social} affair, that it must manifest in community and continuity. Dr. John Coleman, professor of religion and society at the Jesuit School of Theology in Berkeley, stated,

\begin{quote}
By this I mean the stream of Catholic thought that build on Thomistic principles, as particularly reapplied in the work of Jacques Martain. Its adherents stressed that human institutions ought to be manageable in size, respectful of the human scale, and sanely run so that they did not damage the people involved in them.
\end{quote}

% Schumacher extends the approach that the Family and Church exist as mediators and outside of the government- to technology. Would like to find direct exemplats of this.

Coleman continues:

\begin{quote}
The problem with social Catholicism, is that it has been mainly enunciated rather than acted upon.
\end{quote}

This text is not aimed at pontificating or producing particularly new profound ideas. The aim of this text is to grapple with the existing ideas and begin putting them into practice.

\section{Principles}

When we consider principles, we must consider the various stages that a technology goes through: development, proliferation, usage, and obsolecense. For a given technology, its particular units may go through these phases at different rates- some being thrown out before new regions recieve the technology.

\subsection{Subsidiarity}

\subsection{Keeping Humanity in the Picture}

\section{Projects}



\end{document}
