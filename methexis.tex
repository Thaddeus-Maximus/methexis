\documentclass[letterpaper]{article}


\usepackage{fontspec}
\setmainfont{CMU Serif}
\setsansfont{CMU Sans Serif}
\newfontfamily{\greekfont}{CMU Serif}
\newfontfamily{\greekfontsf}{CMU Sans Serif}

\usepackage{polyglossia}
\setmainlanguage{english}
\setotherlanguage{greek}

\usepackage{hyperref}

%\usepackage{slantsc,lmodern}

\usepackage[paper=letterpaper,margin=1in]{geometry}

\begin{document}

\clearpage
%% temporary titles
% command to provide stretchy vertical space in proportion
\newcommand\nbvspace[1][3]{\vspace*{\stretch{#1}}}
% allow some slack to avoid under/overfull boxes
\newcommand\nbstretchyspace{\spaceskip0.5em plus 0.25em minus 0.25em}
% To improve spacing on titlepages
\newcommand{\nbtitlestretch}{\spaceskip0.6em}
\pagestyle{plain}
\begin{center}
  \bfseries
  \nbvspace[1]
  \Huge
  {\nbtitlestretch\huge
    METHEXIS | ΜΕΘΕΞΙΣ}

  \nbvspace[1]
  \normalsize
  REDISCOVERING THE ANCIENT AND CATHOLIC APPROACH TO TECHNOLOGY\\

  \nbvspace[1]
  \small BY\\
  \Large THADDEUS HUGHES\\

  \nbvspace[2]

  \nbvspace[3]
  \normalsize

  \large
  FLOATING IN THE WILD \\
  \small JULY 2021 \\
\end{center}

\newgeometry{top=0.75in,bottom=0.75in,right=0.75in,left=1.25in}
\raggedbottom
%\tableofcontents

%%% META-STEM
% Morals
% Ethics
% Theology (Teleology?)
% Arts

\section{Motivation}

The modern (western) world works predominantly on enlightenment principles. That individuals exist as atomic without any sort of rooting in their pasts that gives them values on which to stand. That morality can and ought to be continuously rediscovered, rather than passed down. That 'innovation' and 'scientific thinking' consistently leads in a profitable path.

These ideas are not rooted in charity and humility, the cornerstones of true wisdom. Rather they are rooted in a presumption that what (we think) we know now is the best- the tip of the spear, the 'cutting edge'. One needs only a cursory glance at history and the decline of civilizations to realise this is nowhere near a guaruntee, and that true wisdom is hard to come by. I think even a number of enlightenment thinkers, such as the influential John Locke, understood this. One cannot read the works of Locke without still seeing a great deal of ancient wisdom that hadn't yet shaken off- indeed many regard him to be quite inconsistent because of this, and consistency is key within an enlightened thinker's framework.

\begin{quote}
  The men who conceived the idea that "morality is bunk" did so with a mind well-stocked with moral ideas. But the minds of the third and fourth generations are no longer well-stocked with such ideas: they are well-stocked with ideas conceived in the nineteenth century, namely, that "morality is bunk," that everything that appears to be "higher" is really nothing but something quite mean and vulgar.
\end{quote}

In the present age there is a longing to rediscover this discarded Western heritage, which is at its core a Christian one. When one comprehends that the true underlying value of a society is its moral nature, and that its development of technology reflects and reinforces it, one can see with clear and fresh eyes first that ancient civilizations had virtue, then appreciate their understanding of virtue, and finally to take this eternally valid paradigm and apply it to ancient life.

As Leo XIII writes in \textit{Rerum Novarum},

\begin{quote}
  From contemplation of this divine Model [of Jesus Christ], it is more easy to understand that the true worth and nobility of man lie in his moral qualities, that is, in virtue.
\end{quote} [para. 24]

\begin{quote}
  If human society is to be healed now, in no other way can it be healed save by a return to Christian life and Christian institutions. When a society is perishing, the wholesome advice to give to those who would restore it is to call it to the principles from which it sprang; for the purpose and perfection of an association is to aim at and to attain that for which it is formed, and its efforts should be put in motion and inspired by the end and object which originally gave it being. Hence, to fall away from its primal constitution implies disease; to go back to it, recovery.
\end{quote} [para. 27]

\section{Metaphysics}

\subsection{The Tree of Knowledge}

There is an endless drumbeat from institutions of learning small and large that we need more technologists (engineers, scientists, the like), with little stopping to ask what they should make. Lip service is paid that they ought to 'better humanity', but at the end of the day, this seems to boil down to following market forces. While some might say 'holistic' education is important, the reality is more severe: \textit{integrated} education is important, along with a recognition that all technologies are built \textit{from} the created firmament, and ought to exist \textit{for} created life.

\begin{quote}
What do I miss, as a human being, if I have never heard of the Second Law of Thermodynamics? The answer is: Nothing. And what do I miss by not knowing Shakespeare? Unless I get my understanding from another source, I simply miss my life.
\end{quote}[SIB, p. 87]

Or, as the stoics might say, "the unexamined life is not worth living". The Christian would simply point to the essential teaching uttered over and over that a life not lived through Jesus Christ is no life at all. If we build wonderful technologies that reduce material sufferring, but we do it at the cost of even one less soul for our Lord, it is for naught.

Indeed to harken back to our eldest ancestors,

\begin{quote}
  Sorrow is knowledge; they who know the most

  Must mourn the deepest o'er the fatal truth,

  The Tree of Knowledge is not that of Life.
\end{quote}[SIB, p. 91, quoting Byron]

\begin{quote}
  Education cannot help us along as it accords no place to metaphysics. Whether the subjects taught are subjects of science or of the humanities, if the teaching does not lead to a clarification of metaphysics, that is to say, of our fundamental convictions, it cannot educate a man, and consequently, is of no value to society.
\end{quote}[SIB, p.93]

This might suggest that we need to have endless 'philosophy of <discipline>' courses- this is absolutely not so! It merely means that the learned skills must constantly be pointing back to underlying truths. This will have a secondary benefit in that knowledge will be linked together, thus being easier to remember and recall.

As Joel Barstad puts it,

\begin{quote}
  I added an insistent thirst to overcome certain conventional and traditional oppositions, among them the separation of intelligence from sanctity, of study from worship, of the liberal arts from the arts of subsistence, of the speculative from the practical and creative. How could I hope to know the Word without bowing before Him in worship and them giving him a birth in the materiality of my life?
\end{quote} [CHP, p. 3]

\subsection{Soul and Body}

There are two main heresies as regards the relationship between soul and body: one is to accept the body, and to reject the soul. This is \textit{materialism}- all that matters (or even, is) is the material realm we can experience with our five senses. Over the past century or more, this heresy has gained significant traction, and is the basis of many influential ideologies, most notably Communism. The word itself has a negative connotation in our society, associated with accumulation of goods, but this is not quite broad enough- accumulation of mere sensory experiences- another form of hedonism: this too is materialist.

The other heresy is in our time is not as prevalent, although in the early centuries A.D. was quite the rage, is the \textit{gnostic} heresy: that material things are flawed, evil, or are in some sense inauthentic or a distraction.

The orthodox Christian understanding has always been one of dualism: there exists both a spiritual realm and a physical realm- they interplay, and both are important places not to be neglected. The field of \textit{Theology of the Body} has borne great fruit towards understanding that the soul and body are not doomed to conflict.

\begin{quote}
  Life is kept going by divergent problems which have to be "lived" and are solved only in death. Convergent problems on the other hand are man's most useful invention; they do not, as such, exist in reality, but are created by a process of abstraction. When they have been solved, the solution can be written down and passed on to others, who can apply it without needing to reproduce the mental effort necessary to find it. If this were the case with human relations - in family life, economics, politics, education, and so forth - well, I am a loss at how to finish the sentence. There would be no more human relations but only mechanical reactions; life would be a living death. Divergent problems, as it were, force man to strain himself to a level above himself; they demand, and thus provoke the supply of, forces from a higher level, thus bringing love, beauty, goodness and truth into our lives. It is only with the help of these higher forces that the opposites can be reconciled in the living situation.
\end{quote} [SIB, p.97-98]

\subsection{The Aim of Economics (οίκονέμoμαι) is the Household (οίκος)}

"Economics" comes from the greek word "οίκονέμoμαι" which means "household management". And even in this, the household still has an aim: for the benefit of its members both materially and spiritually. The essential difference between managing a household and managing oneself is the \textit{social} aspect of it.

Some have confused the aim of economics to be that of acquiring money. Not so! In fact, in one's reading of Aristotle, we find this original meaning juxtaposed to other means of wealth-getting.

\begin{quote}
  There are two sorts of wealth-getting as I have said; one is part of household management, the other is retail trade: the former necessary and honourable, which that which consists in exchange is justly censured; for it is unnatural, and a mode by which men gain from one another. The most hated sort, and with the greatest reason, is usury, which makes a gain out of money itself, and not from the natural object of it. For money was to be used in exchange, but not to increase at interest. And this term interest [τοκος], which means the birth of money from money, is applied to the breeding of money because the offspring resembles the parent. Wherefore of all modes of getting wealth this is the most unnatural.
\end{quote} [Aristotle's \textit{Politics}, Bk.1 Ch.10]

\section{Principles}

\subsection{Subsidiarity}

\begin{quote}
  Neither must it be supposed that the solicitude of the Church is so preoccupied with the spiritual concerns of her children as to neglect their temporal and earthly interests. Her dsire is that the poor, for example, should rise above poverty and wretchedness, and better their condition in life; and for this she makes a strong endeavor.
\end{quote} [para. 28, Rerum Novarum]

Even in this, Leo XIII uses very nuanced language. It would be one thing to say "the poor should not be wretched". It is another to say that the "poor should rise above poverty... and better their condition.".  It is quite clear that the betterment of their condition is not to be imposed upon them externally (even if it is their desire) but is to be derived from their own works.

\subsection{Non-alienation}

\subsection{Temperance}

The developed world is technologically \textit{gluttonous}. This term has connotations with food, the dominant source of energy for the ancients, but there is no reason we cannot apply it to the ever-growing consumption of disposable goods, electricity, and the like. In fact, since now our predominant source of energy (which powers our predominant source of computational power) is not from food, it makes good sense to apply the principle of gluttony versus temperance to these created and refined resources.

A typical (not all) environmentalist response to the growing demand for electricity, which is predominantly provided by fossil fuels, is predominantly that we should seek to accomplish the same ends and levels of consumption with less inputs (i.e. increased efficiency) and forms of energy generation which do not pollute as much. Many keen will advise us to cease the usage of disposable one-use products in favor of durable (or at least compostable) products.

But this is not a problem with technological roots; in many cases it is a manifestation of a glut. We must ask ourselves: is our consumption (at least in the non-destitute of the developed world) to a point of nourishment, or a point of obesity?

If an obese person wishes to obtain for themselves the virtue of temperance, how much more would it be gained by eliminating sweetened drinks altogether from their diet rather than merely switching to 'sugar-free' sodas? These 'diet' options in reality, disrupt the body's metabolism and do not help the person shake their sweet tooth. If one does not obtain virtue by using replacement products which are more difficult to produce, how will switching from 'dirty' energy sources to 'clean' energy sources grant us the virtue of temperance?

Return to pre-industrial levels of energy consumption (a sort of \textit{technological asceticism} may not be necessary, but reconsideration of our lives to see that we are not gluttonous in our usage of energy and technology is. Many have remarked on the need for and benefits of \textit{fasting} from technology. Like a fast from food, such fasting would allow us to bring our passions and desires for technology in line with what is truly necessary for our subsistence, and to make clear how these things place stumbling blocks for the wellness of our souls. 

For one unfamiliar with the practice of fasting, \href{http://rutgersnb.occministries.org/wp-content/uploads/2015/07/St.-Basil-the-Great%E2%80%99s-First-Homily-on-Fasting.pdf}{St. Basil the Great's First Homily on Fasting} is an excellent introduction to this powerful practice. 

\section{Projects}

\subsection{META-STEM}

Many might be familiar with the modern acronym of "STEM" (Science, Technology, Engineering, Mathematics). In some regards, this is a useful collection of loosely related disciplines. In other regards, drawing a line about these and packaging them together suggests that these fields have more to do with each other than they do with arts or philosophy. In practice, this is of course folly- many engineers pull loosely from these other disciplines while working closely with graphic design artists. Are 'social sciences' sciences? Aren't they better connected to theology, morality, and ethics?

There's been a movement to "put the arts in STEM" and turn the acronym to STEAM.

While this is, in some regards, a noble effort and perhaps a slight remedy to the underlying metaphysical mindset (as Dotstoevsky quoth, "Beauty will save the world"), it is still too shortsighted.

I'd propose a new acronym, or a prefix: META-STEM. The META standing for Manual labor, Ethics, Theology, and Arts.

If your stomach churns at the word "Theology" you could replace it with "Teleology". These are, essentially, the same thing: the study of the divine is the study of our end goals.

\subsection{Proliferation of Manufacturing Technologies}

The essential problem that must be wrestled with is this: the short-run monetary costs of utilizing distributed, ethical manufacturing will always be higher than centralized, unethical manufacturing.



\iffalse

\subsection{Some Critiques}

Bear in mind that these critiques are cries for help, rediscovery, and further analysis. Critiques are not prescriptive- they are flags for the doctor to come in. Yelling in pain is important, but is by no means curative.

%% I'm not sure I feel comfortable giving this a platform. I should probably scrub this.
%% "In order to get our message before the public with some chance of making a lasting impression, we've had to kill people."


\subsubsection{Selections from ISAIF}

The source of these first critiques is a most unfortunate one. I do not uphold these writings as from a genius, but as from someone who was abused by society yet intelligent enough to articulate why and how. The lack of constructive action that has come about from his actions and words speaks to their lack of serious substance, and perhaps, their hyperbolic nature. I believe he most certainly has overplayed his hand. I pray that God have mercy on his soul. But I would be amiss if I did not present his empassioned words as exemplars of the thoughts that are racing through many people's minds- even if on an unconscious level, or in different words.

\begin{quote}
Human beings have a need for something that we will call the \textit{power process}... which has four elements. The three most clear-cut of these we call goal, effort and attainment of goal... consider the hypothetical case of a man who can have anything he wants just by wishing for it. Such a man has power, but he will develop serious psychological problems. At first he will have a lot of fun, by by and by he will become acutely boared and demoralized... History shows that leisured aristocracies tend to become decadent. This is not true of fighting aristocracies that have to struggle to maintain their power... Power is not enough. One must have goals toward which to exercise one's power.
\end{quote}

Kacynski also puts forth the important distinction of \textit{surrogate} activities.

\begin{quote}
When people do not have to exert themselves to satisfy their physical needs they often set up artificial goals for themselves. In many cases they pursue these goals with the same energy and emotional involvement that they otherwise would have put in to the search for physical necessities... We use the term "surrogate activity" to designate an activity that is direceted toward an artificial goal that people set up for themselves merely in order to have some goal to work toward, or let us say, merely for the qake of the fulfillment that they get from pursuing the goal.

...

In modern industrial society only minimal effort is necessary to satisfy one's physical needs. It is enough to go through a training program to acquire some petty technical skill, then come to work on time and exert the very modest effort needed to hold a job. The only requirements area moderate amount of intelligence and, most of all, simple obedience. If one has those, society takes care of one from cradle to grave.

...

For many if not most people, surrogate activities are less satisfying than the pursuit of real goals (that is, goals that people would want to attain even if their need for the power process were already fulfilled). One indication of this is the fact that, in many or most cases, people who are deeply involved in surrogate activities are never satisfied, never at rest. Thus the money-maker constantly strives for more and more wealth. The scientist no sooner solves one problem than he moves onto the next. The long-distance runner drives himself to run always farther and faster.
\end{quote}

\fi

\subsection{A Survey of Teachings}

\subsubsection{E.F. Schumacher's Small is Beautiful}

E.F. Schumacher, author of /textit{Small Is Beautiful}, was statistician, economist, a dedicated atheist, but converted to Catholicism. He also helped found the Intermediate Technology Development Group which today operates as Practical Action. \href{https://www.religion-online.org/article/small-is-beautiful-and-so-is-rome-surprising-faith-of-e-f-schumacher/}{Charles Fager writes}:

\begin{quote}
... one of the most frequently cited sections [of \textit{Small is Beautiful}], ``Buddhist Economics", almost made it appear as if he were deeply involved in Eastern religions. But wasn't this section, I inquired, really more informed by the Catholic writings and thinkers he mentioned so frequently elsewhere in the book - the papal encyclicals, Newman, Gilson and, above all, Thomas Aquinas?
\end{quote}

Indeed, Schumacher even admits immediately before the iconic section, "The choice of Buddhism for this purpose is purely incidental; the techings of Christianity, Islam, or Judaism could have been used just as well." [p. 52]

%% Side takeaway from this article, worth thinking more about: Buddhism for Schumacher was an entry point back into Christianity from Atheism insofar as it let him cast aside his modern western antichristian biases.
%% And if I were to go around England passing myself off as a Buddhist, then I would also be thinking that everyone else around me was stupid, because they'd all got the wrong religion. They're all unenlightened, while I'm the one who has the truth. And there are many people in the West these days going around acting like quasi-Orientals, with dreadful results.

Schumacher emphasizes that the Christian message is a very profound and quite specific one rather than some vague feel-good notion which could manifest in a broad variety of teachings.

\begin{quote}
I foudn that in England almost any old nonsense was being written and passed off as Christianity, even by bishops. And so I finally decided that the Catholic tradition was the one where I felt most at home, and where the essentials of Christianity were best preserved.
\end{quote}

Schumacher is quite adamant that Catholicism must be a \textit{social} affair, that it must manifest in community and continuity. Dr. John Coleman, professor of religion and society at the Jesuit School of Theology in Berkeley, stated,

\begin{quote}
By this I mean the stream of Catholic thought that build on Thomistic principles, as particularly reapplied in the work of Jacques Martain. Its adherents stressed that human institutions ought to be manageable in size, respectful of the human scale, and sanely run so that they did not damage the people involved in them.
\end{quote}

% Schumacher extends the approach that the Family and Church exist as mediators and outside of the government- to technology. Would like to find direct exemplats of this.

Coleman continues:

\begin{quote}
The problem with social Catholicism, is that it has been mainly enunciated rather than acted upon.
\end{quote}









%%%

\begin{quote}
The Buddhist point of view takes the function of work to be at least threefold: to give man a chance to utilise and develop his faculties; to enable him to overcome his ego-centredness by joining with other people in a common task; and to bring forth the goods and services needed for a becoming existence.
\end{quote} [p.54-55]

\begin{quote}
The craftsman himself can always, if allowed to, draw the delicate distinction between the machine and the tool. The carpet loom is a tool, a contrivance for holding warp threads at a stretch for the pile to be woven roudn them by the craftsmen's fingers; but the power loom is a machine, and its significance as a destroyer of culture lies in the fact that it does the essentially human part of the work.
\end{quote} - Ananda Coomaraswamy [p.55]

This sounds good at first- but there is something amiss about this. I don't think drawing a perfect line is important, but understanding the principle at play- and how the line might not be between material goods- is. CNC Mills and 3D printers, for example, certainly can be either a \textit{tool} or \textit{machine}. If they are employed to pump out parts en masse, they are a \textit{machine} to those tending them. If they are used by a designer to make their designs manifest (some of which would not be feasible with the use of manual machines), they are most certainly a \textit{tool} as the human aspect of design is still in play. The step from one-off to mass production- from an expression of creativity to brainless toil- this is what can turn a \textit{tool} into a \textit{machine}.

%%% OK, maybe you should have signed up for that icon-painting class.

\begin{quote}
The very start of Buddhist economic planning would be a planning for full employment, and the primary purpose of this would in fact be employment for everyone who needs an "outside" job: it would not be the maximisation of employment nor the maximisation of production. Women, on the whole, do not need an "outside" job, and the large-scale employment of women in offices or factories would be considered a sign of serious economic failure. In particular, to let mothers of young children work in factories while the children run wild would be as uneconomic in the eyes of a Buddhist economist as the employment of a skilled worker as a soldier in the eyes of a modern economist.
\end{quote}

This phrase about women may sound misogynist (bear in mind, this work was written in 1973). But the underlying sentiment is anything but. Schumacher has claimed women to be the most skilled labor of all and that their utilzation in a factory environment is downright inefficient- no, insulting, to women, children, and society.

To the true-libertarian-types out there, Schumacher makes it clear:

\begin{quote}
While the materialist is mainly interested in goods, the Buddhist is mainly interested in liberation... For the modern economist this is very difficult to understand. He is used to measuring the "standard of living" by the amoutn of annual consumption, assuming all the time athat a man who consumes more is "better off" than a man who consumes less. A Buddhist economist would consider this approach excessively irrational: since consumption is merely a means to human well-being, the aim should be to obtain the maximum of well-being with the minimum of consumption.
\end{quote} [p.57]

\begin{quote}
From teh point of view of Buddhist economics, therefore, production from local resources for local needs is the most rational way of economic life, while dependence on imports from afar and the consequent need for produce for export to unknown and distanct peoples is highly uneconomic and justifiable only in eceptional cases and on a small scale.
\end{quote} [p.59]

\begin{quote}
Is it not true that the great prosperity of Germany became possible only through this unification? All the same, the German-speaking Swiss and the German-speaking Austrians, who did not join, did just as well economically, and if we make a list of all the most properous countries in the world, we find that most of them are very small; whereas a list of all the biggest countries in the world shows most of them to be very poor indeed.
\end{quote} [p.64]

\begin{quote}
there always appears to be a need for at least two things simultaneously, which on the face of it, seem to be incompatible and to exclude one another. We always need both freedom and order. We need the freedom of lots and lots of small, autonomous units, and, at the same time, the orderliness of large-scale, possibly global, unity and coordination. When it comes to action, we obviously need small units, because action is a highly personal affair, and one cannot be in touch with more than a very limited number of persons at any one time. But when it comes to the world of ideas, to principles or to ethics, to the indivisibility of peace and also of ecology, we need to recognise the unity of mankind and base our actions upon this recognition.
\end{quote} [p.65]

\begin{quote}
A highly developed transport and communications system has one immensely powerful effect: it makes people footloose. Millions of people start moving about, deserting the rural areas and the smaller towns to follow the city lights, to go to the big city, causing a patholigical growth. Take the country in which all this is perhaps most exemplified - the United States... They freely talk abotu the polarisation of the population of the United States into three immense megalopolitan areas: [Boston to Washington, Chicago, and San Fransisco to San Diego]; the rest of the country being left practically empty; deserted provincial towns, and the land cultivated with vast tractors, combine harvesters, and immense amounts of chemicals. If this is somebody's conception of the future of the United States, it is hardly a future worth having.
\end{quote}[p.68]

Again the date of this publishing proves prophetic: in the US, Far Western states are the fastest growing- in particular, Nevada and Idaho, while California and most of the eastern seaboard stagnates in growth, and Illinois and New York continue to shed population. This trend predates the Coronavirus pandemic and many of these more populous states' lockdown policies. The reasons cited by many are a sense of urban crowding and decreased quality of life with a vague notion that reconnection to nature and community is better served by a more rural, or at least suburban, lifestyle.

\begin{quote}
If so much reliance is today being placed in the power of education to enable ordinary people to cope with the problems thrown up by scientific and technological progress, then there must be something more to education than Lord Snow suggests [that all men must be educated in technical matters]. Science and engineering produce "know-how"; but "know-how" is nothing by itself; it is a means without an end, a mere potentiality, an unfinished sentence. "Know-how" is no more a culture than a piano is music. Can education help us to finish the sentence, to turn the potentiality into a reality to the benefit o man? To do so, the task of education would be, first and foremost, the transmission of ideas of value, of what to do with our lives. There is no doubt also the need to transmit know-how but this must take second place, for it is obviously somewhat foolhardy to put great powers into the hands of people without making sure that they have a reasonable idea of what to do with them.
\end{quote}[p.81-82]

It's worth noting that these words were written in 1977, and Snow said the Russians were doing much better than the Western world and will have a clear edge. We know how that turned out.

\begin{quote}
When people ask for education, they normally mean something more than mere training, something more than mere knowledge of facts, and something more than mere diversion. Maybe they cannot themselves formulate precisely what they are looking for; but I think that what they are looking for is ideas that would make the world, and thair whole lives, intelligible to them. When a thing is intelligible, you have a sense of participation; when a thing is unintelligible you have a sense of estrangement.
\end{quote}[p.84]

\begin{quote}
  The most striking thing about modern industry is that it requires so much and accomplishes so little. Modern industry seems to be inefficient to a degree that surpasses one's ordinary powers of imagination. Its inefficiency therefore remains unnoticed.
\end{quote} [SIB, p.118]

\begin{quote}
  An industrial system that uses forty percent of the world's primary resources to supply less than six percent of the world's population could be called efficient only if it obtained strikingly successful results in terms of human happiness, well-being, culture, peace, and harmony. I do not need to dwell on the fact that the American system fails to do this.
\end{quote}[SIB, p.119]

The operative word may be strikingly- because supplying 6 percent with 40 is a factor of ten times less efficient than 94 with 60.

\begin{quote}
  The religion of economics promotes an idolatry of rapid change, unaffected by the elementary truism that a change is which is not an unquestionable improvement is a doubtful blessing. The burden of proof is placed on those who take the 'ecological viewpoint': unless \textit{they} can produce evidence of marked injury to man, the change will proceed. Common sense, on the contrary, would suggest that the burden of proof should lie on the man who wants to introduce a change; \textit{he} has to demonstrate that there \textit{cannot} be any damaging consequences. But this would take too much time, and would therefore be uneconomic.
\end{quote}[SIB, p.134]

\begin{quote}
  All changes in a complex mechanism involve some risk and should be undertaken only after careful study of all the facts available. Changes should be made on a small scale first so as to provide a test before they are widely applied. When information is incomplete, changes should stay close to the natural processes which have in their favour the indisputable evidence of having supported life for a very long time.
\end{quote}[SIB, p.135]

While some accuse the Amish of being stuck-up and altogether rejecting of technology, this idea of waiting on evidence is really what they are employing rather than some paleolithic rejection.

\begin{quote}
  Rather less than one-half of this coutnry is, as they say, gainfully occupied, and about one-third of these are actual producers in agriculture, mining, construction, and industry. I do mean \textit{actual production}, not people who tell other people what to do, or account for the past, or plan for the future, or distribute what other people have produced. In other words, rather less than one-sixth of the total population is engaged in actual production; on average, each of them supports five others beside himself, of which two are gainfully employed on things other than real production and three are not gainfully employed. Now, a fully employed person, allowing for holidays, sickness, and other absence, spends about one-fifth of his total time on his job. It follows that the proportion of "total social time" spent on actual production - in the narrow sense in which I am using the term - is roughly, one-fifth of one-third of one-half, i.e. 3 1/2 percent. The other 96 1/2 percent of "total social time" is spent in other ways, including sleeping, eating, watching television, doing jobs that are not directly productive, or just killing time more or less humanely.
\end{quote} [SIB, p.149-150]

\begin{quote}
  [these calculations] quote adequately serve to show what technology has enabled us to do: namely, to reuce the amount of time actually spent on production in its most elementary sense to such a tiny percentage of total social time that it pales into insignificance, that it carries no real weight, let along prestige. When you look at industrial society in this way, you cannot be suprised to find that prestige is carried by those who help fill the other 96 1/2 percent of total social time, primarily the entertainers but also the executors of Parkinson's Law. Infact, one might put the following proposition to students of sociology: "The prestige carried by people in modern industrial society variesi n inverse proportion to their closeness to actual production"
\end{quote} [SIB, p.150]

\begin{quote}
  The whole drift of modern technological development is to reduce [directly productive time] further, asymptotically to zero. Imagine we set ourselves a goal in the opposite direction: to increase it sixfold, to about twenty percent, so that twenty percent of total social time would be used for actually producing things, employing hands and brains and, naturally, excellent tools. An incredible thought! Even children would be allowed to make themselves useful, even old people. At one-sixth of present-day productivity, we should be producing as much as at present. There would be six times as much time for any piece of work we chose to undertake- enough to make really good job of it, to enjoy oneself, to produce real quality, even to make things beautiful. Think of the therapeutic value of real work; think of its educational value. No one would then want to raise the school-leaving age or to lower the retirement age, so as to keep people off the labour market. Everybody would be welcome to lend a hand. Everybody would be admitted to what is now the rarest privilege, the opportunity of working usefully, creatively, with his own hands and brains, in his own time, at his own pace- and with excellent tools. Would this mean an enormous extension of working hours? No, people who work in this way do not know the difference between work and leisure. Unless they sleep or eat or occasionally choose to do nothing at all, they are always agreeable, productively engagaed. Many of the "on-cost jobs" would simply disappear; I leave it to the reader's imagination to identify them. There would be little need for mindless entertainment or other drugs, and unquestionably much less illness.
\end{quote}[SIB, p.151-152]

\begin{quote}
  It is almost like a providential blessing that we, the rich countries, have found it in our heart at least to consider the Third World and to try to mitigate its poverty In spite of the mixture of motives and the persisstence of exploitative practices, I think that this fairly recent development in the outlook of the rich is an honourable one. And it could save us; for the poverty of the poor makes it in any case impossible for them successfully to adopt our technology. Of course they often try to do so, and then have to bearth e most dire consequences in terms of mass unemployment, mass migration into cities, rural decay, and intolerable social tensions. They need in fact, the very kind of thing I am talking about, which we also need: a \textit{different} kind of technology, a technology with a human face, which, instead of making human hands and brains redundant, helps them to become far more productive than they ever have before.
\end{quote}[SIB, p.153]

\begin{quote}
  It is my experiene that it is rather more difficult to racpture directness and simplicity than to advance in the direction of ever more sophistication and complexity. Any third-rate engineer or researcher can increase complexity; but it takes a certain flair of real insight to make things simple again. And this insight does not come easily to people who have allowed themselves to become alienated from real, productive work and from the self-balancing system of nature, which never fails to recognise measure and limitation. Any activity which fails to recognise a self-limiting principle is of the devil.
\end{quote}[SIB, p.154]

\begin{quote}
  Let us admit that the people of the forward stampede, like the devil, have all the best tunes or at least the most popular and familiar tunes. You cannot stand still, they say; standing still means going down; you must go forward; there is nothing wrong with modern technology except that it is as yet incomplete; let us complete it.

  ...

  This is the authentic voice of the forward stampede, which talks in much the same tone as Dostoyevsky's Grand Inquisitor: "Why  have you come to hinder us?"
\end{quote}[SIB, p.155]

\begin{quote}
  Strange to say, the sermon on the mount gives pretty precise instructions on how to construct and outlook that could lead to an Economics of Survival.

  - How blessed are those who know they are poor: the Kingdom of Heaven is theirs.
  - How blessed are the sorrowful; they shall find consolation.
  - How blessed are those of a gentle spirit; they shall have the earth for their possession.
  - How blessed are those who hunger and thirst to see right prevail; they shall be satisfied;
  - How blessed are the peacemakers; God shall call them his sons.

  It may seem daring to connect these beatitudes with matters of technology and economics. But may it not be that we are in trouble precisely because we have failed for so long to make this connection? It is not difficult to discern what these beatitudes may mean for us today:

  - We are poor, not demigods.
  - We have plenty to be sorrowful about, and are not emerging into a golden age.
  - We need a gentle approach, a non-violent spirit, and small is beautiful.
  - We must concern ourselves with justice and see right prevail.
  - And all this, only this, can enable us to become peacemakers.
\end{quote}[SIB, p.156-157]

\begin{quote}
  The home-comers base themselves upon a different picture of man from that which motivates the people of the forward stampede. It would be very superficial to say that the latter believe in "growth" while the former do not. In a sense, everybody believes in growth, and rightly so, because growth is an essential feature of life. The whole point, however, is to give to the idea of growth a qualitative determination; for there are always many things that ought to be growing and many things that ought to be diminishing.
\end{quote}[SIB, p.157]

\begin{quote}
  I think it was the Chinese, before WWII, who calculated that it tooks the work of thirty peasants to keep one man or woman at a university. If that person at the university took a five-year course, by the time he had finished he would have consumed 150 peasant-work-years. How can this be justified? Who has the right to appropriate 150 years of peasant work to keep one person at university for five years, and what do the peasants get back for it? These questions lead us to the parting of the ways: is education to be a "passport to privilege" or is it something which people take upon themselves almost like a monastic vow, a sacred obligation to serve the people? The first road takes the educated young man into the fashionable district of Bombay, which a lot of other highly educated people have already gone and where he can join a mutual admiration society, a "trade union of the privileged," to see to it that his privileges are not reoded by the great messes of his contemporaries who have not been educated. This is one way. The other way would be emarked upon in a different spirit and would lead to a different destination. It would take him back to the people who, after all, directly or indirectly, have paid for his education by 150 peasant-work-years; having consumed the fruits of their work, he would feel in honour bound to return something to them.
\end{quote}[SIB, p.207]

\begin{quote}
  The problem is not new. Leo Tolstoy referred to it when he wrote: "I sit on a man's back, choking him, and making him carry me, and yet assure myself and others that I am very sorry for him and wish to ease his lot be any means possible, except getting off his back."
\end{quote}[SIB, p.207-208]

\begin{quote}
  In those sixty years, a vast increase of inflexibility. Galbraith comments: "Had Ford and his associates [in 1903] decided at any point to shift from gasoline to steam power, the machine shop could have accommodated itself to the change in a few hours." Now, try to change even one screw, it takes that many months.
\end{quote}[SIB, p.211-212]

\begin{quote}
  The greatest deprivation anyone can suffer is to have no chance of looking after himself and making a livelihood.
\end{quote}[SIB, p.219]

\begin{quote}
  Modern industrial society, typified by large-scale organizations, gives far too little thought to it. Managements assume that people work simply for money, for the pay-packet at the end of the week. No doubt, this is true up to a point, but when a worker, asked why he worked only four shifts last week, answers: "Because I couldn't make ends meet on three shifts' wages," everyone is stunned and feels check-mated.
\end{quote}[SIB, p.249]

Such a worker disobeys a typical supply curve which suggests that he would work more if his wages were higher: clearly, if his wages were higher, he would work less!

% What were luxuries to our fathers have become necessities for us.

\begin{quote}
  I would very much prefer to take any interested person on a tour of our forty-five acre, ancient Manor House Estate, interspersed with chemical plants and laboratories, than to laboriously write [an] article which is bound to raise as many questions as it answers.
\end{quote}[Ernst Bader, via SIB, p.280]

% Matthew 6:33

\subsection{Principles}

When we consider principles, we must consider the various stages that a technology goes through: development, proliferation, usage, and obsolecense. For a given technology, its particular units may go through these phases at different rates- some being thrown out before new regions recieve the technology.

\subsection{Subsidiarity}

\subsection{Keeping Humanity in the Picture}

\subsection{A Survey of the Modern Condition}

\section{Projects}



\end{document}
