%% I'm not sure I feel comfortable giving this a platform. I should probably scrub this.
%% "In order to get our message before the public with some chance of making a lasting impression, we've had to kill people."


\subsubsection{Selections from ISAIF}

The source of these first critiques is a most unfortunate one. I do not uphold these writings as from a genius, but as from someone who was abused by society yet intelligent enough to articulate why and how. The lack of constructive action that has come about from his actions and words speaks to their lack of serious substance, and perhaps, their hyperbolic nature. I believe he most certainly has overplayed his hand. I pray that God have mercy on his soul. But I would be amiss if I did not present his empassioned words as exemplars of the thoughts that are racing through many people's minds- even if on an unconscious level, or in different words.

\begin{quote}
Human beings have a need for something that we will call the \textit{power process}... which has four elements. The three most clear-cut of these we call goal, effort and attainment of goal... consider the hypothetical case of a man who can have anything he wants just by wishing for it. Such a man has power, but he will develop serious psychological problems. At first he will have a lot of fun, by by and by he will become acutely boared and demoralized... History shows that leisured aristocracies tend to become decadent. This is not true of fighting aristocracies that have to struggle to maintain their power... Power is not enough. One must have goals toward which to exercise one's power.
\end{quote}

Kacynski also puts forth the important distinction of \textit{surrogate} activities.

\begin{quote}
When people do not have to exert themselves to satisfy their physical needs they often set up artificial goals for themselves. In many cases they pursue these goals with the same energy and emotional involvement that they otherwise would have put in to the search for physical necessities... We use the term "surrogate activity" to designate an activity that is direceted toward an artificial goal that people set up for themselves merely in order to have some goal to work toward, or let us say, merely for the qake of the fulfillment that they get from pursuing the goal.

...

In modern industrial society only minimal effort is necessary to satisfy one's physical needs. It is enough to go through a training program to acquire some petty technical skill, then come to work on time and exert the very modest effort needed to hold a job. The only requirements area moderate amount of intelligence and, most of all, simple obedience. If one has those, society takes care of one from cradle to grave.

...

For many if not most people, surrogate activities are less satisfying than the pursuit of real goals (that is, goals that people would want to attain even if their need for the power process were already fulfilled). One indication of this is the fact that, in many or most cases, people who are deeply involved in surrogate activities are never satisfied, never at rest. Thus the money-maker constantly strives for more and more wealth. The scientist no sooner solves one problem than he moves onto the next. The long-distance runner drives himself to run always farther and faster.
\end{quote}