\begin{quote}
	Nothing that came into being is perfect in itself and complete. If complete it would have the power of action, but because it has its being from what is not, it does not have power of action. That which is perfect in itself is uncaused.
	\attrib{\textit{Ambiguum 7}, St Maximus}
\end{quote}

\begin{quote}
	It is absolutely necessary that everything will cease its willfull movement toward something else when the ultimate beauty that satisfies our desire appears.
	\attrib{\textit{Ambiguum 7}, St Maximus}
\end{quote}

\begin{quote}
	But some say, because the soul exists and subsits after death and the dissolution of the body, the soul was able to exist and to subsist before the body. But their argument is not persuasive. For what one means by origin is not the same as what one means by essence...

	For the soul, after the death of the body, is not simply called soul, but the soul of a human being, indeed the soul of a certain human being... For the body, after its separation from the soul, is not simply called body, but the body of a man, indeed the body of a certain man... for like the soul it has the whole human being predicted of it as part of its species according to condition.

	Therefore the human being is composed of soul and body, for soul and body are indissolubly understood to be parts of the whole human species.
	\attrib{\textit{Ambiguum 7}, St Maximus}
\end{quote}

\begin{quote}
	[The passions of pleasure, grief, desire, fear, and the rest] were not originally created together with human nature, for if they had been they would contribute to the definition of human nature. But following what the eminent Gregory of Nyssa taught, I say that, on account of humanity's fall from perfection, the passions were introduced and attached themselves to the more irrational part of human nature...

	The passions, moreover, become good in those who are spiritually earnest once they have wisely separated them from corporeal objects and used them to gain possession of heavenly things. ...

	The passions become good when they are used by those who \textit{take every thought captive in order to obey Christ} (2 Cor 10:5).
	\attrib{\textit{Ad Thalassium 1}, St Maximus}
\end{quote}

\begin{quote}
	Whoever has participated in this deification through cognizant experience is incapable of reverting from right discernment in truth, once he has achieved this in action, to something else besides, which only pretends to be that same discernment. It is like the eye which, once it has looked upon the sun, cannot mistake it for the moon or any of the other stars in the heavens.
	\attrib{\textit{Ad Thalassium 2}, St Maximus}
\end{quote}

\begin{quote}
	The scriptural Word knows of two kinds of knowledge of divine things. On the one hand, there is relative knowledge, rooted only in reason and ideas, and lacking in the kind of experiential perception of what one knows through active engagement; such relative knowledge is what we use to order our affiars in our present life. On the other hand, there is that truly authentic knowledge, gained only be actual experience, apart from reason and ideas, which provides a total perception of the known object through a participation (μεθεξις) by grace. By this latter knowledge, we attain, in the future state, the supernatural deification (θέωσις) that remains unceasingly in effect. They say that the relative knowledge based on reason and ideas can motivate our desire for the participative knowledge acquired by active engagement. They say, moreover, that this active, experiential knowledge which, by participation, furnishes the direct perception of the object known, can supplant the relative knowledge based on reason and ideas.

	For the sages say that it is impossible for rational knowledge (λόγος) of God to coexist with the direct experience (πειρα) of God, or for conceptual knowledge (νόησις) of God to coexist with immediate perception (αϊσθησις) of God. By ``rational knowledge of God" I mean the use of the analogy of created beings in the intellectual contemplation of God; by ``perception" I mean the experience, through participation, of the supernatural goods; by ``conceptual knowledge" I mean the simple and unitary knowledge of God drawn from created beings. This kind of distinction may be recognized with every other kind of knowledge as well, since the direct ``experience" of a thing suspends rational knowledge of it and direct ``perception" of a thing renders the ``conceptual knowledge" of it useless. By ``experience" (πειρα) I mean that knowledge, based on active engagement, which surpasses all reason. By ``perception" (αϊσθησις) I mean that participation in the known object which manifests itself beyond all conceptualization. This may very well be what the great Apostle is secretly teaching when he says, \textit{As for prophecies, they will pass away; as for tongues, they will cease; as for knowledge, it will disappear} (1 Cor 13:8). Clearly he is referring here to that knowledge which is found in knowledge and ideas.
	\attrib{\textit{Ad Thalassium 60}, St Maximus}
\end{quote}

\begin{quote}
	At the instant when he was created, the first man, by use of his senses, squandered this spiritual capacity [for pleasure] -- the natural desire of the mind for God -- on sensible things. In this, his very first movement, he activated an unnatural pleasure through the medium of the senses. Being, in his providence, concerned for our salvation, God therefore affixed pain (όδύνη) alongside this sensible pleasure (ήδονή) as a kind of punitive faculty, whereby the law of death was wisely implanted in our corporeal nature to curb the foolish mind in its desire to incline unnaturally toward sensible things.
	\attrib{\textit{Ad Thalassium 61}, St Maximus}
\end{quote}