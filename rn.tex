\begin{quote}
  It is no easy matter to define the relative rights and mutual duties of the rich and of the poor, of capital and labor. And the danger lies in this, that crafty agitators are intent on making use of these differences of opinion to pervert men's judgements and to stir up the people to revolt.
  \attrib{\textit{RN} 2, Leo XIII 1891}
\end{quote}

\begin{quote}
  To this must be added that the hiring of labor and the conduct of trade are concentrated in the hands of comparatively few; so that a small number of very rich men have been able to lay upon the teeming masses of the laboring poor a yoke little better than that of slavery itself.
  \attrib{\textit{RN} 3, Leo XIII 1891}
\end{quote}

\begin{quote}
  The socialists... are striving to do away with private property. ... They are, moreover, emphatically unjust, for they would rob the lawful posessor, distort the functions of the State, and create utter confusion in the community.
  \attrib{\textit{RN} 4, Leo XIII 1891}
\end{quote}

\begin{quote}
  Man's needs do not die out, but forever recur; although satisfied today, they demand fresh supplies for tomorrow. Nature accordingly must have given to man a source that is stable and remaining always with him, from which he might look to draw continual supplies.
  \attrib{\textit{RN} 7, Leo XIII 1891}
\end{quote}

\begin{quote}
 The rights here spoken of, belonging to each individual man, are seen in much stronger light when considered in relation to man's social and domestic obligations. In choosing a state of life, it is indisputable that all are at full liberty to follow the counsel of Jesus Christ as to observing virginity, or to bind themselves by the marriage tie.
  \attrib{\textit{RN} 12, Leo XIII 1891}
\end{quote}

\begin{quote}
  The contention, then, that the civil government should at its option intrude into and exercise intimate control over the family and the household is a great and pernicious error. True, if a family finds itself in exceeding distress, utterly deprived of the counsel of friends, and without any prospect of extricating itself, it is right that extreme necessity be met by public aid, since each family is part of the commonwealth.... But the rules of the commonwealth must go no further; here nature bids them stop. Paternal authority can be neighter abolished nor absorbed by the State; for it has the same source as human life itself. ``The child belongs to the father," and is, as it were, the continuoation of the father's personality; and speaking strictly, the child takes its place in civil society, not of its own right, but in its quality as member of the family in which it is born.
  \attrib{\textit{RN} 14, Leo XIII 1891}
\end{quote}

\begin{quote}
  There is no intermediary more powerful than religion in drawing the rich an the working class together, by reminding each of its duties to the other, and especially of the obligations of justice.
  \attrib{\textit{RN} 19, Leo XIII 1891}
\end{quote}

\begin{quote}
  According to natural reason and Christian philosophy, workign for gain is creditable, not shameful, to a man, since it enables him to earn and honorable livelihood; but to misues men as though they were things in the pursuit of gain, or to value them solely for their physical powers - that is truly shameful and inhuman.
  \attrib{\textit{RN} 20, Leo XIII 1891}
\end{quote}

\begin{quote}
  But the Church, with Jesus Christ as her Master and Guide, aims higher still. She lays down precepts yet more perfect, and tries to bind class to class in friendliness and good feeling.
  \attrib{\textit{RN} 21, Leo XIII 1891}
\end{quote}

\begin{quote}
  Riches do not bring freedom from sorrow and are of no avail for eternal happiness, but rather are obstacles.
  \attrib{\textit{RN} 22, Leo XIII 1891}
\end{quote}

\begin{quote}
  ``It is lawful," says St. Thomas Aquinas, ``for a man to hold private property; and it is also necessary for the carrying on of uman existence." But if the question be asked: How must one's possessions be used? - the Church replies without hesistation in the words of the same holy Doctor: ``Man should not consider his material possesssions as his own, but as common to all, so as to share them without hesitation when others are in need. Whence the Apostle with, `Command the rich of the world... to offer with no stint, to approtion largely.'"
  \attrib{\textit{RN} 22, Leo XIII 1891}
\end{quote}

St. Gregory the Great:
\begin{quote}
	He that hath a talent, let him see that he hide it not; he that hath abundance, let him quicken himself to mercy and generosity; he that hath art and skilll, let him do his best to share the use and the utility hereof with his neighbor.
	\attrib{\textit{Hom. in Evang.,} 9, n. 7 (PL 76, 1109B)}
\end{quote}

% Poerty is no disgrace

\begin{quote}
	From contemplation of this divine Model, it is more easy to understand that the true worth and nobility of man lie in his moral qualities, that is, in virtue; that virtue is, moreover, the common inheritance of men, equally within the reach of high and low, rich and poor; and that in virtue, and virtue alone, wherever found, will be followed by the rewards of everlasting happiness.
	\attrib{\textit{RN} 24, Leo XIII 1891}
\end{quote}

\begin{quote}
  If human society is to be healed now, in no other way can it be healed save by a return to Christian life and Christian institutions. When a society is perishing, the wholesome advice to give to those who would restore it is to call it to the principles from which it sprang; for the purpose and perfection of an association is to aim at and to attain that for which it is formed, and its efforts should be put in motion and inspired by the end and object which originally gave it being. Hence, to fall away from its primal constitution implies disease; to go back to it, recovery.
  \attrib{\textit{RN} 27, Leo XIII 1891}
\end{quote}

\begin{quote}
	Neither must it be supposed that the solicitude of the Church is so preoccupied with the spiritual concerns of her children as to neglect their temporal and earthly interests. Her desire is that the poor, for example, should rise above poverty and wretchedness, and better their condition in life; and for this she makes a strong endeavor.
	\attrib{\textit{RN} 28, Leo XIII 1891}
\end{quote}

\begin{quote}
	From this follows the obligation of the cessation from work and labor on Sundays and certain holy days. The rest from labor is not to be understood as mere giving way to idleness; much less must it be an occasion for spending money and for vicious indulgence, as many would have it to be; but it should be rest from labor, hallowed by religion. Rest (combined with religious observances) disposes man to forget for a while the business of his everyday life, to turn his thoughts to things heavenly, and to the worship which he so strictly owes to the eternal Godhead. It is this, above all, which is the reason arid motive of Sunday rest; a rest sanctioned by God's great law of the Ancient Covenant-``Remember thou keep holy the Sabbath day," and taught to the world by His own mysterious ``rest" after the creation of man: ``He rested on the seventh day from all His work which He had done."
	\attrib{\textit{RN} 41, Leo XIII 1891}
\end{quote}

\begin{quote}
	If a workman's wages be sufficient to enable him comfortably to support himself, his wife, and his children, he will find it easy, if he be a sensible man, to practice thrift, and he will not fail, by cutting down expenses, to put by some little savings and thus secure a modest source of income. Nature itself would urge him to this. We have seen that this great labor question cannot be solved save by assuming as a principle that private ownership must be held sacred and inviolable. The law, therefore, should favor ownership, and its policy should be to induce as many as possible of the people to become owners.
	\attrib{\textit{RN} 46, Leo XIII 1891}
\end{quote}

\begin{quote}
	What advantage can it be to a working man to obtain by means of a society material well-being, if he endangers his soul for lack of spiritual food? ``What doth it profit a man, if he gain the whole world and suffer the loss of his soul?" This, as our Lord teaches, is the mark or character that distinguishes the Christian from the heathen. ``After all these things do the heathen seek... Seek ye first the Kingdom of God and His justice: and all these things shall be added unto you."
	\attrib{\textit{RN} 57, Leo XIII 1891}
\end{quote}

