
Return to pre-industrial levels of energy consumption (a sort of \textit{technological asceticism}) may not be necessary, but reconsideration of our lives to see that we are not gluttonous in our usage of energy and technology is. Many have remarked on the need for and benefits of \textit{fasting} from technology. Like a fast from food, such fasting would allow us to bring our passions and desires for technology in line with what is truly necessary for our subsistence, and to make clear how these things place stumbling blocks for the wellness of our souls. 

Fasting is like a palate cleanser, which is why fasting is such powerful remedy. It removes the poor taste for sin we had and allows our inmost appetite for virtue to show forth.

For one unfamiliar with the practice of fasting, \href{http://rutgersnb.occministries.org/wp-content/uploads/2015/07/St.-Basil-the-Great%E2%80%99s-First-Homily-on-Fasting.pdf}{St. Basil the Great's First Homily on Fasting} is an excellent introduction to this powerful practice. 

\begin{quote}
  You would surely agree that the pilot of a merchant ship is better able to safely guide it to port if it is not fully loaded, when it is in excellent condition and light. The ship completely loaded down is sunk by a minor swell in the waters. But the boat that has a captain smart enough to toss overboard the extra weight will ride high above even surging waves.
  That’s like people in burdened down bodies. A person gets absorbed with filling up, getting weighed down until finally falling into ill health. But those who are well-equipped, light, and truly nourished, avoid the prospect of serious disease. They are like the boat in stormy weather that goes right over a dangerous rock.
  \attrib{\textit{On Fasting} 4, St. Basil}
\end{quote}

\begin{quote}
  Wine wasn’t in paradise; there was not yet any slaughtering of animals, not yet any eating of meat. After the flood there was wine. After the flood, ``you will eat all kinds of things, like you eat vegetables that grow from the ground." When perfection was despaired, then the enjoyment of those things was allowed.
  Now the wine is an example of inexperience, as Noah was ignorant of the use of wine. For it had not yet come into use in life, neither been known in human custom. Since he had neither seen another do it, nor tried it himself, he was unguardedly hurt by it. ``For Noah planted a vineyard, and he drank from the fruit, and he got drunk." He wasn’t out-of-control drunk, he just wasn’t aware of the potent thing he was consuming.
  \attrib{\textit{On Fasting} 5, St. Basil}
\end{quote}

\begin{quote}
  What did Esau throw away, and so was made a slave of his brother? Didn’t he sell his rights as first-born for a single meal? By contrast, wasn’t it with fasting and prayer that Hannah was favored to become the mother of Samuel?
  \attrib{\textit{On Fasting} 6, St. Basil}
\end{quote}