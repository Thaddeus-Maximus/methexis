\begin{quote}
  The development of science and technology, this splendid testimony of the human capacity for understanding and for perseverance, does not free humanity from the obligation to ask the ultimate religious questions. Rather, it spurs us on to face the most painful and decisive of struggles, those of the heart and of the moral conscience.
  \attrib{\textit{Veritatis Splendor} 1, John Paul II}
\end{quote}

\begin{quote}
  "You shall love your neighbor as yourself." In this commandment we find a precise expression of \textit{the singular dignity of the human person,} "the only creature that God has wanted for its own sake."
  \attrib{\textit{Veritatis Splendor} 13, John Paul II}
\end{quote}

\begin{quote}
  Those who live "by the flesh" experience God's law as a burden, and indeed as a denial or at least a restriction of their own freedom. On the other hand, those who are impelled by love and "walk by the Spirit", and who desire to serve others, find in God's Law the fundamental and necessary way in which to practice love as something freely chosen and freely lived out. Indeed, they feel and interior urge- a genuine "necessity" and no longer a form of coercion - not to stop at the minimum demands of the Law, but to live them in their "fullness."... to live our moral life in a way worth of our sublime vocation as "sons in the Son."
  \attrib{\textit{Veritatis Splendor} 18, John Paul II}
\end{quote}