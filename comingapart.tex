\begin{quote}
  Harvard economist Robert Reich was the first to put a name to an evolving new class of workers in his 1991 book, \textit{The Work of Nations}, calling them "symbolic analysts". Reich surveyed the changing job market and divided jobs into three categories: routine production services, in-person services, and symbol-analytic services. In Reiche's formulation, the new class of symbolic analysts consisted of managers, engineers, attorneys, scientists, professors, executives, journalists, consultants, and other "mind workers" whose work consists of processing information. He observed that the new economy was ideall suited to their talents and rewarded them accordingly.
  \attrib{\textit{Coming Apart} 16, Charles Murray 2012}
\end{quote}

\begin{quote}
  In an age when the majority of parents in the top five centiles of cognitive ability worked as farmers, shopkeepers, blue-collar workers, and housewives - a sitution that necessarially prevailed a century ago, given the occupational and educational distributions during the early 1900s - these relationships between the cognitive ability of parents and children had no ominous implications. Today... they do.
  \attrib{\textit{Coming Apart} 68, Charles Murray 2012}
\end{quote}


quoting Massey research paper:
\begin{quote}
  During the late twentieth century, in other words, the well educated and the affulent increasingly segmented themselves off from the rest of American society.
  \attrib{\textit{Coming Apart} 16, Charles Murray 2012}
\end{quote}

\begin{quote}
  Harvard economist Robert Reich was the first to put a name to an evolving new class of workers in his 1991 book, \textit{The Work of Nations}, calling them "symbolic analysts". Reich surveyed the changing job market and divided jobs into three categories: routine production services, in-person services, and symbol-analytic services. In Reiche's formulation, the new class of symbolic analysts consisted of managers, engineers, attorneys, scientists, professors, executives, journalists, consultants, and other "mind workers" whose work consists of processing information. He observed that the new economy was ideall suited to their talents and rewarded them accordingly.
  \attrib{\textit{Coming Apart} 16, Charles Murray 2012}
\end{quote}

\begin{quote}
  Longitudinal evidence reveals that people don't get happier as they go from a modest income to affluence.
  \attrib{\textit{Coming Apart} 265, Charles Murray 2012}
\end{quote}