\documentclass[letterpaper]{article}

\usepackage{parskip} 
\setlength{\parskip}{10pt}
\usepackage{attrib}

%\usepackage{fontspec}
%\setmainfont{CMU Serif}
%\newfontfamily{\greekfont}{CMU Serif}

\usepackage{polyglossia}
\setmainlanguage{english}
\setotherlanguage{greek}

\usepackage{hyperref}

%\usepackage{slantsc,lmodern}

\usepackage[paper=letterpaper,margin=1in]{geometry}

\usepackage{setspace}
\singlespacing


\begin{document}

\clearpage
%% temporary titles
% command to provide stretchy vertical space in proportion
\newcommand\nbvspace[1][3]{\vspace*{\stretch{#1}}}
% allow some slack to avoid under/overfull boxes
\newcommand\nbstretchyspace{\spaceskip0.5em plus 0.25em minus 0.25em}
% To improve spacing on titlepages
\newcommand{\nbtitlestretch}{\spaceskip0.6em}
\pagestyle{plain}
\begin{center}
  \bfseries
  \nbvspace[1]
  \Huge
  {\nbtitlestretch\huge
    THEOLOGY OF TECHNOLOGY}
    %THEOLOGIAE TECHNOLOGIA
    %ΘΕΟΛΟΓΙΑ ΤΗΣ ΤΕΧΝΟΛΟΓΙΑΣ

  \nbvspace[1]
  \normalsize
  REDISCOVERING AND APPLYING THE ANCIENT AND CATHOLIC APPROACH TO TECHNOLOGY\\

  \nbvspace[1]
  \small BY\\
  \Large THADDEUS HUGHES\\

  \nbvspace[2]

  \nbvspace[3]
  \normalsize

  \large
  FLOATING IN THE WILD \\
  \small \MakeUppercase{\today} \\
\end{center}

\newgeometry{top=0.75in,bottom=0.75in,right=0.75in,left=1.25in}
\raggedbottom
\tableofcontents

\newpage

% "Diagnose, then cure."
% O heavenly king, physician of our souls and bodies, come and fill me, and be with my hands as they write to glorify your most holy name and make you known to this world.

\section{Introduction}

There is no more defining aspect of modern life than prevalence technology, yet little is written about how it ought to be developed, considering God and the order he created. Technological progress moves at an unprecedented rate, displacing traditional ways of life unknowningly and unquestioningly. It encroaches on all fronts, often rending our eyes from holy things and towards the secular. We all know the Church once played a key role in the development of new technology both as it motivated new forms of architecture and mediated scientific discussion.

\begin{quote}
  Science replaced religion as preeminent intellectual authority as definer, judge, and guardian of the cultural world view. Human reason and empirical observation replaced theological doctrine and scriptural revelation as the principal means for comprehending the universe.
  \attrib{\textit{Passion} 270-271, Tarnas via \textit{Theosis} 32}
\end{quote}

This is not because religion is at odds with scientific understandings or cannot bear the changes that technologies demand. Rather, it is because of our impatience and our desire to worship progress rather than take the time to plan out how technology should be utilized in our lives.

\begin{quote}
  The development of science and technology, this splendid testimony of the human capacity for understanding and for perseverance, does not free humanity from the obligation to ask the ultimate religious questions. Rather, it spurs us on to face the most painful and decisive of struggles, those of the heart and of the moral conscience.
  \attrib{\textit{Veritatis Splendor} 1, John Paul II}
\end{quote}

In all of scripture, there is nothing that decries technology. \textbf{Technology is capable of having its right place: an aid to man in his journey towards virtue.} That things of a worldly nature could possibly help us is well-known, for our faith is an incarnate one. Most notably, the Logos was made flesh and dwelt among us with a fully human nature. But this is not a one time event: rosaries, prayer ropes, icons, and all manner of liturgical things open windows into the divine, inviting and entreating us to grow closer to God and in virtue. But just as you can lead to a horse to water but not make him drink, \textbf{technology cannot supplant the effort required of man to become virtuous.} We do not place our hope for salvation in material posessions, but in our own spirits. But with our renewed spirits, the works we perform should be changed and given new life. As it is written in the Psalms,

\begin{quote}
  Had you desired sacrifice, I would have offered it,
  but you will not be satisfied with whole-burnt offerings.

  Sacrifice to God is a contrite spirit:
  a crushed and humbled heart God will not spurn.
  In your kindness O Lord, be bountiful to Sion;
  may the walls of Jerusalem be restored.

  Then will You delight in just oblation,
  in sacrifice and whole-burnt offerings.
  Then shall they ofer calves upon your altar.
  \attrib{\textit{Psalm 51:17-21}}
\end{quote}

Why should we be concerned with our technology? After all, it's just about how we use it, right? Not so. Every tool has certain things it is better at doing. A hammer is good for hitting objects, a wrench is good for turning nuts and bolts, a saw for cutting material. And within these are many varieties of each tool: different sized wrenches, wood saws versus metal saws, band saws, rigging axes versus ball pein hammers... the variety goes on.

The furnisher of a workshop dictates what a workshop can do. Those who create and furnish tools not only are opening possibilities for what can be made next, they are inviting new forms directly in. They are also ushering old forms out as there is only so much room in the world for them to be stored - or at least, to be used. And if the lives and work of craftsmen is shaped by the toolmakers that came before them, how much more are the lives of those who recieve from these craftsmen! C.S. Lewis came to the same conclusion more elegantly:

\begin{quote}
  What we call Man's power over Nature turns out to be a power exercised by some men over other men with Nature as its instrument.
  \attrib{\textit{The Abolition of Man}, C.S. Lewis, 1943}
\end{quote}

Every time we choose to develop one tool rather than another, we alter the lives of those who come after us. To write off that how to use the tools given to future generations is their problem, rather than our own, is no argument against sin. In the act of creating the means for the tool, we enable the sins of future generations. Understanding this, how boldly and humbly must an engineer approach their work!

Without bombs, wars could not be fought on the level of casualty they are. Without medicines, more would die to diseases. It is not as simple as to say `make medicines, not weapons', of course. Wars are fought for good reason, and more precise weaponry that minimizes casualties, especially of innocents, is laudable. And medicines often have unintended consequences. In the past century we have seen man given powers he could never have dreamed of, and now, he is charged with their proper use.

\iffalse
For myself, I know the talents of technological prowess God has gifted me, and I would be remiss if I cannot use them to glorify Him, rather than selling them out for the evil one. If I cannot learn to do this, I should cease practicing them altogether. The deep longing to do this has come in many forms and from many angles. What pushed me over the cliff was in reading the works of St. Maximus the Confessor.

\begin{quote}
  The scriptural Word knows of two kinds of knowledge of divine things. On the one hand, there is relative knowledge, rooted only in reason and ideas, and lacking in the kind of experiential perception of what one knows through active engagement; such relative knowledge is what we use to order our affiars in our present life. On the other hand, \textbf{there is that truly authentic knowledge, gained only be actual experience, apart from reason and ideas, which provides a total perception of the known object through a participation (methexis) by grace.} By this latter knowledge, we attain, in the future state, the supernatural deification (theosis) that remains unceasingly in effect. They say that the relative knowledge based on reason and ideas can motivate our desire for the participative knowledge acquired by active engagement. They say, moreover, that this active, experiential knowledge which, by participation, furnishes the direct perception of the object known, can supplant the relative knowledge based on reason and ideas.
  \attrib{\textit{Ad Thalassium 60}, St Maximus}
\end{quote}

Basking in the wisdom of our forefather, I gained deep humility and appreciation. The fathers knew how to tackle the problems we face. We do not need innovative philosophizing, but to heed the wisdom of our most venerable forefathers, and to actively participate in creation rather than to be content with right opinions.
\fi

\section{Diagnosis}

Let us begin by assessing the state of today's broken world. We are torn asunder by our sins, so any analysis of it will be disjointed and broken as well, but we must have some idea of how things are going wrong if they are to be cured.

\subsection{Materialism}

To understand the state of the modern world, one must first understand its central heresy of \textit{materialism}. This heresy insists all that matters (or even, is) is the material realm we can expereince with our five senses. Though it is true that many throughout the ages have persisted in denial of the divine, it is only in modern society that it is the rule and expectation, rather than the exception. Indeed the modern conception of `freedom of religion' prohibits religion from having any meaningful say on the lives of its adherents, since it is not permitted to creep into `secular' things.

Not only does it deny divine revelation, but a materialist worldview leads to the notion that individuals exist as atomic, lacking any sort of traditional roots giving them values on which to stand. This is not based in charity and humility towards our forefathers, but rather in a presumption that what we `know' now is the best. One needs only a cursory glance at history to realise that true wisdom is hard to come by, so to expect it at the cutting edge is folly.

Even a number of enlightenment thinkers, such as the influential John Locke, understood this. One cannot read the works of Locke without still seeing a great deal of ancient wisdom (which he saw as so evident as to call it `natural law') that hadn't yet shaken off. Indeed many regard him to be quite inconsistent because of this, and consistency is key within an enlightened, rational thinker's framework.

\begin{quote}
  The men who conceived the idea that ``morality is bunk" did so with a mind well-stocked with moral ideas. But the minds of the third and fourth generations are no longer well-stocked with such ideas: they are well-stocked with ideas conceived in the nineteenth century, namely, that ``morality is bunk," that everything that appears to be ``higher" is really nothing but something quite mean and vulgar.
  \attrib{\textit{Small is Beautiful} 68, Schumacher 1973}
\end{quote}

%Over the past century, Communism in many countries has been the most blatant manifestation of materialism.
The term \textit{materialism} has an association with \textit{consumerism} in our society. We know from experience at this point that ever-increasing material well-being does not give us ever-increasing happiness. As a renowned social scientist noted,

\begin{quote}
  Longitudinal evidence reveals that people don't get happier as they go from a modest income to affluence.
  \attrib{\textit{Coming Apart} 265, Charles Murray 2012}
\end{quote}

Hedonism does not work - human happiness is not dictated by how many amusements we have. If it were, doubtless, these would be the happiest times, and developed nations would be markedly happier than undeveloped ones.

However, materialism also takes root when we focus on creating and producing things, not just when we collect and consume them. Even insistence upon a basic standard of living for all as the \textit{primary} goal of life misses the mark. The real sickness we see isn't a lack of altruism, it is a lack of spirituality, a lack of real connection to things created and uncreated.

% IDEA: talk about utilitarianism
% IDEA: drug addicts and pleasure vs. self-domination

\subsection{Alienation}

\begin{quote}
  Economic freedom is only one element of human freedom. When it becomes autonomous, \textbf{when man is seen more as a producer or consumer of goods than as a subject who produces and consumes in order to live}, then economic freedom loses its necessary relationship to the human person and ends up by alienating and oppressing him.
  \attrib{\textit{Centessimus Annus} 39, Pope John Paul II 1991, emphasis added}
\end{quote}


\iffalse
Most people understand the idea of alienation at some basic level, the lack of a connection to the fruits of their labor and a sense of pride in their work. Clearly, the placement of a tool between a man and his work invokes a distancing, just as clothes prevent intimacy and an in-person meeting provides more expression than a phone call. Man is \textit{sensual}: he experiences the world through touch, taste, sight, smell, and sound. Among these there is a sort of heirarchy, with sight being the least intimate, and taste being the most.

% IDEA: transition because I just got done trashing materialism and here we are calling man sensual

Tools do not create impassable barriers, but do diminish man's immediate sensory experiences. This can be good to an extent- it is better for him to read a voltage level off a meter than to be electrocuted. Furthermore, not everything that man does is doable without tools. Man cannot work iron without a fire, hammer, and tongs. This being said, the blacksmith feels a much more tactile relationship to his workpiece than the machine operator who must interact with it through dials and instruments. We have a deep longing for these simpler modes of production - hobbyists seeking pure liesure and reconnection do not usually seek high-tech equipment, but the older, tactile modes of production. We know that there is something we have left behind as we move forwards and desire to keep it in view, even if relegated.
\fi

The most popular usage of the term \textit{alienation} comes from the writings of Marx, who utilized the term quite frequently, though failing to grasp what it fully since he lacked a complete vision of what it means to for a man to live (that is, striving for virtue), rather than simply be a producer or consumer of goods. Pope John Paul II continues:

\begin{quote}
  Marxism criticized capitalist bourgeois societies, blaming them for the commercialization and alienation of human existence. This rebuke is of course based on a mistaken and inadequate idea of alienation, derived solely from the sphere of relationships of production and ownership, that is, giving them a materialistic foundation and moreover denying the legitimacy and positive value of market relationships even in their own sphere. Marxism thus ends up by affirming that only in a collective society can alienation be eliminated. However, the historical experience of socialist countries has sadly demonstrated that collectivism does not do away with alienation but rather increases it, adding to it a lack of basic necessities and economic inefficiency.

  The historical experience of the West, for its part, shows that even if the Marxist analysis and its foundation of alienation are false, nevertheless alienation - and the loss of the authentic meaning of life - is a reality in Western societies too. This happens \textbf{in consumerism, when people are ensnared in a web of false and superficial gratifications rather than being helped to experience their personhood in an authentic and concrete way.} \textbf{Alienation is found also in work, when it is organized} so as to ensure maximum returns and profits \textbf{with no concern whether the worker, through his own labour, grows or diminishes as a person}, either through increased sharing in a genuinely supportive community or through increased isolation in a maze of relationships marked by destructive competitiveness and estrangement, in which he is considered only a means and not an end.

  The concept of alienation needs to be led back to the Christian vision of reality, by recognizing in alienation a reversal of means and ends. \textbf{When man does not recognize in himself and in others the value and grandeur of the human person, he effectively deprives himself of the possibility of benefitting from his humanity and of entering into that relationship of solidarity and communion with others for which God created him.} Indeed, it is through the free gift of self that man truly finds himself. This gift is made possible by the human person's essential ``capacity for transcendence". Man cannot give himself to a purely human plan for reality, to an abstract ideal or to a false utopia. As a person, he can give himself to another person or to other persons, and ultimately to God, who is the author of his being and who alone can fully accept his gift. \textbf{A man is alienated if he refuses to transcend himself and to live the experience of self-giving and of the formation of an authentic human community oriented towards his final destiny, which is God.} \textbf{A society is alienated if its forms of social organization, production and consumption make it more difficult to offer this gift of self and to establish this solidarity between people.}
  \attrib{\textit{Centessimus Annus} 41, Pope John Paul II 1991, emphases added}
\end{quote}

% TODO really break apart this quote because there's a lot to it. Every \textbf should be its own minisection.

Let us take each of these examples in turn.

% TODO: work, with no concern whether the worker grows or diminishes as a person through their own work
% TODO: nonrecognition of others, where he does not see the posibility of benefitting from or to others
% TODO: refusal to transcend and give self

\subsection{Surrogate Activity}

\begin{quote}
  ...in consumerism, when people are ensnared in a web of false and superficial gratifications.
  \attrib{\textit{Centessimus Annus} 41, Pope John Paul II 1991}
\end{quote}

The modern world provides surrogate activities to fulfill our desire to offer self to some endeavor or idol. The materialist worldview lends itself to hedonism - if this material realm is all there is, we must not lose time, and extract all the pleasures from it that we can. 

% TODO Silence quotes here

Ray Bradbury's infamous novel \textit{Farenheit 451} portrays the modern media scene, which has come to dominate life, clearly:

\begin{quote}
  ``Picture it. Nineteenth-century man with his horses, dogs, carts, slow motion. Then, in the twentieth century, speed up your camera. Books cut shorter. Condensations. Digests, Tabloids. Everything boils down to the gag, the snap ending." ``Snap ending." Mildred nodded. ``Classics cut to fit fifteen-minute radio shows, then cut again to fill a two-minute book column, winding up at last as a ten- or twelve-line dictionary resume. I exaggerate, of course. The dictionaries were for reference... Speed up the film, Montag, quick. Click, Pic, Look, Eye, Now, Flick, Here, There, Swift, Pace, Up, Down, In, Out, Why, How, Who, What, Where, Eh? Uh! Bang! Smack! Wallop, Bing, Bong, Boom! Digest-digests, digest-digest-digests. Politics? One column, two sentences, a headline! Then, in mid-air, all vanishes! \textbf{Whirl man’s mind around about so fast under the pumping hands of publishers, exploiters, broadcasters that the centrifuge flings off all unnecessary, time-wasting thought!} ... Life is immediate, the job counts, pleasure lies all about after work. Why learn anything save pressing buttons, pulling switches, fitting nuts and bolts?"
  \attrib{\textit{Farenheit 451}, Ray Bradbury 1953, emphasis added}
\end{quote}

It doesn't have to be this way, and surely, hedonistic patterns have existed before ours. Most people \textit{know} that hedonism is bad, and will simply say that they have \textit{guilty pleasures}. 

So, we should go to the gym rather than watch television, right? Perhaps, but the increased interest in gyms over the past few decades isn't just a result of increased interest in bodily health - in many ways it is driven by vanity or simply the fact that less of us work physically difficult jobs and seek surrogate exercise to replace it. Simply put, rather than simply work hard and hoe a field, we utilize tractors to hoe the field, then go to the gym afterwards to get he physical exercise which we need. It could be argued that this is a more healthy pattern as it prevents `back-breaking' labor. That said, using a gym without paying close attention to one's form will lead to the same health issues as laboring with poor form.

Our desire for surrogate activity largely arises from our inability to be still and satisfied with ourselves - we feel uncomfortable sitting still and that we must fill a void.

\begin{quote}
  Sounds and emotions detach us from ourselves, whereas silence always forces man to reflect upon his own life.
  \attrib{\textit{The Power of Silence}, Cardinal Robert Sarah 2016}
\end{quote}

Postmodern seekers miss the mark here still, turning to overdone psychoanalysis, hallucinogens, yoga, and other earthly experiences. None of these address the real problem: we cannot stand to be with ourselves in silence for any amount of time. Like sharks, we must keep swimming.

% TODO: Laurus quote on Vertical vs Horizontal Motion

\iffalse
\begin{quote}
  One should remember that it is in accordance with the taste of one's heart that the future eternal mansion will be given, and that the taste in one's heart there will be the very one that is formed here. It is evident that theaters, shows, and similar things are not suitable for Christians.
  \attrib{\textit{The Path to Salvation} 60, St. Theophan the Recluse 1996}
\end{quote}
\fi

\iffalse
\subsection{Complexity}

Complexity is a more twisted form of alienation. While alienation occurs when one is separated from a subject they could reasonably grasp, complexity prevents one from comprehending the scope of a subject. Consequently, a great many evils can take place in complex systems because they cannot be analyzed holistically. Even when they are understood, they are not readily remedied.

There is the simple form of complexity: equipment that is difficult to understand, discouraging appreciation, repair, and non-alienated interaction. This is the sort of complexity that E.F. Schumacher is speaking of when he writes,

\begin{quote}
  It is my experience that it is rather more difficult to capture directness and simplicity than to advance in the direction of ever more sophistication and complexity. \textbf{Any third-rate engineer or researcher can increase complexity; but it takes a certain flair of real insight to make things simple again. And this insight does not come easily to people who have allowed themselves to become alienated from real, productive work} and from the self-balancing system of nature, which never fails to recognise measure and limitation. Any activity which fails to recognise a self-limiting principle is of the devil.
  \attrib{\textit{Small is Beautiful} 154, Schumacher 1973}
\end{quote}

A mild form of complexity arises in proprietary connections: razors that require special proprietary cartridges, phones requiring specific charging ports, complex financial agreements that `lock you in'. Simpler solutions exist, but the introduction of a special solution which at least claims to offer a minute improvement creates a more complex landscape by its differentiation and incompatibility. It would be one thing for someone to make a better razor cartridge if it retained compatibility with other razors, but this is often, deliberately, not the case.

The worst are \textit{industrial complexes}, where technological businesses become extremely intertwined into the political and social spheres. This has taken many forms, from company towns where a community of people are heavily reliant on some larger system to live (e.g. `coaltowns'), or when defense contractors make political donations or bolster local economies in exchange for political favors for their business (i.e. the `Congressional-Military Industrial Complex'). In the U.S., the medical sector forms another sort of complex. It has multiple regulatory bodies that have intimate interplay with a handful of key corporations, a flow of money that shields consumers from understanding and feeling the true cost of things, and perverse incentives and conditions for providers to secure additional contracts rather than fix problems in full.

% Profiteering

%\iffalse
I do not mean to write about the legality of these practices or political rammifications- much has been written about this already. Complexity can take many forms from mere incompatibility to the predatory practice of 'embrace, extend, extinguish'. This latter practice allows for an outward appearance of reducing complexity with an end result that makes the entire landscape complex and harsh. To make things simple rather than complex is not a mere legal rule to be followed, it is a spirit that must be embraced.

Complexity is not a problem in itself, but it frustrates things. Complex systems do not want to be examined. They do not want to be grasped. They do not want to be overhauled. When one pushes against their walls to diminish their influence they are met with harsh resistance meant to crush spirits. It discourages investigation and curiousity, it encourages one to keep their head down, it promotes machine-like, siloed behaviour.
\fi

\subsection{Usury}

The modern world has a very misguided idea of what economics is, that it somehow has to do inherently with money. The term ``economics" originally means ``household management" in the Greek. From one of the first thinkers to ponder the existence and role of the state, Aristotle,

\begin{quote}
  There are two sorts of wealth-getting as I have said; one is part of household management, the other is retail trade: the former necessary and honourable, which that which consists in exchange is justly censured; for it is unnatural, and a mode by which men gain from one another. The most hated sort, and with the greatest reason, is usury, which makes a gain out of money itself, and not from the natural object of it. For money was to be used in exchange, but not to increase at interest. And this term interest, which means the birth of money from money, is applied to the breeding of money because the offspring resembles the parent. Wherefore of all modes of getting wealth this is the most unnatural.
  \attrib{\textit{Politics} Bk.1 Ch.10, Aristotle}
\end{quote}

This self-begetting interest must always be checked lest it become usury. Like a cancer, what unchecked usury does is crowd out truly productive endeavors, encumbering them and leeching off their resources. Of course, money ultimately must be exchanged for goods and services. That is why the angelic doctor says,

%\begin{quote}
%  To take usury for the lending of money is in itself unjust, because it is a case of selling what is non-existent; and that is manifestly the setting up of an inequality contrary to justice. In evidence of this we must observe that there are certain things, the use of which is the consumption of the thing; as we consume wine by using it to drink, and we consume wheat by using it for food. Hence in such things the use of the thing ought not to be reckoned apart from the thing itself; but whosoever has the use granted to him, has thereby granted to him the thing; and therefore in such things lending means the transference of ownership. If therefore any vendor wanted to make two separate sales, one of the wine and the other of the use of the wine, he would be selling the same thing twice over, or selling the non-existent: hence clearly he would be committing the sin of injustice.
%  \attrib{\textit{Aquinas Ethicus}, Question LXXVIII, St. Thomas Aquinas 1485, Translated/Compiled by Joseph Rickaby 1896}
%\end{quote}

%\begin{quote}
%  In the Carboniferous Epoch we were promised abundance for all,
%  By robbing selected Peter to pay for collective Paul;
%  But, though we had plenty of money, there was nothing our money could buy,
%  And the Gods of the Copybook Headings said: "If you don't work you die."
%  \attrib{\textit{Gods of the Copybook Headings}, Rudyard Kipling, 1919}
%\end{quote}

This focus on monetary behavior rather than productivity has led us to choose some very poor metrics. The measure of ``Gross Domestic Product" (GDP) is often cited by economists and politicians alike as a measure of economic well-being. But let's consider for a moment what this measure is. GDP, for a given country and year, is the market value of all final goods produced in that country and year. This excludes goods that are not brought to market, and services that are performed for oneself!

If one decides to cook vegetables grown in their own garden and eat them rather than go out to a restauraunt, they have harmed the GDP, and by that measure, made the economy worse. If GDP is really good, then don't you dare do your own dishes. Now, this is a ridiculous prospect. Anyone with any sort of sense knows that to be frugal and self-reliant is a good, not an evil. Yet this foolish metric is used as justification for many economic policies, and it prys our eyes away from fruitful works and towards alluring monetary numbers.

How does GDP tie into technology? Quite simply, it has led us to favor technologies that are overly complicated and require markets to function - favoring products brought to market over goods made at home, favoring services rather than self-reliance, and denying the spiritual good that is ascesis.

% https://lukesmith.xyz/articles/why-its-bad-to-have-high-gdp

\iffalse
\subsection{Gnosticism}

Seeing these disconnecting patterns, the other heresy has become apparent in our times as well: the \textit{gnostic} heresy: that material things are flawed, evil, or are in some sense inauthentic or a distraction. These heretics have been throroughly rebuked by such saints as Irenaeus.
\fi

\subsection{Gluttony}

\begin{quote}
  In their greed and solicitude, the gluttons seem absolutely to sweep the world with a drag-net to gratify their luxurious tastes. These gluttons, surrounded with the sound of hissing frying-pans, and wearing their whole life away at the pestle and mortar, cling to matter like fire. 
  \attrib{\textit{The Instructor, Book II}, Clement of Alexandria}
\end{quote}

The developed world is technologically \textit{gluttonous}. We are, of course, familiar with how this would be applied to food. And, of course, even the poorest in developed nations have luxury to be gluttons.

\begin{quote}
  More than that, \textbf{they emasculate plain food, namely bread, by straining off the nourishing part of the grain, so that the necessary part of food becomes matter of reproach to luxury.} There is no limit to epicurism among men. For it has driven them to sweetmeats, and honey-cakes, and sugar-plums; inventing a multitude of desserts, hunting after all manner of dishes. A man like this seems to me to be all jaw, and nothing else. “Desire not,” says the Scripture, “rich men’s dainties;” for they belong to a false and base life. They partake of luxurious dishes, which a little after go to the dunghill. But we who seek the heavenly bread must role the belly, which is beneath heaven, and much more the things which are agreeable to it, which “God shall destroy,” says the apostle, justly execrating gluttonous desires.
  \attrib{\textit{The Instructor, Book II}, Clement of Alexandria}
\end{quote}

This emasculation is rampant in our food today, from mere white bread to think anything marked as `diet'. How telling is it that we have put so much effort into making food that is less nourishing! We have developed technologies and foods that allow us to eat without being fed. We accept this as a part of life without moral panic, but emasculation is yet another thing that leads us to alienation and surrogate activity.

But enough about food- it is not as central to us as it was the ancients. It once was the dominant source of energy - now times have changed (at least in developed nations). We should instead be asking whether our energy utilization is gluttonous.

A typical environmentalist response to the growing demand for electricity is that we should seek the same levels of consumption with less inputs (i.e. increased efficiency) and forms of energy generation which do not pollute as much. Is this really the only way? Are we so bullheaded that we cannot see another way?

There is another way, because the problem is not of a technical nature, but a manifestation of glut. We must ask ourselves: is our consumption (at least in the non-destitute of the developed world) to a point of nourishment, or a point of obesity?

% TODO: find some sources on obesity!

\subsection{Class Divides}

% TODO: this section needs a lot more prose

Modern technology has fundamentally changed the economic system and ushered in new class lines. Prominent social scientists have studied this phenomenon:

\begin{quote}
  Harvard economist Robert Reich was the first to put a name to an evolving new class of workers in his 1991 book, \textit{The Work of Nations}, calling them ``symbolic analysts". Reich surveyed the changing job market and divided jobs into three categories: routine production services, in-person services, and symbol-analytic services. In Reiche's formulation, the new class of symbolic analysts consisted of managers, engineers, attorneys, scientists, professors, executives, journalists, consultants, and other ``mind workers" whose work consists of processing information. He observed that the new economy was ideall suited to their talents and rewarded them accordingly.
  \attrib{\textit{Coming Apart} 16, Charles Murray 2012}
\end{quote}

% IDEA: People tend to compare to their immediate peers, not the globe

We really haven't paused to ask if a confluence of factors around our current modes of production has changed the types of people we normally interact with, and in what manner.

\begin{quote}
  In an age when the majority of parents in the top five centiles of cognitive ability worked as farmers, shopkeepers, blue-collar workers, and housewives - a sitution that necessarially prevailed a century ago, given the occupational and educational distributions during the early 1900s - these relationships between the cognitive ability of parents and children had no ominous implications. Today... they do.
  \attrib{\textit{Coming Apart} 68, Charles Murray 2012}
\end{quote}

If we see more and more people stratified into castes (even if, on paper, completely voluntary and mobile), we are doubtless less capable of being on fraternal terms - let alone good ones. Many companies that utilize sophisticated manufacturing technology do not employ unskilled, low paying labor. The closest they get is to contract out to janitorial services, who come in after hours. The organizational and temporal divide produces and affirms a sort of caste system between these groups.

% IDEA: Why the fight for 15 isn't enough

\section{The Ideal}

% TODO: each section should have a clearly identifiable thesis

If the last section has made you despair, despair no more. The decaying state of affairs will be fertile compost for an old ideal to be planted. In the present age there is a longing to rediscover discarded tradition - and for us westerns and near-easterns, this is a Christian one.

\begin{quote}
  Western Christianity went to sleep in a modern world governed by the gods of reason and observation. It is awakening to a postmodern world open to revelation and hungry for experience. Indeed, one of the last places postmoderns expect to be 'spiritual' is the church.
  \attrib{\textit{Post-Modern Pilgrims} 28, Sweet (via \textit{Theosis} 86, Gama)}
\end{quote}

\subsection{Virtue}

\begin{quote}
  From contemplation of this divine Model [of Jesus Christ], it is more easy to understand that \textbf{the true worth and nobility of man lie in his moral qualities}, that is, \textbf{in virtue}.
  \attrib{\textit{Rerum Novarum} 24, Pope Leo XIII 1891, emphases added}
\end{quote}

We must heed the essential teachings of our blessed Lord that the real value of a society is its moral nature. If you cannot swallow this fundamental truth, no true wisdom will make any sense or be of any use. Once swallowed and digested, things become clear. We recognize that our forefathers had virtue. We desire to learn their understanding, apply it to our own life, and pass it on to our children. The solution to our groanings and pains is not in the modern religions of `innovation' and `scientific thinking'. It is in the past.

\begin{quote}
  All this lyrical stuff about entering the Aquarian Age and reaching a new level of consciousness and taking the next step in evolution is nonsense. Much of it is a sort of delusion of grandeur, the kind of thing you hear from people in the loony bin. What I'm struggling to do is help recapture something our ancestors had. If we can just regain the consciousness the West had before the Cartesian Revolution, which I call the Second Fall of Man, then we'll be getting somewhere.

  \attrib{\href{https://www.religion-online.org/article/small-is-beautiful-and-so-is-rome-surprising-faith-of-e-f-schumacher/}{\textit{Small is Beautiful, and So is Rome}}, Schumacher 1977}
\end{quote}

Or as Leo XIII tells us again to keep our eyes on the source of life and font of immortality,

\begin{quote}
  If human society is to be healed now, in no other way can it be healed save by a return to Christian life and Christian institutions. When a society is perishing, the wholesome advice to give to those who would restore it is to call it to the principles from which it sprang; for the purpose and perfection of an association is to aim at and to attain that for which it is formed, and its efforts should be put in motion and inspired by the end and object which originally gave it being. Hence, to fall away from its primal constitution implies disease; to go back to it, recovery.
  \attrib{\textit{RN} 27, Leo XIII 1891}
\end{quote}

Our technology, then, must cease its obsession with material goods and turn to man in his journey towards virtue: Chastity, Temperance, Charity, Diligence, Patience, Kindness, Humility. This is not necessarially accomplished by merely improving quality of life or reducing sufferring, as trial and tribulation may form virtue. Our Lord accepted and carried his cross, not destroying it, though he did accept help from Simon of Cyrene.

Let us turn then to some prescriptive principles that may be used to instruct our technological endeavors.

\subsection{Wisdom}

\begin{quote}
  It should be placed as an unfailing law that every kind of learning which is taught to a Christian should be penetrated with Christian principles and, more precisely, Orthodox ones. Every branch of learning is capable of this approach, and it will be a true kind of learning only when this condition is fulfilled.
  \attrib{\textit{The Path to Salvation} 64, St. Theophan the Recluse, 1996}
\end{quote}

Or, as the stoics might say, ``the unexamined life is not worth living". The Christian would simply point to the essential teaching uttered over and over that a life not lived through Jesus Christ is no life at all: all others perish. If we build wonderful technologies that reduce material sufferring, but we do it at the cost of even one less soul desiring our Lord, it is for naught. Wisdom, not mere `scientific knowledge', must guide us.

\iffalse
Indeed to harken back to Eden,

\begin{quote}
  Sorrow is knowledge; they who know the most
  Must mourn the deepest o'er the fatal truth,
  The Tree of Knowledge is not that of Life.
  \attrib{\textit{Manfred}, George Gordon Byron 1817}
\end{quote}
\fi

\begin{quote}
  Education cannot help us along as it accords no place to metaphysics. Whether the subjects taught are subjects of science or of the humanities, if the teaching does not lead to a clarification of metaphysics, that is to say, of our fundamental convictions, it cannot educate a man, and consequently, is of no value to society.
  \attrib{\textit{Small is Beautiful} 93, Schumacher 1973}
\end{quote}

Some suggest we need strong `philosophy of science' courses but this is absolutely not so! What we need is for our learning to constantly point back to divine truths. For example, in studying anatomy we gain appreciation for the way the body was laid out: since we are made in God's image, the study of our selves in a pious way will lead us to understand truths about God. One should not get hung up in materialist thinking, that somehow God is composed of atoms and genetic sequences as we are. Rather, the understanding of patterns and analogism is the aim - for some, this is overt, for others, this is subtly done. For we know that the divine cannot be fully grasped and comprehended, only striven towards. This glorification of the divine is an essential purpose of scientific and technological education. If we had good integrated courses like this, there would be scant need for these `philosophy of science' courses: one would simply take the integrated course and be enriched by it. Joel Barstad puts it more poetically in his \href{https://byzantinela.com/cappadocian-house-proposal/}{\textit{Cappadocian House Proposal}},

\begin{quote}
  I added an insistent thirst to overcome certain conventional and traditional oppositions, among them the separation of intelligence from sanctity, of study from worship, of the liberal arts from the arts of subsistence, of the speculative from the practical and creative. How could I hope to know the Word without bowing before Him in worship and them giving him a birth in the materiality of my life?
  \attrib{\textit{Cappadocian House Proposal} 3, Joel Barstad 2017}
\end{quote}

% This will have a secondary benefit which many teachers currently acknowledge, in that knowledge will be linked together thus being easier to remember and recall. 

\subsection{Temperance}

\begin{quote}
  Nothing subdues and controls the body as does the practice of temperance. It is this temperance that serves as a control to those youthful passions and desires.
  \attrib{St. Basil the Great}
\end{quote}

Just as good, nourishing food is hard to indulge in, so too would well-developed technology. I do not mean phones with timers that prevent us from checking email for too long, I mean that utilizing the technology should satiate. It should be self-limiting, never having itself as a goal.

% IDEA: Self-limiting principle
% IDEA: Yield to human virtue where possible

% TODO SILENCE QUOTES

\subsection{Diligence}

\begin{quote}
  Man matures through work that inspires him to do difficult good.
  \attrib{Pope John Paul II}
\end{quote}

The manual, human aspect of work must not be forgotten. There are a great many ills that are curable by labor. We have already discussed the prominence of gyms and how they seek to fill this virtue, but how much better would the virtue of diligence be instilled if it were provided alongside others required for hard manual labor in an integrated fashion? Does not the ends of hard labor provide better motivation and instill in us the value of self-gift, rather than vanity or health?

Diligence is a virtue, unlike belligerence. Diligence has its eyes open and wishes to incite other virtues. Earlier we discussed back-breaking labor. If this is the result of bodily misuse, this is belligerence, as it is abusing the bodily gift we are granted for short-term gain. To slow down and be aware of one's body, to put intent into every stroke not only to produce a superior product but to be kinder to one's body - this is true diligence. Technologies should alert and call us to better bodily form.

\subsection{Patience}

If we do all these things, `progress' as we are accustomed to will no doubt slow. However, we will be able to take heart that we are moving in a direction that is right. Rather than needing to traverse down each pathway to check if it is profitable, we can utilize our sense of virtue and reason to plot our course with confidence. We need only the patience to be satisfied when others ignore this and rush to their demise.

The patience to sit, be bored, and reflect in silence also will help us immensely. Rather than being dependent on technologies to provide us with ever-increasing living standards and amusement, we can be content and less reliant, being able to make wiser, unattached discernment on what technologies will truly benefit us.

We must be patient for test results to come in. Patient when a design did not work out the way we intended.

\subsection{Fraternity}

Friendliness towards God and all his creation is the goal. If anything gets in the way of this, it must be cut off. If the pursuit of increased efficiency causes us to look down upon others as unproductive and thusly less worthy of our attention, how can we justify it? Therefore, technology must always develop in the context of fraternity.

\begin{quote}
  But the Church, with Jesus Christ as her Master and Guide, aims higher still. She lays down precepts yet more perfect, and tries to bind class to class in friendliness and good feeling.
  \attrib{\textit{Rerum Novarum} 21, Leo XIII 1891}
\end{quote}

%TODO: Strike mention of communism.. Be positive

The goal is not mere kindness and compassion towards others, but friendliness. The Communists' atheism stemmed from a strong sense that religion kept masters from sympathizing with their serfs, but overlooked the right relation in the other direction. We know of course to turn the other cheek and paint as much of the picture properly as possible. If someone paints the sky green we should not paint the grass blue in response. Far from being the `opiate of the masses',

\begin{quote}
  There is no intermediary more powerful than religion in drawing the rich an the working class together, by reminding each of its duties to the other, and especially of the obligations of justice.
  \attrib{\textit{Rerum Novarum} 19, Leo XIII 1891}
\end{quote}

After all, what sort of opium leaves one with the sense that they must shoulder responsibility? What opium reminds one of their sins, and tells them they are worthy of hell? Clearly, true religion is no opium: it is that virtue which binds man to man in friendly responsibility and creates that which opium seeks to destroy: fraternity.


% TODO : don't love this quote / section maybe just nuke it
\begin{quote}
  Labor... is not a mere commodity. On the contrary, the worker's human dignity in it must be recognized. It therefore cannot be bought and sold like a commodity. Nevertheless, as the situation now stands, hiring and offering for hire in the so-called labor market separate men into two divisions, as into battle lines, and the contest between these divisions turns the labor market itself almost into a battlefield where, face to face, the opposing lines struggle bitterly. Everyone understands that this grave evil which is plunging all human society to destruction must be remedied as soon as possible. But complete cure will not come until this opposition has been abolished and well-ordered members of the social body - Industries and Professions - are constituted in which men may have their place, not according to the position each has in the labor market but according to the respective social functions which each performs. For under nature's guidance it comes to pass that just as those who are joined together by nearness of habitation establish towns, so those who follow the same industry or profession - whether in the economic or other field - form guilds or associations, so that many are wont to consider these self-governing organizations, if not essential, at least natural to civil society.

  Because order, as St. Thomas well explains, is unity arising from the harmonious arrangement of many objects, a true, genuine social order demands that the various members of a society be united together by some strong bond. This unifying force is present not only in the producing of goods or the rendering of services - in which the employers and employees of an identical Industry or Profession collaborate jointly - but also in that common good, to achieve which all Industries and Professions together ought, each to the best of its ability, to cooperate amicably. And this unity will be the stronger and more effective, the more faithfully individuals and the Industries and Professions themselves strive to do their work and excel in it. 
  \attrib{\textit{Quadragesimo Anno} 83-84, Pope Pius XI, 1931}
\end{quote}

% TODO Luddites

When considering a technology that will displace livelihoods, those whose livelihoods are affected should be consulted to ensure that the new technology meets the full needs (as often certain customers' requirements and needs are not fully understood by the technologists, but are by those whose feet are on the ground). Care should also be made that these livelihoods are not completely abjected and a new suitable line of work is found. It would be unfair, though, to outlaw new and emergent technology on the basis that it interrupts these livelihoods. To continue a job knowing that a machine \textit{could} replace you, implying that the work you provide is of a lesser nature than a mere machine, is a worse fate than to simply be displaced from a line of work and need to find another one. People should be treated with charity that desires their wellbeing, but not with coddling infantilism that denies their agency - this is true dignity.

\subsection{Subsidiarity}

The principle of subsidiarity, that matters ought to be handled by the smallest, lowest, least centralized authority, is one that is lauded by many and unarticulated to most.

\begin{quote}
  It is a fundamental principle of social philosophy, fixed and unchangeable, that one should not withdraw from individuals and commit to the community what they can accomplish by their own enterprise and industry.
  \attrib{\textit{Quadragesimo Anno} 79, Pope Pius XI, 1931}
\end{quote}

When the church calls for an improvement of living conditions, she does not call for it without respect to the means and agents. Leo XIII uses nuanced language when he writes,

\begin{quote}
  Neither must it be supposed that the solicitude of the Church is so preoccupied with the spiritual concerns of her children as to neglect their temporal and earthly interests. Her desire is that the poor, for example, should rise above poverty and wretchedness, and better their condition in life; and for this she makes a strong endeavor.
  \attrib{\textit{Rerum Novarum} 28, Pope Leo XIII 1891}
\end{quote}

It would be one thing to say `the poor should not be wretched'. It is another to say that the ``poor should... better their condition."  It is quite clear that the betterment of their condition would be best derived from their own works, not to be imposed upon them externally (even if this would be their own wish).

% IDEA: It's applied to pol/econ affairs but not tech
% IDEA: It's not only a sensible design policy (root cause) but the consideration of 'where' to solve a problem allows for virtue to enter the picture as a viable answer.

% Standards!
% Thesis


\section{Going Forth}




\iffalse
\subsection{Ascesis}

Return to pre-industrial levels of energy consumption (a sort of \textit{technological asceticism}) may not be necessary, but reconsideration of our lives to see that we are not gluttonous in our usage of energy and technology is. Many have remarked on the need for and benefits of \textit{fasting} from technology. Like a fast from food, such fasting would allow us to bring our passions and desires for technology in line with what is truly necessary for our subsistence, and to make clear how these things place stumbling blocks for the wellness of our souls. 

Fasting is like a palate cleanser, which is why fasting is such powerful remedy. It removes the poor taste for sin we had and allows our inmost appetite for virtue to show forth.

For one unfamiliar with the practice of fasting, \href{http://rutgersnb.occministries.org/wp-content/uploads/2015/07/St.-Basil-the-Great%E2%80%99s-First-Homily-on-Fasting.pdf}{St. Basil the Great's First Homily on Fasting} is an excellent introduction to this powerful practice. 

\begin{quote}
  You would surely agree that the pilot of a merchant ship is better able to safely guide it to port if it is not fully loaded, when it is in excellent condition and light. The ship completely loaded down is sunk by a minor swell in the waters. But the boat that has a captain smart enough to toss overboard the extra weight will ride high above even surging waves.
  That’s like people in burdened down bodies. A person gets absorbed with filling up, getting weighed down until finally falling into ill health. But those who are well-equipped, light, and truly nourished, avoid the prospect of serious disease. They are like the boat in stormy weather that goes right over a dangerous rock.
  \attrib{\textit{On Fasting} 4, St. Basil}
\end{quote}

\iffalse
\begin{quote}
  Wine wasn’t in paradise; there was not yet any slaughtering of animals, not yet any eating of meat. After the flood there was wine. After the flood, ``you will eat all kinds of things, like you eat vegetables that grow from the ground." When perfection was despaired, then the enjoyment of those things was allowed.
  Now the wine is an example of inexperience, as Noah was ignorant of the use of wine. For it had not yet come into use in life, neither been known in human custom. Since he had neither seen another do it, nor tried it himself, he was unguardedly hurt by it. ``For Noah planted a vineyard, and he drank from the fruit, and he got drunk." He wasn’t out-of-control drunk, he just wasn’t aware of the potent thing he was consuming.
  \attrib{\textit{On Fasting} 5, St. Basil}
\end{quote}

\begin{quote}
  What did Esau throw away, and so was made a slave of his brother? Didn’t he sell his rights as first-born for a single meal? By contrast, wasn’t it with fasting and prayer that Hannah was favored to become the mother of Samuel?
  \attrib{\textit{On Fasting} 6, St. Basil}
\end{quote}
\fi

\subsection{Intermediate Technology}
\hfill

\hfill

\hfill

\hfill

\hfill

\hfill

\hfill

\subsection{Right to Repair}
\hfill

\hfill

\hfill

\hfill

\hfill

\hfill

\hfill

\subsection{Recreation}
\hfill

\hfill

\hfill

\hfill

\hfill

\hfill

\hfill

\subsection{Federation}
\hfill

\hfill

\hfill

\hfill

\hfill

\hfill

\hfill
\fi

\end{document}
% uh