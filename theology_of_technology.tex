\documentclass[letterpaper]{article}

\usepackage{attrib}

\usepackage{fontspec}
\setmainfont{CMU Serif}
%\setsansfont{CMU Sans Serif}
\newfontfamily{\greekfont}{CMU Serif}
%\newfontfamily{\greekfontsf}{CMU Sans Serif}

\usepackage{polyglossia}
\setmainlanguage{english}
\setotherlanguage{greek}

\usepackage{hyperref}

%\usepackage{slantsc,lmodern}

\usepackage[paper=letterpaper,margin=1in]{geometry}

\usepackage{setspace}
\singlespacing

\begin{document}

\clearpage
%% temporary titles
% command to provide stretchy vertical space in proportion
\newcommand\nbvspace[1][3]{\vspace*{\stretch{#1}}}
% allow some slack to avoid under/overfull boxes
\newcommand\nbstretchyspace{\spaceskip0.5em plus 0.25em minus 0.25em}
% To improve spacing on titlepages
\newcommand{\nbtitlestretch}{\spaceskip0.6em}
\pagestyle{plain}
\begin{center}
  \bfseries
  \nbvspace[1]
  \Huge
  {\nbtitlestretch\huge
    THEOLOGY OF TECHNOLOGY}
    %THEOLOGIAE TECHNOLOGIA
    %ΘΕΟΛΟΓΙΑ ΤΗΣ ΤΕΧΝΟΛΟΓΙΑΣ

  \nbvspace[1]
  \normalsize
  REDISCOVERING AND APPLYING THE ANCIENT AND CATHOLIC APPROACH TO TECHNOLOGY\\

  \nbvspace[1]
  \small BY\\
  \Large THADDEUS HUGHES\\

  \nbvspace[2]

  \nbvspace[3]
  \normalsize

  \large
  FLOATING IN THE WILD \\
  \small SEPTEMBER 2021 \\
\end{center}

\newgeometry{top=0.75in,bottom=0.75in,right=0.75in,left=1.25in}
\raggedbottom
\tableofcontents

\newpage

\section{Introduction}

\subsection{Motivation}
Much has been written about theology and its application to the world. 

The Church once played a key role in the development of new technologies- both as it motivated new forms of architecture, and as it mediated scientific discussion. In modern times, her more plodding and hesitant approach has gained eyre in favor of rampant, fast development in order to be `first to market'. The speed by which modern institutions develop new material things is unprecedented, surely. But do these things have \textit{virtue}? Nobody wants to talk much about this and is instead content to worship progress, whatever that may be.

The term `technology' is a combination of the greek terms τεκνη (art or skill) and -λογια (logos, a very connotation-laden word encompassing divine reason, speech, study, language). Based on this, it would be best to consider technology as the process by which new physical things are made. That is to say, the development of a wheel, including (and especially) the thought behind it encompasses technology- the wheel is a \textit{fruit} of technology.

So when we discuss ``how to use technology to better mankind", both how to allocate existing material things and the types of new things which should be brought forth are important considerations. 

The deep longing to do this has come in many forms and from many angles. What pushed me over the cliff was in reading the works of St. Maximus the Confessor.

\begin{quote}
  The scriptural Word knows of two kinds of knowledge of divine things. On the one hand, there is relative knowledge, rooted only in reason and ideas, and lacking in the kind of experiential perception of what one knows through active engagement; such relative knowledge is what we use to order our affiars in our present life. On the other hand, \textbf{there is that truly authentic knowledge, gained only be actual experience, apart from reason and ideas, which provides a total perception of the known object through a participation (μεθεξις) by grace.} By this latter knowledge, we attain, in the future state, the supernatural deification (θέωσις) that remains unceasingly in effect. They say that the relative knowledge based on reason and ideas can motivate our desire for the participative knowledge acquired by active engagement. They say, moreover, that this active, experiential knowledge which, by participation, furnishes the direct perception of the object known, can supplant the relative knowledge based on reason and ideas.
  \attrib{\textit{Ad Thalassium 60}, St Maximus}
\end{quote}

Basking in the wisdom of our forefather, I gained deep humility and appreciation. The fathers knew how to tackle the problems we face. We do not need innovative philosophizing, but to heed the wisdom of our most venerable forefathers. I took upon the cross to learn and manifest this \textit{μέθεξις}. With my station as an engineer, and in a world dominated by technology, this seemed quite obviously to be initiated by studying what our relationship towards our technology ought to be.

\subsection{Getting Past the `Enlightenment'}

\begin{quote}
  [Metaphysical reconstruction] is the first task, because without it all these various technological fixes will only add to the confusion. But nowadays, to talk openly about such issues is hardly permitted in polite society.
  \attrib{\href{https://www.religion-online.org/article/small-is-beautiful-and-so-is-rome-surprising-faith-of-e-f-schumacher/}{\textit{SIB+SIR}}, Schumacher 1977}
\end{quote}

The modern (developed) world operates on enlightenment principles. Namely, that individuals exist as atomic: they lack any sort of rooting in their pasts that gives them values on which to stand.

This idea is not rooted in charity and humility, the cornerstones of true wisdom. Rather they are rooted in a presumption that what (we think) we know now is the best- the tip of the spear, the `cutting edge'. One needs only a cursory glance at history and the decline of civilizations to realise this is nowhere near a guaruntee, and that true wisdom is hard to come by.

Even a number of enlightenment thinkers, such as the influential John Locke, understood this. One cannot read the works of Locke without still seeing a great deal of ancient wisdom that hadn't yet shaken off. Indeed many regard him to be quite inconsistent because of this, and consistency is key within an enlightened thinker's framework.

\begin{quote}
  The men who conceived the idea that ``morality is bunk" did so with a mind well-stocked with moral ideas. But the minds of the third and fourth generations are no longer well-stocked with such ideas: they are well-stocked with ideas conceived in the nineteenth century, namely, that ``morality is bunk," that everything that appears to be ``higher" is really nothing but something quite mean and vulgar.
  \attrib{\textit{SIB} 68, Schumacher 1973}
\end{quote}

In the present age there is a longing to rediscover the discarded tradition, which is a Christian one. We must heed the essential teachings of our blessed Lord that the real value of a society is its moral nature. Its development of technology is a manifestation which reinforces this nature. As Leo XIII writes in \textit{Rerum Novarum},

\begin{quote}
  From contemplation of this divine Model [of Jesus Christ], it is more easy to understand that the true worth and nobility of man lie in his moral qualities, that is, in virtue.
  \attrib{\textit{RN} 24, Leo XIII 1891}
\end{quote}

Once this switch in mindset is made, one sees with clear and fresh eyes. Firstly, that our forefathers had virtue. Secondly, we appreciate their understanding of virtue. Finally, we desire to take their understanding and apply it to our own life and pass it on to our children.

The solution to our groanings and pains is not in the modern religions of `innovation' and `scientific thinking'. It is in the past.

\begin{quote}
  All this lyrical stuff about entering the Aquarian Age and reaching a new level of consciousness and taking the next step in evolution is nonsense. Much of it is a sort of delusion of grandeur, the kind of thing you hear from people in the loony bin. What I'm struggling to do is help recapture something our ancestors had. If we can just regain the consciousness the West had before the Cartesian Revolution, which I call the Second Fall of Man, then we'll be getting somewhere.

  \attrib{\href{https://www.religion-online.org/article/small-is-beautiful-and-so-is-rome-surprising-faith-of-e-f-schumacher/}{\textit{SIB+SIR}}, Schumacher 1977}
\end{quote}

Or as Leo XIII tells us again to keep our eyes on the source of life and font of immortality,

\begin{quote}
  If human society is to be healed now, in no other way can it be healed save by a return to Christian life and Christian institutions. When a society is perishing, the wholesome advice to give to those who would restore it is to call it to the principles from which it sprang; for the purpose and perfection of an association is to aim at and to attain that for which it is formed, and its efforts should be put in motion and inspired by the end and object which originally gave it being. Hence, to fall away from its primal constitution implies disease; to go back to it, recovery.
  \attrib{\textit{RN} 27, Leo XIII 1891}
\end{quote}

\section{Metaphysics}

\subsection{The Tree of Knowledge}

There is an endless drumbeat from institutions of learning small and large that we need more technologists (engineers, scientists, the like), with little stopping to ask what they should make.

Lip service is paid that they ought to `better humanity' or that there is `corporate social responsibility' (CSR), but at the end of the day, this seems to boil down to following market forces. Even CSR is especially so since it is frequently used as an advertising campaign- a sin our Lord was quite explicit about.

\begin{quote}
  So when you give to the needy, do not announce it with trumpets, as the hypocrites do in the synagogues and on the streets, to be honored by others. Truly I tell you, they have received their reward in full. But when you give to the needy, do not let your left hand know what your right hand is doing, so that your giving may be in secret. Then your Father, who sees what is done in secret, will reward you.
  \attrib{\textit{Matthew 6:2-4}, NIV}
\end{quote}

While some might say `holistic' education is important, the reality is more severe: \textit{integrated} education is important, along with a recognition that all technologies are built \textit{from} the created firmament, and ought to exist \textit{for} created life.

\begin{quote}
What do I miss, as a human being, if I have never heard of the Second Law of Thermodynamics? The answer is: Nothing. And what do I miss by not knowing Shakespeare? Unless I get my understanding from another source, I simply miss my life.
\attrib{\textit{SIB} 87, Schumacher 1973}
\end{quote}

Or, as the stoics might say, ``the unexamined life is not worth living". The Christian would simply point to the essential teaching uttered over and over that a life not lived through Jesus Christ is no life at all: all others perish. If we build wonderful technologies that reduce material sufferring, but we do it at the cost of even one less soul desiring our Lord, it is for naught.

\iffalse
Indeed to harken back to Eden,

\begin{quote}
  Sorrow is knowledge; they who know the most

  Must mourn the deepest o'er the fatal truth,

  The Tree of Knowledge is not that of Life.
  \attrib{\textit{Manfred}, George Gordon Byron 1817}
\end{quote}
\fi

\begin{quote}
  Education cannot help us along as it accords no place to metaphysics. Whether the subjects taught are subjects of science or of the humanities, if the teaching does not lead to a clarification of metaphysics, that is to say, of our fundamental convictions, it cannot educate a man, and consequently, is of no value to society.
  \attrib{\textit{SIB} 93, Schumacher 1973}
\end{quote}

This might suggest that we need to have endless `philosophy of <<discipline>>' courses- this is absolutely not so! It merely means that the learned skills must constantly be pointing back to underlying truths, and this is the essential purpose, not an ancillary benefit. This will have a secondary benefit in that knowledge will be linked together, thus being easier to remember and recall. As Joel Barstad puts it in his \href{https://byzantinela.com/cappadocian-house-proposal/}{\textit{Cappadocian House Proposal}},

\begin{quote}
  I added an insistent thirst to overcome certain conventional and traditional oppositions, among them the separation of intelligence from sanctity, of study from worship, of the liberal arts from the arts of subsistence, of the speculative from the practical and creative. How could I hope to know the Word without bowing before Him in worship and them giving him a birth in the materiality of my life?
  \attrib{\textit{Cappadocian House Proposal} 3, Joel Barstad 2017}
\end{quote}

\subsection{Soul and Body}

There are two main heresies as regards the relationship between soul and body: one is to accept the body, and to reject the soul. This is \textit{materialism}- all that matters (or even, is) is the material realm we can experience with our five senses. Over the past century or more, this heresy has gained significant traction, and is the basis of many influential ideologies.

Over the past century, Communism in many countries has been the most blatant manifestation of materialism, breeding both atheism and `scientific thinking'. While materialism has a negative connotation in our society, associated with accumulation of goods, this is not the full depth of the evil. Accumulation of mere sensory experiences, another form of hedonism, is materialist. Even insistence upon a basic standard of living for all as the \textit{primary} goal of life misses the mark. Again, the goal is not material well-being, but virtue.

The other heresy is in our time is not as prevalent, although in the early centuries A.D. was quite the rage, is the \textit{gnostic} heresy: that material things are flawed, evil, or are in some sense inauthentic or a distraction. These heretics have been throroughly rebuked by such saints as Irenaeus.

% TODO: Put a quote here or something? idk

The orthodox Christian understanding has always been one of dualism: there exists both a spiritual realm and a physical realm- they interplay, and both are important places not to be neglected. The field of \textit{Theology of the Body} has borne great fruit towards understanding that the soul and body are not doomed to conflict.

% TODO: Expand this ... keep going

\subsection{The Aim of Economics (οίκονέμoμαι) is the Household (οίκος)}

``Economics" comes from the greek word ``οίκονέμoμαι" which means ``household management". And even in this, the household still has an aim: for the benefit of its members both materially and spiritually.

Some have confused the aim of economics to be that of acquiring money. Not so! In fact, in one's reading of Aristotle, we find this original meaning juxtaposed to other means of wealth-getting.

\begin{quote}
  There are two sorts of wealth-getting as I have said; one is part of household management, the other is retail trade: the former necessary and honourable, which that which consists in exchange is justly censured; for it is unnatural, and a mode by which men gain from one another. The most hated sort, and with the greatest reason, is usury, which makes a gain out of money itself, and not from the natural object of it. For money was to be used in exchange, but not to increase at interest. And this term interest [τοκος], which means the birth of money from money, is applied to the breeding of money because the offspring resembles the parent. Wherefore of all modes of getting wealth this is the most unnatural.
  \attrib{\textit{Politics} Bk.1 Ch.10, Aristotle}
\end{quote}

This self-begetting principle must always be checked. As the critic Edward Abbey wittily quoth,

\begin{quote}
  Growth for the sake of growth is the ideology of a cancer cell.
  \attrib{\textit{Edward Abbey}}
\end{quote}

Like a cancer, what unchecked usury does is crowd out truly productive endeavors, encumbering them and leeching off their resources.

\section{Principles}

\subsection{Temperance}

\subsubsection{Technological Gluttony}

The developed world is technologically \textit{gluttonous}. This term has connotations with food, the dominant source of energy for the ancients, but there is no reason we cannot apply it to the ever-growing consumption of disposable goods, electricity, and the like. In fact, since now our predominant source of energy (which powers our predominant source of computational power) is not from food, it makes good sense to apply the virtue of \textit{temperance} to these created and refined resources and goods.

A typical environmentalist response to the growing demand for electricity, primarily is that we should seek the same levels of consumption with less inputs (i.e. increased efficiency) and forms of energy generation which do not pollute as much.

The ascetic would see clearly that this is not a problem with technological roots: it is a manifestation of glut. We must ask ourselves: is our consumption (at least in the non-destitute of the developed world) to a point of nourishment, or a point of obesity?

If an obese person wishes to obtain for themselves the virtue of temperance, how much more would it be gained by eliminating sweetened drinks altogether from their diet rather than merely switching to `sugar-free' sodas? These `diet' options disrupt the body's metabolism and do not help the person shake their sweet tooth. They have not obtained self-rule, they have merely ejected one demon for another. If one does not obtain virtue by using replacement products which are more difficult to produce, how will switching from `dirty' energy sources to `clean' energy sources grant us the virtue of temperance?

\subsubsection{Ascesis}

Return to pre-industrial levels of energy consumption (a sort of \textit{technological asceticism}) may not be necessary, but reconsideration of our lives to see that we are not gluttonous in our usage of energy and technology is. Many have remarked on the need for and benefits of \textit{fasting} from technology. Like a fast from food, such fasting would allow us to bring our passions and desires for technology in line with what is truly necessary for our subsistence, and to make clear how these things place stumbling blocks for the wellness of our souls. 

Fasting is like a palate cleanser, which is why fasting is such powerful remedy. It removes the poor taste for sin we had and allows our inmost appetite for virtue to show forth.

For one unfamiliar with the practice of fasting, \href{http://rutgersnb.occministries.org/wp-content/uploads/2015/07/St.-Basil-the-Great%E2%80%99s-First-Homily-on-Fasting.pdf}{St. Basil the Great's First Homily on Fasting} is an excellent introduction to this powerful practice. 

\begin{quote}
  You would surely agree that the pilot of a merchant ship is better able to safely guide it to port if it is not fully loaded, when it is in excellent condition and light. The ship completely loaded down is sunk by a minor swell in the waters. But the boat that has a captain smart enough to toss overboard the extra weight will ride high above even surging waves.
  That’s like people in burdened down bodies. A person gets absorbed with filling up, getting weighed down until finally falling into ill health. But those who are well-equipped, light, and truly nourished, avoid the prospect of serious disease. They are like the boat in stormy weather that goes right over a dangerous rock.
  \attrib{\textit{On Fasting} 4, St. Basil}
\end{quote}

\iffalse
\begin{quote}
  Wine wasn’t in paradise; there was not yet any slaughtering of animals, not yet any eating of meat. After the flood there was wine. After the flood, “you will eat all kinds of things, like you eat vegetables that grow from the ground.” When perfection was despaired, then the enjoyment of those things was allowed.
  Now the wine is an example of inexperience, as Noah was ignorant of the use of wine. For it had not yet come into use in life, neither been known in human custom. Since he had neither seen another do it, nor tried it himself, he was unguardedly hurt by it. “For Noah planted a vineyard, and he drank from the fruit, and he got drunk.” He wasn’t out-of-control drunk, he just wasn’t aware of the potent thing he was consuming.
  \attrib{\textit{On Fasting} 5, St. Basil}
\end{quote}

\begin{quote}
  What did Esau throw away, and so was made a slave of his brother? Didn’t he sell his rights as first-born for a single meal? By contrast, wasn’t it with fasting and prayer that Hannah was favored to become the mother of Samuel?
  \attrib{\textit{On Fasting} 6, St. Basil}
\end{quote}
\fi

\subsection{Subsidiarity}

\begin{quote}
  Neither must it be supposed that the solicitude of the Church is so preoccupied with the spiritual concerns of her children as to neglect their temporal and earthly interests. Her desire is that the poor, for example, should rise above poverty and wretchedness, and better their condition in life; and for this she makes a strong endeavor.
  \attrib{\textit{RN} 28, Leo XIII 1891}
\end{quote}

Even in this, Leo XIII uses very nuanced language. It would be one thing to say `the poor should not be wretched'. It is another to say that the ``poor should... better their condition."  It is quite clear that the betterment of their condition would be best derived from their own works, not to be imposed upon them externally (even if this would be their own wish).

\subsection{Non-alienation}

Most people understand the idea of alienation at some basic level, that one should feel a connection to the fruits of their labor and a sense of pride in their work. Marx in his writings utilized the term quite frequently, but failed to grasp what it truly meant.

Pope John Paul II in his encyclical \textit{Centesimus Annus} reclaims this principle from the Marxists and illumines it:

\begin{quote}
  All of this can be summed up by repeating once more that economic freedom is only one element of human freedom. When it becomes autonomous, when man is seen more as a producer or consumer of goods than as a subject who produces and consumes in order to live, then economic freedom loses its necessary relationship to the human person and ends up by alienating and oppressing him.
  \attrib{\textit{CA} 39, JPII 1991}
\end{quote}

\begin{quote}
  Marxism criticized capitalist bourgeois societies, blaming them for the commercialization and alienation of human existence. This rebuke is of course based on a mistaken and inadequate idea of alienation, derived solely from the sphere of relationships of production and ownership, that is, giving them a materialistic foundation and moreover denying the legitimacy and positive value of market relationships even in their own sphere. Marxism thus ends up by affirming that only in a collective society can alienation be eliminated. However, the historical experience of socialist countries has sadly demonstrated that collectivism does not do away with alienation but rather increases it, adding to it a lack of basic necessities and economic inefficiency.

  The historical experience of the West, for its part, shows that even if the Marxist analysis and its foundation of alienation are false, nevertheless alienation — and the loss of the authentic meaning of life — is a reality in Western societies too. This happens in consumerism, when people are ensnared in a web of false and superficial gratifications rather than being helped to experience their personhood in an authentic and concrete way. \textbf{Alienation is found also in work, when it is organized} so as to ensure maximum returns and profits \textbf{with no concern whether the worker, through his own labour, grows or diminishes as a person}, either through increased sharing in a genuinely supportive community or through increased isolation in a maze of relationships marked by destructive competitiveness and estrangement, in which he is considered only a means and not an end.

  The concept of alienation needs to be led back to the Christian vision of reality, by recognizing in alienation a reversal of means and ends. \textbf{When man does not recognize in himself and in others the value and grandeur of the human person, he effectively deprives himself of the possibility of benefitting from his humanity and of entering into that relationship of solidarity and communion with others for which God created him.} Indeed, it is through the free gift of self that man truly finds himself. This gift is made possible by the human person's essential ``capacity for transcendence". Man cannot give himself to a purely human plan for reality, to an abstract ideal or to a false utopia. As a person, he can give himself to another person or to other persons, and ultimately to God, who is the author of his being and who alone can fully accept his gift. A man is alienated if he refuses to transcend himself and to live the experience of self-giving and of the formation of an authentic human community oriented towards his final destiny, which is God. A society is alienated if its forms of social organization, production and consumption make it more difficult to offer this gift of self and to establish this solidarity between people.
  \attrib{\textit{CA} 41, JPII 1991, emphases added}
\end{quote}

So what are ways in which \textit{technologies} alienate people- make it more difficult to offer self?

\subsubsection{Physical Separation}

Clearly, the placement of a tool between a man and his work invokes a distancing, just as clothes prevent intimacy. Man is \textit{sensual} and experiences the world through sensory experience.

It is true, though, that the tool does not create an impassable barrier, but perhaps one that should be let down from time to time in order for one to be reminded of the reality of the subject matter- just as lovers may bond in the conjugal act, but then go about their lives with a similar daily love that is not diminished by the lack of this act, but enhanced by the fact that it took place.

% Hi, sorry, I read Humanae Vitae and this is on the mind ok

But, of course, not everything that man does is doable without tools- he cannot work iron without a fire, hammer, and tongs. But the blacksmith feels a much more tactile relationship to his workpiece than the CNC machine operator who must interact with it through dials and instruments.

If one is not persuaded enough by this reminder, no testimony will be enough- this is a truth that must be experienced and lived- and then it is known to be true.

If this physical separation is true, then periodic abstinence from modern tools may allow one to retain connection to the workpiece.

\subsubsection{Prescription and Predestination}

Every tool has an intended use. But even moreso, every tool has certain things it is better at doing. A hammer is good for hitting objects, a wrench is good for turning nuts and bolts, a saw for cutting material. And within these are many varieties of each tool: different sized wrenches, wood saws versus metal saws, band saws, rigging axes versus ball pein hammers... the variety goes on.

The furnisher of a workshop dictates what a workshop can do. In my years at a community makerspace, I have heard the words 'well now that we have this new tool, I could do <this new project>'! Those who create and furnish tools not only are opening possibilities for what can be made next, they are inviting new forms directly in. They are also ushering old forms out as there is only so much room in the world for them to be stored- or at least, to be used.

And if the lives and work of craftsmen is shaped by the toolmakers that came before them, how much more are the lives of those who recieve from these craftsmen! C.S. Lewis came to the same conclusion more elegantly:

\begin{quote}
  What we call Man's power over Nature turns out to be a power exercised by some men over other men with Nature as its instrument.
  \attrib{\textit{The Abolition of Man}, C.S. Lewis, 1943}
\end{quote}

Many people discuss the wonders of 3D printing and how it might open up a future like Star Trek's replicators. But these printers are not capable of producing the same quality of goods- printed plastic is not as strong as wood in many cases, and certainly weaker than metal.

But let's set aside the technical issues, and presume that we will one day have atom-level replicators. If these printers proliferate and displace traditional forms of creating goods, how will this alter our relationship towards creation? Our toil will be reduced and we will have exactly what we want. The materialists would claim this is obviously a win- but us, with our eyes set higher, understand the potential loss of virtue.

When we develop new technologies, we are writing a prescription for the future generations, and must be careful that we do not prescribe pain medicine they will get hooked on, and thusly destroy the rest of their lives.

\subsubsection{Complexity}

These methods culminate in complexity: the addition of layers or technologies to accomplish the same end goals.

\begin{quote}
  It is my experience that it is rather more difficult to capture directness and simplicity than to advance in the direction of ever more sophistication and complexity. \textbf{Any third-rate engineer or researcher can increase complexity; but it takes a certain flair of real insight to make things simple again. And this insight does not come easily to people who have allowed themselves to become alienated from real, productive work} and from the self-balancing system of nature, which never fails to recognise measure and limitation. Any activity which fails to recognise a self-limiting principle is of the devil.
  \attrib{\textit{SIB} 154, Schumacher 1973}
\end{quote}

Complexity can take multiple forms, but it is usually kept hidden by its very nature. Like a large winding river, it requires one to 'step above' in order to see that it is twisted.

Small complexities come in elaborate gadgets- many people remark about the countless kitchen gadgets which offer minute enhancements to niche problems by adding additional components.

Complexity can take the form of supply chains- razors that require special proprietary cartridges, phones requiring specific charging ports. Simpler solutions exist, but the introduction of a special solution which at least claims to offer a minute improvement creates a more complex landscape by its differentiation- and most importantly- incompatibility. It would be one thing for someone to make a better razor cartridge- if it retained compatibility with other razors.

At the worst, though, is \textit{industrial complexes}. The true definition of these is difficult to pin down, but these occur where technological businesses become extremely intertwined into the political and social spheres.

For the US, the 'Military-Congressional Industrial Complex' has been identified as a key example of this, we can see that other systems share similarities. The second most obvious would be the medical field, with multiple regulatory bodies that have intimate interplay with a handful of key corporations, a flow of money that shields consumers from understanding and feeling the true cost of things, and perverse incentives and conditions for providers to secure additional contracts rather than fix problems in full.

I do not mean to write about the legality of these practices or political rammifications- much has been written about this already. Complexity can take many forms from mere incompatibility to the predatory practice of 'embrace, extend, extinguish'. This latter practice allows for an outward appearance of reducing complexity with an end result that makes the entire landscape complex and harsh. To make things simple rather than complex is not a mere legal rule to be followed, it is a spirit that must be embraced.

Complex systems do not want to be examined. They do not want to be grasped. When one pushes against their walls they are met with the same manner of reply: don't rock the boat, that's just the way things are. It discourages investigation and curiousity, it encourages one to keep their head down, it promotes machine-like, siloed behaviour instead of friendliness and good feeling.

\subsection{Fraternity}

Friendliness towards God and all his creation is the goal. If anything gets in the way of this, it must be cut off. If the pursuit of increased efficiency causes us to look down upon others as unproductive and thusly less worthy of our attention, how can we justify it?

\begin{quote}
  But the Church, with Jesus Christ as her Master and Guide, aims higher still. She lays down precepts yet more perfect, and tries to bind class to class in friendliness and good feeling.
  \attrib{\textit{RN} 21, Leo XIII 1891}
\end{quote}

Let us be clear here, though: the goal is not mere kindness and compassion towards others, but friendliness. The Communists' atheism stemmed from a strong sense that religion kept masters from sympathizing with their serfs, but overlooked the right relation in the other direction. We know of course to turn the other cheek and paint as much of the picture properly as possible. If someone paints the sky green we should not paint the grass blue in response. Far from being the 'opaite of the masses',

\begin{quote}
  There is no intermediary more powerful than religion in drawing the rich an the working class together, by reminding each of its duties to the other, and especially of the obligations of justice.
  \attrib{\textit{RN} 19, Leo XIII 1891}
\end{quote}

After all, what sort of opium leaves one with the sense that they must shoulder responsibility? What opium reminds one of their sins, and tells them they are worthy of hell? Clearly, true religion is no opium: it is that virtue which binds man to man in friendly responsibility and creates that which opium seeks to destroy: fraternity.

\iffalse
\section{Projects}

\subsection{META-STEM}

Many might be familiar with the modern acronym of ``STEM" (Science, Technology, Engineering, Mathematics). In some regards, this is a useful collection of loosely related disciplines. In other regards, drawing a line about these and packaging them together suggests that these fields have more to do with each other than they do with arts or philosophy. In practice, this is of course folly- many engineers pull loosely from these other disciplines while working closely with graphic design artists. Are `social sciences' sciences? Aren't they better connected to theology, morality, and ethics?

There's been a movement to ``put the arts in STEM" and turn the acronym to STEAM.

While this is, in some regards, a noble effort and perhaps a slight remedy to the underlying metaphysical mindset (as Dotstoevsky quoth, ``Beauty will save the world"), it is still too shortsighted.

A new acronym, encompassing and embracing these necessary and foiling disciplines, is needed: META-STEM. The META standing for Manual labor, Ethics, Theology, and Arts.

If your stomach churns at the word ``Theology" you could replace it with ``Teleology". These are, essentially, the same thing: the study of the divine is the study of our end goals.

\subsection{Intermediate and Flexible Manufacturing Technologies}

The essential problem that must be wrestled with is this: the short-run monetary costs of utilizing distributed, ethical manufacturing will always be higher than centralized, unethical manufacturing.

Schumacher preaches quite fervently about the \textit{need} for intermediate technologies: technologies that find some middle ground between those used in the developed world and the undeveloped. But what are the principles that define these technologies?

\subsection{Technological Asceticism}
\fi

\end{document}
