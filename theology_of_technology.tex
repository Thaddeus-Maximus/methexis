\documentclass[letterpaper]{article}

\usepackage{parskip} 
\setlength{\parskip}{10pt}
\usepackage{attrib}

%\usepackage{fontspec}
%\setmainfont{CMU Serif}
%\newfontfamily{\greekfont}{CMU Serif}

\usepackage{polyglossia}
\setmainlanguage{english}
%\setotherlanguage{greek}


\usepackage[paper=letterpaper,margin=1in]{geometry}

\usepackage{setspace}


\usepackage{hyperref}
\singlespacing


\begin{document}

\clearpage
% command to provide stretchy vertical space in proportion
\newcommand\nbvspace[1][3]{\vspace*{\stretch{#1}}}

\newcommand{\nbstretchyspace}{\spaceskip0.5em plus 0.25em minus 0.25em}

\newcommand{\nbtitlestretch}{\spaceskip0.6em}
\thispagestyle{empty}
\pagestyle{plain}
\begin{center}
  \bfseries
  \nbvspace[1]
  \Huge
  {\nbtitlestretch\huge
    A THEOLOGY OF TECHNOLOGY}
    %THEOLOGIAE TECHNOLOGIA
    %ΘΕΟΛΟΓΙΑ ΤΗΣ ΤΕΧΝΟΛΟΓΙΑΣ

  \nbvspace[1]
  \normalsize
  THE ANCIENT AND CATHOLIC APPROACH TO TECHNOLOGY\\

  \nbvspace[1]
  \nbvspace[2]

  \nbvspace[3]
  \normalsize

  %\small BY\\
  \Large THADDEUS HUGHES\\


  \large
  IN THE WILD \\
  \small \MakeUppercase{\today} \\
\end{center}

\newgeometry{top=0.75in,bottom=0.75in,right=0.75in,left=1.25in}
\raggedbottom

\clearpage
\setcounter{page}{1}

\tableofcontents

\newpage


% "Diagnose, then cure."
% O heavenly king, physician of our souls and bodies, come and fill me, and be with my hands as they write to glorify your most holy name and make you known to this world.

\section{Introduction}

\begin{quote}
  The development of science and technology, this splendid testimony of the human capacity for understanding and for perseverance, does not free humanity from the obligation to ask the ultimate religious questions. Rather, it spurs us on to face the most painful and decisive of struggles, those of the heart and of the moral conscience.
  % Spurs us on - Concitat - rush, rouse, impel
  \attrib{\textit{Veritatis Splendor} 1, John Paul II}
\end{quote}

There is no more defining aspect of modern life than technology's prevalence in it, yet little is written about how to develop it considering God and his created order. In lieu of a guiding light, technological progress moves at an unprecedented rate, displacing traditional ways of life unknowningly and unquestioningly. It encroaches on all fronts, often rending our eyes from holy things and towards the secular. We all know the Church once played a key role in the development of new technology both as it motivated new forms of architecture and mediated scientific discussion, but no longer serves this role as mediator. This is not because religion is at odds with scientific understandings or cannot bear the changes that technologies demand. Rather, it is because of our impatience and desire to worship progress rather than take the time to plan out how technology should be utilized in our lives.

%\begin{quote}
%  Science replaced religion as preeminent intellectual authority as definer, judge, and guardian of the cultural world view. Human reason and empirical observation replaced theological doctrine and scriptural revelation as the principal means for comprehending the universe.
%  \attrib{\textit{Passion} 270-271, Tarnas via \textit{Theosis} 32}
%\end{quote}

In all of scripture, there is nothing that decries technology outright. \textbf{Technology is capable of having its right place as an aid to man in his journey towards virtue.} That things of a worldly nature could possibly help us is well-known, for our faith is an incarnate one. Most notably, the Logos was made flesh and dwelt among us with a fully human nature. But this is not a one time event: rosaries, prayer ropes, icons, and all manner of liturgical things open windows into the divine, inviting and entreating us to grow closer to God and in virtue. But just as you can lead to a horse to water but not make him drink, \textbf{technology cannot supplant the effort required of man to become virtuous.} We do not place our hope for salvation in material goods, but in our own spirits. With our renewed spirits, the works we perform should be changed and given new life. As it is written in the Psalms,

\begin{quote}
  Had you desired sacrifice, I would have offered it,
  but you will not be satisfied with whole-burnt offerings.

  Sacrifice to God is a contrite spirit:
  a crushed and humbled heart God will not spurn.
  In your kindness O Lord, be bountiful to Sion;
  may the walls of Jerusalem be restored.

  Then will You delight in just oblation,
  in sacrifice and whole-burnt offerings.
  Then shall they ofer calves upon your altar.
  \attrib{\textit{Psalm 51:17-21}}
\end{quote}

Since technology tends to be instrumental to (rather than being composed of) virtue, many people consider technology be an amoral or purely secular endeavor. Clearly, how we utilize it is laden with moral responsibility, but what is developed is also morally charged. Every tool has certain things it is better at doing: a hammer is good for hitting objects, a wrench is good for turning nuts and bolts, a saw for cutting material. And within these are many varieties of each tool: different sized wrenches, wood saws versus metal saws, band saws, rigging axes versus ball pein hammers... the variety goes on.

The furnisher of a workshop dictates what a workshop can do. Those who create and furnish tools not only are opening possibilities for what can be made next, they are inviting new forms directly in. They are also ushering old forms out as there is only so much room in the world for them to be stored - or at least, to be used. And if the lives and work of craftsmen is shaped by the toolmakers that came before them, how much more are the lives of those who recieve from these craftsmen! C.S. Lewis came to the same conclusion more elegantly:

\begin{quote}
  What we call Man's power over Nature turns out to be a power exercised by some men over other men with Nature as its instrument.
  \attrib{\textit{The Abolition of Man}, C.S. Lewis, 1943}
\end{quote}

Every time we choose to develop one tool rather than another, we alter the lives of those who come after us. To write off that how to use the tools given to future generations is their problem, rather than our own, is no argument against sin. In the act of creating the means for the tool, we enable the sins of future generations.

Before we discuss how technology affects our virtuousity, let's dabble in how technologies affect the material realm.

\textbf{Technologies increase efficiency of labor.} Plows, hammers, hoes - rudimentary tools enable us to do things we could by nature of our own bodies, but faster - enabling us to produce more with less effort. Over time, we have learned to harness combustion of fuels and other energy sources, enabling us to produce more with more effort - just not our own. Some technologies have such a multiplier effect that the man-provided labor is merely a signal, a button push.

\textbf{Technologies allow storage and rapid release.} This isn't talked about much, and I think is an often overlooked aspect of technology. Think of batteries, or dams, or stores of grain - we can build up energy when things go well, and insure that when things go wrong, we will survive. On the flipside, we have bombs - investment in refinment of goods to make explosive materials which can be deployed to induce carnage. We can make plans - for better or worse.

\textbf{Technologies enable new experiences.} Flight, space travel, instant global communication - these are hallmarks unique to the last century. But this isn't all positive: nuclear fallout, cyberbullying, crashes... these things are also unique to our time.

\textbf{Technologies alter pain and sufferring.} Modern sanitation and medicine has prevented and de-escalated innumerable cases of illness. Many with previously deathly diseases can be at least treated and stay alive. We also have painkillers of all sorts to numb pains from diseases we cannot cure or treat. Modern manufacturing and materials constantly pose new risks to our health, though, too - from radiation to carcinogens. It would be foolish to think that Asbestos will be the last material widely deployed that we find to have an intolerable health profile.

% IDEA: Talk about the three stages (purgative/illuminative/unitive http://thechristianmysteries.blogspot.com/2011/12/three-stages-of-christian-life.html)

\section{Diagnosis}

To understand the state of modern technology, one must first understand its central heresy of \textit{materialism}. This heresy insists all that matters (or even, is) is the material realm we can expereince with our five senses. Though it is true that many throughout the ages have persisted in denial of the divine, it is only in modern society that it is the rule and expectation, rather than the exception. Indeed the modern conception of `freedom of religion' prohibits religion from having any meaningful say on the lives of its adherents, since it is not permitted to creep into `secular' things.

Not only does it deny divine revelation, but a materialist worldview leads to the notion that individuals exist as atomic, lacking any sort of traditional roots giving them values on which to stand. This is not based in charity and humility towards our forefathers, but rather in a presumption that what we `know' now is the best. One needs only a cursory glance at history to realise that true wisdom is hard to come by, so to expect it at the cutting edge is folly.

Even a number of enlightenment thinkers, such as the influential John Locke, understood this. One cannot read the works of Locke without still seeing a great deal of ancient wisdom (which he saw as so evident as to call it `natural law') that hadn't yet shaken off. Indeed many regard him to be quite inconsistent because of this, and consistency is key within an enlightened, rational thinker's framework.

\begin{quote}
  The men who conceived the idea that ``morality is bunk" did so with a mind well-stocked with moral ideas. But the minds of the third and fourth generations are no longer well-stocked with such ideas: they are well-stocked with ideas conceived in the nineteenth century, namely, that ``morality is bunk," that everything that appears to be ``higher" is really nothing but something quite mean and vulgar.
  \attrib{\textit{Small is Beautiful} 68, Schumacher 1973}
\end{quote}

The term \textit{materialism} often means \textit{consumerism} in our society. We know from experience at this point that ever-increasing material well-being does not give us ever-increasing happiness. Social scientists note,

\begin{quote}
  Longitudinal evidence reveals that people don't get happier as they go from a modest income to affluence.
  \attrib{\textit{Coming Apart} 265, Charles Murray 2012}
\end{quote}

Hedonism does not work - human happiness is not dictated by how many amusements we have. If it were, doubtless, these would be the happiest times, and developed nations would be markedly happier than undeveloped ones.

However, materialism also takes root when we focus on creating and producing things, not just when we collect and consume them. Even insistence upon a basic standard of living for all as the \textit{primary} goal of life misses the mark. The real sickness we see isn't a lack of altruism, it is a lack of spirituality, a lack of real connection to things created and uncreated. This central critique that Marx made of capitalist societies has gripped those across the political spectrum (his prescriptive ideas, not so much) because we have felt this immense wedge being driven between us: Man has become alienated. Pope John Paul II, remarking on the Marxist conception of alienation, reclaims it, showing how it is present in modern society:

\begin{quote}
  Economic freedom is only one element of human freedom. When it becomes autonomous, when man is seen more as a producer or consumer of goods than as a subject who produces and consumes in order to live, then economic freedom loses its necessary relationship to the human person and ends up by alienating and oppressing him.

  [...]

  %Marxism criticized capitalist bourgeois societies, blaming them for the commercialization and alienation of human existence. This rebuke is of course based on a mistaken and inadequate idea of alienation, derived solely from the sphere of relationships of production and ownership, that is, giving them a materialistic foundation and moreover denying the legitimacy and positive value of market relationships even in their own sphere. Marxism thus ends up by affirming that only in a collective society can alienation be eliminated. However, the historical experience of socialist countries has sadly demonstrated that collectivism does not do away with alienation but rather increases it, adding to it a lack of basic necessities and economic inefficiency.

  %The historical experience of the West, for its part, shows that even if the Marxist analysis and its foundation of alienation are false, nevertheless alienation - and the loss of the authentic meaning of life - is a reality in Western societies too.
  This happens in consumerism, when people are ensnared in a web of false and superficial gratifications rather than being helped to experience their personhood in an authentic and concrete way. Alienation is found also in work, when it is organized so as to ensure maximum returns and profits with no concern whether the worker, through his own labour, grows or diminishes as a person, either through increased sharing in a genuinely supportive community or through increased isolation in a maze of relationships marked by destructive competitiveness and estrangement, in which he is considered only a means and not an end.

  The concept of alienation needs to be led back to the Christian vision of reality, by recognizing in alienation a reversal of means and ends. When man does not recognize in himself and in others the value and grandeur of the human person, he effectively deprives himself of the possibility of benefitting from his humanity and of entering into that relationship of solidarity and communion with others for which God created him. Indeed, it is through the free gift of self that man truly finds himself. This gift is made possible by the human person's essential ``capacity for transcendence". Man cannot give himself to a purely human plan for reality, to an abstract ideal or to a false utopia. As a person, he can give himself to another person or to other persons, and ultimately to God, who is the author of his being and who alone can fully accept his gift. A man is alienated if he refuses to transcend himself and to live the experience of self-giving and of the formation of an authentic human community oriented towards his final destiny, which is God. A society is alienated if its forms of social organization, production and consumption make it more difficult to offer this gift of self and to establish this solidarity between people.
  \attrib{\textit{Centessimus Annus} 39-41, Pope John Paul II 1991}
\end{quote}

This alienation from man from his right order - from God - is, essentially, sin. So, let's examine each capital sin.

\subsection{Lust}

\begin{quote}
  A society is alienated if its forms of social organization, production and consumption make it more difficult to offer this gift of self and to establish this solidarity between people.
  \attrib{\textit{Centessimus Annus} 41, Pope John Paul II 1991}
\end{quote}

One might be confused, and even exhausted, to see talk of sex in a tract on technology - but this must be explored even if uncomfortable. The past century has seen a drastic loosening of sexual mores, to the point where this vice is in certain circles seen as a virtue. While it is true that certain purely philosophical approaches needed to alter to get to this point, technology has played key roles in numbing pains which were rightfully instated by our creator to encourage us away from sin. As Pope Paul VI writes in his encyclical on the topic,

\begin{quote}
  Not much experience is needed to be fully aware of human weakness and to understand that human beings - and especially the young, who are so exposed to temptation - need incentives to keep the moreal law, and it is an evil thing to make it easy for them to break that law. Another effect that gives cause for alarm is that a man who grows accustomed to the use of contraceptive methods may forget the reverence due to a woman, and, disregarding her physical and emotional equilibrium, reduce her to bein a mere instrument for the satisfaction of his own desires, no longer considering her as his partner whom he should surround with care and affection.
  \attrib{\textit{Humane Vitae} 17, Pope Paul VI, 1968}
\end{quote}

Man is a sensual being, experiencing the world through sight, smell, sound, taste and touch. These senses inform and affect our moral desires. Man also feels pain, and we do not deny this as a useful and important sensation. The pain of being burned by a hot pan encourages us to not hold onto it and ruin our bodies. The pains associated with sexual perversion, then, should not be numbed. Attenuated, perhaps - but they cannot be silenced, lest they allow man to burn with lust.

Technology also has expanded sexual experience. High-speed, digital, online proliferation of pornography allows young children to see more genitals in an evening than our ancestors could have in a lifetime. Noisy imaging and flashy media - these things also shape our passions and pervert what we desire towards the base and obscene.

\begin{quote}
  For some years now there has been a constant onslaught of images, lights, and colors that blind man. His interior dwelling is violated by the unhealthy, provocative images of pornogrpahy, bestial violence, and all sorts of worldly obscenities that assault purity of heart and infiltrate through the door of sight.
  \attrib{\textit{The Power of Silence} 43, Robert Cardinal Sarah 2016}
\end{quote}

This is no small alteration - and much as Paul VI writes, this `reduces human beings to mere instruments for satisfaction of desires'. This breakdown in how we view others as mere instruments rather than as other humans which we also wish to entice into the kingdom of God breaks our capacity to bond properly and establish solidarity.

\subsection{Gluttony}

\begin{quote}
  ...in consumerism, when people are ensnared in a web of false and superficial gratifications.
  \attrib{\textit{Centessimus Annus} 41, Pope John Paul II 1991}
\end{quote}

Gluttony, the desire of things for their own sake, is rampant, and tears our passions from the divine and towards these earthly things.

\begin{quote}
  In their greed and solicitude, the gluttons seem absolutely to sweep the world with a drag-net to gratify their luxurious tastes. These gluttons, surrounded with the sound of hissing frying-pans, and wearing their whole life away at the pestle and mortar, cling to matter like fire. 
  \attrib{\textit{The Instructor, Book II}, Clement of Alexandria}
\end{quote}

We are, of course, familiar with how gluttony would be applied to food. And, of course, even the poorest in developed nations have luxury to be gluttons.

\begin{quote}
  More than that, \textbf{they emasculate plain food, namely bread, by straining off the nourishing part of the grain, so that the necessary part of food becomes matter of reproach to luxury.} There is no limit to epicurism among men. For it has driven them to sweetmeats, and honey-cakes, and sugar-plums; inventing a multitude of desserts, hunting after all manner of dishes. A man like this seems to me to be all jaw, and nothing else. ``Desire not,'' says the Scripture, ``rich men’s dainties;'' for they belong to a false and base life. They partake of luxurious dishes, which a little after go to the dunghill. But we who seek the heavenly bread must role the belly, which is beneath heaven, and much more the things which are agreeable to it, which “God shall destroy,” says the apostle, justly execrating gluttonous desires.
  \attrib{\textit{The Instructor, Book II}, Clement of Alexandria, emphasis added}
\end{quote}

This emasculation is rampant in our food today, from skim milk to anything marked as `diet'. How telling is it that we have put so much effort into making food that is less nourishing! We have developed technologies and foods that allow us to eat without being fed.

\subsection{Greed}

\begin{quote}
  Alienation is found also in work, when it is organized so as to ensure maximum returns and profits with no concern whether the worker, through his own labour, grows or diminishes as a person.
  \attrib{\textit{Centessimus Annus} 41, Pope John Paul II 1991}
\end{quote}

The modern world has a very misguided idea of what economics is, that it somehow has to do inherently with money. The term ``economics" originally means ``household management" in the Greek. From one of the first thinkers to ponder the existence and role of the state, Aristotle,

\begin{quote}
  There are two sorts of wealth-getting as I have said; one is part of household management, the other is retail trade: the former necessary and honourable, which that which consists in exchange is justly censured; for it is unnatural, and a mode by which men gain from one another. The most hated sort, and with the greatest reason, is usury, which makes a gain out of money itself, and not from the natural object of it. For money was to be used in exchange, but not to increase at interest. And this term interest, which means the birth of money from money, is applied to the breeding of money because the offspring resembles the parent. Wherefore of all modes of getting wealth this is the most unnatural.
  \attrib{\textit{Politics} Bk.1 Ch.10, Aristotle}
\end{quote}

This self-begetting interest must always be checked lest it become usury. Like a cancer, what unchecked usury does is crowd out truly productive endeavors, encumbering them and leeching off their resources. Of course, money ultimately must be exchanged for goods and services - it cannot be an aim.

This focus on monetary behavior rather than productivity has led us to choose some very poor metrics. The measure of ``Gross Domestic Product" (GDP) is often cited by economists and politicians alike as a measure of economic well-being. But let's consider for a moment what this measure is. GDP, for a given country and year, is the market value of all final goods produced in that country and year. This excludes goods that are not brought to market, and services that are performed for oneself!

If one decides to cook vegetables grown in their own garden and eat them rather than go out to a restauraunt, they have harmed the GDP, and by that measure, made the economy worse. If GDP is really good, then don't you dare do your own dishes. Now, this is a ridiculous prospect. Anyone with any sort of sense knows that to be frugal and self-reliant is a good, not an evil. Yet this foolish metric is used as justification for many economic policies, and it prys our eyes away from fruitful works and towards alluring monetary numbers.

How does GDP tie into technology? Quite simply, it has led us to favor technologies that are overly complicated and require markets to function - favoring products brought to market over goods made at home, favoring services rather than self-reliance, and denying the spiritual good that is ascesis.

% https://lukesmith.xyz/articles/why-its-bad-to-have-high-gdp

\subsection{Acedia / Sloth}

\begin{quote}.
  ... it is through the free gift of self that man truly finds himself.
  \attrib{\textit{Centessimus Annus} 41, Pope John Paul II 1991}
\end{quote}

Modern technology has been billed as labor-saving, allowing us more time to rest. Throughout all of scripture, setting aside time to not work is billed as a good.

\begin{quote}
  The Sabbath was made for man, not man for the Sabbath.
  \attrib{\textit{Mark 2:27}}
\end{quote}

But this does not mean that we are free to do with the Sabbath whatsoever we wish. There is clearly a general principle which is laid out.

\begin{quote}
  But every Lord's day do ye gather yourselves together, and break bread, and give thanksgiving after having confessed your transgressions, that your sacrifice may be pure.
  \attrib{\textit{Didache} Chapter XIV.11, A.D. 70}
\end{quote}

Our reprieve from work has a purpose - to face each other, grow in community, and give thanks. For so many, this is not the end use of our labor saving. Instead we retract further into ourselves, exhausted from the work day. Escapism, which we see in so many, is true slothfulness.

\begin{quote}
  [Sloth, which Evagrius personifies as a demon] makes it appear that the sun moves slowly or not at all, and that the day seems to be fifty hours long. Then he compels the monk to look constantly toward the windows, to jump out of the cell, to watch the sun to see how far it is from the ninth hour, to look this way and that ... And further he instills in him a dislike for the place and for his state of life itself, for manual labor, and also the idea that love has disappeared from among the brothers and there is no one to console him.
  \attrib{Evagrius of Pontus}
\end{quote}

So while increased interest in gyms may appear as a manifestation of virtue at first, often, it is out of the fact that our jobs do not fulfill our need for exercise, or that we seek an escape. Simply put, rather than simply work hard and hoe a field, we utilize tractors to hoe the field, then go to the gym afterwards to get he physical exercise which we need. It could be argued that this is a more healthy pattern as it prevents `back-breaking' labor, but using a gym without paying close attention to one's form leads to the same health issues as laboring with poor form.

\subsection{Envy}

\begin{quote}
  ... through increased isolation in a maze of relationships marked by destructive competitiveness and estrangement ...
  \attrib{\textit{Centessimus Annus} 41, Pope John Paul II 1991}
\end{quote}

Far from being the bastion of connection that it was billed to, modern media has provided us with distorted lenses into which we see each others' lives: some of us focus on positives, eschewing the negative realities that plague us, leaving a false perception towards others that our lives are better than theirs by a longshot and thus seeding jealousy. Our envious desires arise also from our inability to be still and satisfied with ourselves and our own lot. We feel uncomfortable sitting still and that we must fill a void. We feel constantly that we are missing out on something greater in a noisy world where so much motion is happening - a constant `fear of missing out'.

\begin{quote}
  Sounds and emotions detach us from ourselves, whereas silence always forces man to reflect upon his own life.
  \attrib{\textit{The Power of Silence} 24, Cardinal Robert Sarah 2016}
\end{quote}

A great many in the secular realm have written on the benefits of `detoxing' from media, but something still must replace this. Postmodern seekers miss the mark here, turning to overdone psychoanalysis, hallucinogens, yoga, and other earthly experiences, prowling around the world in search of something. None of these address the real problem: we cannot stand to be with ourselves in silence for any amount of time. Like sharks, we keep swimming to keep breathing - wasteful horizontal motion that does not move us towards our purpose.

Ray Bradbury's infamous novel \textit{Farenheit 451} portrays the modern media scene, which has come to dominate life, clearly:

\begin{quote}
  ``Picture it. Nineteenth-century man with his horses, dogs, carts, slow motion. Then, in the twentieth century, speed up your camera. Books cut shorter. Condensations. Digests, Tabloids. Everything boils down to the gag, the snap ending." ``Snap ending." Mildred nodded. ``Classics cut to fit fifteen-minute radio shows, then cut again to fill a two-minute book column, winding up at last as a ten- or twelve-line dictionary resume. I exaggerate, of course. The dictionaries were for reference... Speed up the film, Montag, quick. Click, Pic, Look, Eye, Now, Flick, Here, There, Swift, Pace, Up, Down, In, Out, Why, How, Who, What, Where, Eh? Uh! Bang! Smack! Wallop, Bing, Bong, Boom! Digest-digests, digest-digest-digests. Politics? One column, two sentences, a headline! Then, in mid-air, all vanishes! \textbf{Whirl man’s mind around about so fast under the pumping hands of publishers, exploiters, broadcasters that the centrifuge flings off all unnecessary, time-wasting thought!} ... Life is immediate, the job counts, pleasure lies all about after work. Why learn anything save pressing buttons, pulling switches, fitting nuts and bolts?"
  \attrib{\textit{Farenheit 451}, Ray Bradbury 1953, emphasis added}
\end{quote}

It doesn't have to be this way, and surely, hedonistic patterns have existed before ours. Most people \textit{know} that hedonism is bad, and will simply say that they have \textit{guilty pleasures}. 

\begin{quote}
  ``I am not saying wandering is useless: there is a point to it. Do not become like your beloved Alexander [the Great] who had a journey but had no goal. And do not be enamored of excessive horizontal motion.''

  ``Then what should I be enamored of?'' asked Arseny.

  ``Vertical motion,'' answered the elder, pointing above.
  \attrib{\textit{Laurus}, Eugene Vodolazkin 2016}
\end{quote}

\subsection{Wrath}

\begin{quote}
  Modern man is capable of all sorts of noise, all sorts of wars, and so many solemn false statements, in an infernal chaos, because he has excluded God from his life, from his battles, and from his gargantuan ambition to transform the world for his selfish benefit alone.
  \attrib{\textit{The Power of Silence} 34, Robert Cardinal Sarah 2016}
\end{quote}

Since we have ushered out all other virtues in our technology, it is no surprise that we would create devices with little concern how they impact the landscape. We are extracting energy from the bowels of the earth at a tremendous rate in the form of fossil fuels - and these will run out. I do not mean to say that the sky is falling, but it is entirely reasonable to say that the extraction costs will become extremely high within the millenia. We owe it to our children to use much less of this mined energy.

Nuclear energy as it currently stands, billed by many as a good alternative to fossil fuels, will run out in about two centuries at current consumption rate. The plants cannot be readily decomissioned and the waste cannot be readily safely reabsorbed into the environment. Far from being a savior, this is only prolonging the inevitable while creating byproducts that are more difficult to deal with.

And here the renewable and electric advocates might chime in. But as it stands, it requires enormous amounts of electricity - and petroleum products - to produce batteries, solar panels, and wind turbines of the scale we desire. Widespread deployment of these technologies in their immature states will not solve our problems, only exacberate them.

\begin{quote}
  Scientific or technological `solutions' which poison the environment or degrade the social structure and man himself are of no benefit, no matter how brilliantly conceived or how great their superficial attraction. Ever bigger machines, entailing ever bigger concentrations of economic power and exerting ever greater violence aains the environment, do not represent progress: they are a denial of wisdom. [...] Peace, as has often been said, is indivisible - how then could peace be built on a foundation of reckless science and violent technology?
  \attrib{\textit{Small is Beautiful} 20, E.F. Schumacher 1973}
\end{quote}

We also know that even to mere users, modern technology is also increasingly frustrating. So often, it causes us to despise and bemoan using it. So many modern automobiles are a perfect example of this frustration - a friend told me that they needed to replace the gasket on their oil pan. Being accustomed to older vehicles, I said no problem. When I jacked up the car, I found that what used to be a simple operation of removing the bolts on and replacing one component turned out to require a full car lift, a second lift to hold the engine, the exhaust to be removed, among so many other things. I need only to remind anyone who has used a printer how frustrating these devices - which should be simple - are. When we encounter these frustratingly complex systems, it invokes anger and wrath within us as we are incapable of seeing a way forward other than complete rejection of the system.

\begin{quote}
  It is my experience that it is rather more difficult to capture directness and simplicity than to advance in the direction of ever more sophistication and complexity. \textbf{Any third-rate engineer or researcher can increase complexity; but it takes a certain flair of real insight to make things simple again. And this insight does not come easily to people who have allowed themselves to become alienated from real, productive work} and from the self-balancing system of nature, which never fails to recognise measure and limitation. Any activity which fails to recognise a self-limiting principle is of the devil.
  \attrib{\textit{Small is Beautiful} 154, Schumacher 1973}
\end{quote}

%There is the simple form of complexity: equipment that is difficult to understand, discouraging appreciation, repair, and non-alienated interaction. This is the sort of complexity that E.F. Schumacher is speaking of when he writes,

%A mild form of complexity arises in proprietary connections: razors that require special proprietary cartridges, phones requiring specific charging ports, complex financial agreements that `lock you in'. Simpler solutions exist, but the introduction of a special solution which at least claims to offer a minute improvement creates a more complex landscape by its differentiation and incompatibility. It would be one thing for someone to make a better razor cartridge if it retained compatibility with other razors, but this is often, deliberately, not the case.

%The worst are \textit{industrial complexes}, where technological businesses become extremely intertwined into the political and social spheres. This has taken many forms, from company towns where a community of people are heavily reliant on some larger system to live (e.g. `coaltowns'), or when defense contractors make political donations or bolster local economies in exchange for political favors for their business (i.e. the `Congressional-Military Industrial Complex'). In the U.S., the medical sector forms another sort of complex. It has multiple regulatory bodies that have intimate interplay with a handful of key corporations, a flow of money that shields consumers from understanding and feeling the true cost of things, and perverse incentives and conditions for providers to secure additional contracts rather than fix problems in full.

% Profiteering

%I do not mean to write about the legality of these practices or political rammifications- much has been written about this already. Complexity can take many forms from mere incompatibility to the predatory practice of 'embrace, extend, extinguish'. This latter practice allows for an outward appearance of reducing complexity with an end result that makes the entire landscape complex and harsh. To make things simple rather than complex is not a mere legal rule to be followed, it is a spirit that must be embraced.

%Complexity is not a problem in itself, but it frustrates things. Complex systems do not want to be examined. They do not want to be grasped. They do not want to be overhauled. When one pushes against their walls to diminish their influence they are met with harsh resistance meant to crush spirits. It discourages investigation and curiousity, it encourages one to keep their head down, it promotes machine-like, siloed behaviour.

\subsection{Pride}
\begin{quote}
  ...[in] nonrecognition of others, where he does not see the posibility of benefitting from or to others.
  \attrib{\textit{Centessimus Annus} 41, Pope John Paul II 1991}
\end{quote}

Modern technology has fundamentally changed the economic system and ushered in new class lines. Productive technology has always possesed a labor-multiplying effect but never to the magnitude that it does now. When this is compounded with industry clusters (e.g. Silicon Valley) and a modern impulse to devote enormous amounts of time and energy towards productive endeavors (living to work rather than working to live), we see those who are capable of understanding, wielding, and generating developed technology socially isolating from those who cannot. Prominent social scientists have studied this phenomenon:

\begin{quote}
  Harvard economist Robert Reich was the first to put a name to an evolving new class of workers in his 1991 book, \textit{The Work of Nations}, calling them ``symbolic analysts". Reich surveyed the changing job market and divided jobs into three categories: routine production services, in-person services, and symbol-analytic services. In Reiche's formulation, the new class of symbolic analysts consisted of managers, engineers, attorneys, scientists, professors, executives, journalists, consultants, and other ``mind workers" whose work consists of processing information. He observed that the new economy was ideall suited to their talents and rewarded them accordingly.
  \attrib{\textit{Coming Apart} 16, Charles Murray 2012}
\end{quote}

%\begin{quote}
%  In an age when the majority of parents in the top five centiles of cognitive ability worked as farmers, shopkeepers, blue-collar workers, and housewives - a sitution that necessarially prevailed a century ago, given the occupational and educational distributions during the early 1900s - these relationships between the cognitive ability of parents and children had no ominous implications. Today... they do.
%  \attrib{\textit{Coming Apart} 68, Charles Murray 2012}
%\end{quote}

Since people also tend to compare themselves to their immediate peers whom they can readily see, rather than those around the globe, values about what consistutes baseline `intelligence' get formed with a sort of bias. If we see more and more people stratified into castes (even if, on paper, completely voluntary and mobile), we are doubtless less capable of being on fraternal terms - let alone good ones. Many companies that utilize sophisticated manufacturing technology do not employ unskilled, low paying labor. The closest they get is to contract out to janitorial services, who come in after hours and are not seen as a real part of the company. The organizational and temporal divide produces and affirms a sort of caste system between these groups.

\section{The Ideal}

If the last section has made you despair, despair no more. The decaying state of affairs will be fertile compost for an old ideal to be planted. In the present age there is a longing to rediscover discarded tradition - and for us westerns and near-easterns, this is a Christian one.

\begin{quote}
  Western Christianity went to sleep in a modern world governed by the gods of reason and observation. It is awakening to a postmodern world open to revelation and hungry for experience. Indeed, one of the last places postmoderns expect to be 'spiritual' is the church.
  \attrib{\textit{Post-Modern Pilgrims} 28, Sweet (via \textit{Theosis} 86, Gama)}
\end{quote}

%\subsection{Virtue}

\begin{quote}
  From contemplation of this divine Model [of Jesus Christ], it is more easy to understand that \textbf{the true worth and nobility of man lie in his moral qualities}, that is, \textbf{in virtue}.
  \attrib{\textit{Rerum Novarum} 24, Pope Leo XIII 1891, emphases added}
\end{quote}

We must heed the essential teachings of our blessed Lord that the real value of a society is its moral nature. If you cannot swallow this fundamental truth, no true wisdom will make any sense or be of any use. Once swallowed and digested, things become clear. We recognize that our forefathers had virtue. We desire to learn their understanding, apply it to our own life, and pass it on to our children. The solution to our groanings and pains is not in the modern religions of `innovation' and `scientific thinking'. It is in the past.

\begin{quote}
  All this lyrical stuff about entering the Aquarian Age and reaching a new level of consciousness and taking the next step in evolution is nonsense. Much of it is a sort of delusion of grandeur, the kind of thing you hear from people in the loony bin. What I'm struggling to do is help recapture something our ancestors had. If we can just regain the consciousness the West had before the Cartesian Revolution, which I call the Second Fall of Man, then we'll be getting somewhere.

  \attrib{\href{https://www.religion-online.org/article/small-is-beautiful-and-so-is-rome-surprising-faith-of-e-f-schumacher/}{\textit{Small is Beautiful, and So is Rome}}, Schumacher 1977}
\end{quote}

Or as Leo XIII tells us again to keep our eyes on the source of life and font of immortality,

\begin{quote}
  If human society is to be healed now, in no other way can it be healed save by a return to Christian life and Christian institutions. When a society is perishing, the wholesome advice to give to those who would restore it is to call it to the principles from which it sprang; for the purpose and perfection of an association is to aim at and to attain that for which it is formed, and its efforts should be put in motion and inspired by the end and object which originally gave it being. Hence, to fall away from its primal constitution implies disease; to go back to it, recovery.
  \attrib{\textit{RN} 27, Leo XIII 1891}
\end{quote}

\textbf{Our technology must cease its obsession with material goods and turn back to man in his journey towards virtue}: Chastity, Temperance, Charity, Diligence, Patience, Kindness, Humility.
% This is not necessarially accomplished by merely improving quality of life or reducing sufferring, as trial and tribulation may form virtue. Our Lord accepted and carried his cross, not destroying it, though he did accept help from Simon of Cyrene.

\begin{quote}
  It should be placed as an unfailing law that every kind of learning which is taught to a Christian should be penetrated with Christian principles and, more precisely, Orthodox ones. Every branch of learning is capable of this approach, and it will be a true kind of learning only when this condition is fulfilled.
  \attrib{\textit{The Path to Salvation} 64, St. Theophan the Recluse, 1996}
\end{quote}

\textbf{Technology should provoke wisdom; it should illuminate and reflect the Logos.} Scientific exploration should not be a mere quest for knowledge, but driven by a thirst to understand the Logos. 

\begin{quote}
  Education cannot help us along as it accords no place to metaphysics. Whether the subjects taught are subjects of science or of the humanities, if the teaching does not lead to a clarification of metaphysics, that is to say, of our fundamental convictions, it cannot educate a man, and consequently, is of no value to society.
  \attrib{\textit{Small is Beautiful} 93, Schumacher 1973}
\end{quote}

Some suggest we need strong `philosophy of science' courses but this is absolutely not so! What we need is for our learning to constantly point back to divine truths. For example, in studying anatomy we gain appreciation for the way the body was laid out: since we are made in God's image, the study of our selves in a pious way will lead us to understand truths about God. One should not get hung up in materialist thinking, that somehow God is composed of atoms and genetic sequences as we are. Rather, the understanding of patterns and analogism is the aim - for some, this is overt, for others, this is subtly done. For we know that the divine cannot be fully grasped and comprehended, only striven towards. This glorification of the divine is an essential purpose of scientific and technological education. If we had good integrated courses like this, there would be scant need for these `philosophy of science' courses: one would simply take the integrated course and be enriched by it. Joel Barstad puts it more poetically in his \href{https://byzantinela.com/cappadocian-house-proposal/}{\textit{Cappadocian House Proposal}},

\begin{quote}
  I added an insistent thirst to overcome certain conventional and traditional oppositions, among them the separation of intelligence from sanctity, of study from worship, of the liberal arts from the arts of subsistence, of the speculative from the practical and creative. How could I hope to know the Word without bowing before Him in worship and them giving him a birth in the materiality of my life?
  \attrib{\textit{Cappadocian House Proposal} 3, Joel Barstad 2017}
\end{quote}

% This will have a secondary benefit which many teachers currently acknowledge, in that knowledge will be linked together thus being easier to remember and recall. 

Innovation must not be a means to produce material things, but participation in the Logos. When we pick up a tool, we should appreciate the purpose it was made for - and make sure it is used to glorify God in some way. This participative wisdom, the kind of which we say `cannot be described', is of a superior kind than that which is merely lectured. If God had desired us to learn about Him by words alone, he would have made life but a phone call. He did not.

\begin{quote}
  For the sages say that it is impossible for rational knowledge (Logos) of God to coexist with the direct experience of God, or for conceptual knowledge of God to coexist with immediate perception of God... This may very well be what the great Apostle is secretly teaching when he says, \textit{As for prophecies, they will pass away; as for tongues, they will cease; as for knowledge, it will disappear} (1 Cor 13:8). Clearly he is referring here to that knowledge which is found in knowledge and ideas.
  \attrib{\textit{Ad Thalassium 60}, St Maximus}
\end{quote}

\subsection{Chastity}

If our technology is to be chaste.... wtf does that mean?

The conjugal act is most ordered when inhibitions towards it and its consequences are let down - when the prospect of life is fully embraced.

\subsection{Temperance}

% TODO better quote
\begin{quote}
  Nothing subdues and controls the body as does the practice of temperance. It is this temperance that serves as a control to those youthful passions and desires.
  \attrib{St. Basil the Great}
\end{quote}

\textbf{Technology should provoke us to partake in it as a means, not as an end.} 

Just as good, nourishing food is hard to indulge in, so too would well-developed technology. I do not mean phones with timers that prevent us from checking email for too long, I mean that utilizing the technology should satiate. It should be self-limiting, never having itself as a goal.

We should also be wary to pursue marginal material gains at great spiritual expense. Periodic technological fasting may also be good to keep us in line and recognize just how great or small these gains and expenses are.

\begin{quote}
  You would surely agree that the pilot of a merchant ship is better able to safely guide it to port if it is not fully loaded, when it is in excellent condition and light. The ship completely loaded down is sunk by a minor swell in the waters. But the boat that has a captain smart enough to toss overboard the extra weight will ride high above even surging waves.
  That’s like people in burdened down bodies. A person gets absorbed with filling up, getting weighed down until finally falling into ill health. But those who are well-equipped, light, and truly nourished, avoid the prospect of serious disease. They are like the boat in stormy weather that goes right over a dangerous rock.
  \attrib{\textit{On Fasting} 4, St. Basil}
\end{quote}

\subsection{Charity}

\begin{quote}
  Jesus sat down opposite the place where the offerings were put and watched the crowd putting their money into the temple treasury. Many rich people threw in large amounts. But a poor widow came and put in two very small copper coins, worth only a few cents.

  Calling his disciples to him, Jesus said, “Truly I tell you, this poor widow has put more into the treasury than all the others. They all gave out of their wealth; but she, out of her poverty, put in everything—all she had to live on.”
  \attrib{\textit{Mark 12:41-44}}
\end{quote}

Charity towards God and all his creation is the goal. If anything gets in the way of this, it must be cut off. If the pursuit of increased efficiency causes us to look down upon others as unproductive and thusly less worthy of our attention, how can we justify it? In charity we look most to the material wellbeing of a person, but must always be mindful of how these things affect their soul.

The corporal works of mercy are to feed the hungry, give drink to the thirsty, shelter the homeless, visit the sick, visit the prisoners, bury the dead, and give alms to the poor.

The materialist spirit looks at this list and counts and sees what would be an acceptable amount- what would be the baseline where humans should live? But our Lord has called us higher - noting that the amount we give is not an absolute of how much is bestowed upon the downtrodden, but a relative of how much of ourselves we offer. \textbf{Technological progress will not supplant the need for charity.} Even if we get to the point of a fully automated society - the new currency will become time and how much we are willing to spend with others.

\subsection{Diligence}

\begin{quote}
  Man matures through work that inspires him to do difficult good.
  \attrib{Pope John Paul II}
\end{quote}

\textbf{Technology should aid human participation in difficult labor.}

The manual, human aspect of work must not be forgotten. There are a great many ills that are curable by labor - there is no other cure for the sin of sloth. We have already discussed the prominence of gyms and how they seek to fill this virtue, but to focus on only one virtue leads us prone to sins. We must be on guard at all times from all sources.

Diligence is a virtue, unlike belligerence. Diligence has its eyes open and wishes to incite other virtues. Earlier we discussed back-breaking labor. If this is the result of bodily misuse, this is belligerence, as it is abusing the bodily gift we are granted for short-term gain. To slow down and be aware of one's body, to put intent into every stroke not only to produce a superior product but to be kinder to one's body - this is true diligence. Technologies should alert and call us to better bodily form.

With this in mind, technologies like blacksmithing forges do not necessarially violate this principle. If the way they are installed inclines them to hunch (as many anvil setups do), this will in the long run prevent them from future, enduring participation - and far from allowing man to develop, will prevent him from continued active participation.

That human labor is a necessity is an essential teaching of the fathers. 
% TODO Fathers quote?
In the past few centuries, Catholic social teaching has enshrined this in the principle of subsidiarity:

\begin{quote}
  It is a fundamental principle of social philosophy, fixed and unchangeable, that one should not withdraw from individuals and commit to the community what they can accomplish by their own enterprise and industry.
  \attrib{\textit{Quadragesimo Anno} 79, Pope Pius XI, 1931}
\end{quote}

While generally applied to economics and politics, it is applicable to any system. Generally speaking, matters ought to be handled by the smallest, lowest, least centralized authority. Let us turn to Pope Leo XIII to show us how charity leads to this principle.

\begin{quote}
  [The Church's] desire is that the poor, for example, should rise above poverty and wretchedness, and better their condition in life; and for this she makes a strong endeavor.
  \attrib{\textit{Rerum Novarum} 28, Pope Leo XIII 1891}
\end{quote}

It would be one thing to say `the poor should not be wretched' - this could be accomplished with transfers of wealth or goods or an `automated society'. It is another to say that the `poor should better their condition.'  It is quite clear that the betterment of their condition would be best derived from their own works, not to be imposed upon them externally (even if this would be their own wish). The old aphorism is, of course, to teach a man to fish rather than simply give him one. An individual is generally more astutely aware of their material needs and better equipped to determine what they need - it would be better and more meaningful for them if they could provide for themselves and contribute to their community around them.

Subsidiarity, the consideration of `where' to solve a problem, allows for virtue to enter the picture as a viable answer - and here properly developed technology can offer great fruits. The problem of what to do with displaced workers from jobs that have been taken over by automated equipment is often cited as an exemplar of where `offsetting' - taxing profits and redistributing them to those affected - could be a good solution. But this is not a charitable approach as it denies the agency of those displaced in making their own way, nor would outlawing these more productive modes be fair to consumers - and workers. To continue a job knowing that a machine \textit{could} replace you, implying that the work you provide is of a lesser nature than a mere machine, is a worse fate than to simply be displaced from a line of work and need to find another one. People should be treated with charity that desires their wellbeing, but not with coddling infantilism that denies their agency - this is true dignity.

There is a different way forward, and it doesn't involve destroying machinery as the Luddites did, and belligerently ignoring the benefits that technologies can produce.

\begin{quote}
  Technology of production by the masses, making use of the best of modern knowledge and experience, is conducive to decentralisation, compatible with the laws of ecology, gentle in its use of scarce resources, and designed to serve the human person instead of making him the servant of machines. I have named it intermediate technology to signify that it is vastly superior to the primitive technology of bygone ages but at the same time much simpler, cheaper, and freer than the super-technology of the rich. One can also call it self-help technology, or democratic or people's technology - a technology to which everybody can gain admittance and which is not reserved to those already rich and powerful.
  \attrib{\textit{Small is Beautiful} 112-113, E.F. Schumacher, 1973}
\end{quote}

Or in other words,

\begin{quote}
  We need methods and equipment which are cheap enough so that they are accessible to virtually everyone: suitable for small-scale application; and compatible with man's need for creativity.
  \attrib{\textit{Small is Beautiful} 20, E.F. Schumacher, 1973}
\end{quote}

\subsection{Patience}

\begin{quote}
  So is patience set over the things of God, that one can obey no precept, fulfil no work well-pleasing to the Lord, if estranged from it.
  \attrib{\textit{Of Patience}, Tertullian, Second Century}
\end{quote}

\textbf{Technology should allow us to wait confidently in patience, and must be developed with this spirit.}

Contrary to what the world would like to say, technology that will help us to sit, be bored, and reflect in silence is needed.

We are impatient because we lack confidence. We do things quickly in order to minimize our time investment lest the project not turn out as planned. Prototypes are made quickly, tests redlined, for fear that they will not work. We order items because we aren't sure they'll work, and we rush order parts after they break and fail. Yet, the reason why many of these things fail is because they are rushed and not thoroughly thought out, their rammifications not examined. Patience to allow diligence to manifest will allow us to grow in confidence, and thusly, further patience.

Another reason, especially for longer projects, stems again from a materialist and hedonistic impulse. We want to see things come to fruition in our lifetime, because we despair that we will not get to witness and partake in something greater in the afterlife. To remedy this, one should hear stories where it took nine generations to build a cathedral. With our eyes set upon the divine, any amount of time is sufferable for completely right results.

What is more, if our efforts are truly motivated by faith, time spent in prayer and reflection will give us renewed mind and spirit, making us more liable to succeed without losing sight on what is important.


\subsection{Kindness}

\begin{quote}
  Labor... is not a mere commodity. On the contrary, the worker's human dignity in it must be recognized. It therefore cannot be bought and sold like a commodity. Nevertheless, as the situation now stands, hiring and offering for hire in the so-called labor market separate men into two divisions, as into battle lines, and the contest between these divisions turns the labor market itself almost into a battlefield where, face to face, the opposing lines struggle bitterly. Everyone understands that this grave evil which is plunging all human society to destruction must be remedied as soon as possible. But complete cure will not come until this opposition has been abolished and well-ordered members of the social body - Industries and Professions - are constituted in which men may have their place, not according to the position each has in the labor market but according to the respective social functions which each performs. For under nature's guidance it comes to pass that just as those who are joined together by nearness of habitation establish towns, so those who follow the same industry or profession - whether in the economic or other field - form guilds or associations, so that many are wont to consider these self-governing organizations, if not essential, at least natural to civil society.
  \attrib{\textit{Quadragesimo Anno} 83, Pope Pius XI, 1931}
\end{quote}

Kindness also takes the form of providing a frustration-free experience to users. This means not only good design principles, but also providing wiring schematics, drawings, bills of materials, and other documentation that would be useful in repairing something rather than relying upon a company service technician.


\begin{quote}
  But the Church, with Jesus Christ as her Master and Guide, aims higher still. She lays down precepts yet more perfect, and tries to bind class to class in friendliness and good feeling.
  \attrib{\textit{Rerum Novarum} 21, Leo XIII 1891}
\end{quote}

\subsection{Humility}

\begin{quote}
  There is no intermediary more powerful than religion in drawing the rich an the working class together, by reminding each of its duties to the other, and especially of the obligations of justice.
  \attrib{\textit{Rerum Novarum} 19, Leo XIII 1891}
\end{quote}

\hfill

\hfill

\hfill

\hfill

\hfill

\hfill

\hfill

\hfill

\hfill

\section{Going Forth}

\begin{quote}
  Lord God, in your goodness have mercy on me: do not look on my sins, but take away all my guilt. Create in me a clean heart and renew within me an upright spirit.
  \attrib{\textit{Psalm 51}}
\end{quote}

We must learn from our mistakes and walk forth justly. We must build, select, and use technologies that aid man in his journey towards virtue. It should help us to grow in wisdom, to participate in hard labor, to wait confidently in patience, and truly desire this for all of those around us.

All the while we must be on guard for heresy. We must not let the things of this world become our idols once again. We cannot serve two masters - our technology must remain subservient to us and our God.

\end{document}