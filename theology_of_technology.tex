\documentclass[letterpaper]{article}

\usepackage{parskip} 
\usepackage{attrib}

\usepackage{fontspec}
\setmainfont{CMU Serif}
%\setsansfont{CMU Sans Serif}
\newfontfamily{\greekfont}{CMU Serif}
%\newfontfamily{\greekfontsf}{CMU Sans Serif}

\usepackage{polyglossia}
\setmainlanguage{english}
\setotherlanguage{greek}

\usepackage{hyperref}

%\usepackage{slantsc,lmodern}

\usepackage[paper=letterpaper,margin=1in]{geometry}

\usepackage{setspace}
\singlespacing

\setlength{\parskip}{10pt}

\begin{document}

\clearpage
%% temporary titles
% command to provide stretchy vertical space in proportion
\newcommand\nbvspace[1][3]{\vspace*{\stretch{#1}}}
% allow some slack to avoid under/overfull boxes
\newcommand\nbstretchyspace{\spaceskip0.5em plus 0.25em minus 0.25em}
% To improve spacing on titlepages
\newcommand{\nbtitlestretch}{\spaceskip0.6em}
\pagestyle{plain}
\begin{center}
  \bfseries
  \nbvspace[1]
  \Huge
  {\nbtitlestretch\huge
    THEOLOGY OF TECHNOLOGY}
    %THEOLOGIAE TECHNOLOGIA
    %ΘΕΟΛΟΓΙΑ ΤΗΣ ΤΕΧΝΟΛΟΓΙΑΣ

  \nbvspace[1]
  \normalsize
  REDISCOVERING AND APPLYING THE ANCIENT AND CATHOLIC APPROACH TO TECHNOLOGY\\

  \nbvspace[1]
  \small BY\\
  \Large THADDEUS HUGHES\\

  \nbvspace[2]

  \nbvspace[3]
  \normalsize

  \large
  FLOATING IN THE WILD \\
  \small SEPTEMBER 2021 \\
\end{center}

\newgeometry{top=0.75in,bottom=0.75in,right=0.75in,left=1.25in}
\raggedbottom
\tableofcontents

\newpage

% "Diagnose, then cure."
% O heavenly king, physician of our souls and bodies, come and fill me, and be with my hands as they write to glorify your most holy name and make you known to this world.

\section{Introduction}

Technological progress in the past century has raised some serious alarms. It moves forward at an unprecedented rate, often displacing traditional ways of life unknowningly, unquestioningly, and inadvertently. Technology seems to encroach on all sides and dominate all spheres, rending our eyes from holy things and towards secular technology. Yet, in all of scripture, there is nothing that decries it outright. It is very obvious then, that technology is something akin to the many things of this world- capable of having its right place, though this may be difficult to comprehend and practice. A theological approach to technology can not only avert the present crisis, but provide the fruits that material progress was billed to allow us.

%%%%%%%%%%%%%%%%%%%%%%%%

The Church once played a key role in the development of new technologies- both as it motivated new forms of architecture, and as it mediated scientific discussion. Now her more plodding and hesitant approach has been long abandoned in favor of rapid development in order to be `first to market'. The speed by which modern institutions develop new material things is unprecedented, surely. But do these things have \textit{virtue}? Nobody wants to talk much about this and is instead content to worship progress, whatever that may be.

Let us first examine what we mean by `technology'. The word is a combination of the Greek terms τεκνη (art or skill) and -λογια (logos, a very connotation-laden word encompassing divine reason, speech, study, language). Based on this, it would be best to consider technology as the process by which new physical things are made. That is to say, the development of a wheel, including (and especially) the thought behind it encompasses technology - the wheel is a \textit{fruit} of technology.

So when we discuss ``how to use technology", we must consider how to allocate existing useful things, the types of new things which should be brought forth, and how we should talk about and orient ourselves around useful things. 

For myself, I know the great talents of technological prowess God has gifted me, and I would be remiss if I cannot use them to glorify His most Holy Spirit, rather than selling them out to the evil one.

The deep longing to do this has come in many forms and from many angles. What pushed me over the cliff was in reading the works of St. Maximus the Confessor.

\begin{quote}
  The scriptural Word knows of two kinds of knowledge of divine things. On the one hand, there is relative knowledge, rooted only in reason and ideas, and lacking in the kind of experiential perception of what one knows through active engagement; such relative knowledge is what we use to order our affiars in our present life. On the other hand, \textbf{there is that truly authentic knowledge, gained only be actual experience, apart from reason and ideas, which provides a total perception of the known object through a participation (μεθεξις) by grace.} By this latter knowledge, we attain, in the future state, the supernatural deification (θέωσις) that remains unceasingly in effect. They say that the relative knowledge based on reason and ideas can motivate our desire for the participative knowledge acquired by active engagement. They say, moreover, that this active, experiential knowledge which, by participation, furnishes the direct perception of the object known, can supplant the relative knowledge based on reason and ideas.
  \attrib{\textit{Ad Thalassium 60}, St Maximus}
\end{quote}

Basking in the wisdom of our forefather, I gained deep humility and appreciation. The fathers knew how to tackle the problems we face. We do not need innovative philosophizing, but to heed the wisdom of our most venerable forefathers.

\iffalse
I took upon the cross to learn and manifest this \textit{μέθεξις}. With my station as an engineer, and in a world dominated by technology, this seemed quite obviously to be initiated by studying what our relationship towards our technology ought to be.
\fi


\section{A Diagnosis}

% This could get hairy and broken, because it is analyzing our broken world. It should be disjointed and not make much sense. Don't worry about that.

\subsection{Materialism}

To understand the state of the modern world, one must first understand its central heresy of \textit{materialism}. This heresy insists all that matters (or even, is) is the material realm we can expereince with our five senses. Though it is true that many throughout the ages have persisted in denial of the divine, it is only in modern society that it is the rule and expectation, rather than the exception to do this. Indeed the modern conception of `freedom of religion' prohibits religion from having any useful say on the lives of its adherents, since it is not to creep into `secular' things.

The modern world operates on these materialist enlightenment principles, chief among them that individuals exist as atomic, lacking any sort of rooting in their pasts that gives them values on which to stand. This idea is not rooted in charity and humility towards our forefathers, but rather in a presumption that what we `know' now is the best- the `cutting edge' is always the best. One needs only a cursory glance at history and the decline of civilizations to realise this is nowhere near a guaruntee, and that true wisdom is hard to come by, so to expect it at the every new offshoot is obviously wrong.

Even a number of enlightenment thinkers, such as the influential John Locke, understood this. One cannot read the works of Locke without still seeing a great deal of ancient wisdom (which he saw as so evident as to name it `natural law') that hadn't yet shaken off. Indeed many regard him to be quite inconsistent because of this, and consistency is key within an enlightened thinker's framework.

\begin{quote}
  The men who conceived the idea that ``morality is bunk" did so with a mind well-stocked with moral ideas. But the minds of the third and fourth generations are no longer well-stocked with such ideas: they are well-stocked with ideas conceived in the nineteenth century, namely, that ``morality is bunk," that everything that appears to be ``higher" is really nothing but something quite mean and vulgar.
  \attrib{\textit{Small is Beautiful} 68, Schumacher 1973}
\end{quote}

Over the past century, Communism in many countries has been the most blatant manifestation of materialism. Materialism has a negative connotation in our society associated with accumulation of goods, but this is not the full depth of the evil. Accumulation of mere sensory experiences, another form of hedonism, is materialist. Even insistence upon a basic standard of living for all as the \textit{primary} goal of life misses the mark. Again, the goal is not material well-being, but virtue.

\subsection{Distractability}

\begin{quote}
  Sounds and emotions detach us from ourselves, whereas silence always forces man to reflect upon his own life.
  \attrib{\textit{The Power of Silence}, Cardinal Robert Sarah}
\end{quote}

The modern condition is abhorrent when true self reflectance is committed.

\hfill

\hfill

\hfill

\hfill

\subsection{Alienation}

Most people understand the idea of alienation at some basic level, that one should feel a connection to the fruits of their labor and a sense of pride in their work. Marx in his writings utilized the term quite frequently, but failed to grasp what it truly meant.

Pope John Paul II in his encyclical \textit{Centesimus Annus} reclaims this principle from the Marxists:

\begin{quote}
  All of this can be summed up by repeating once more that economic freedom is only one element of human freedom. When it becomes autonomous, \textbf{when man is seen more as a producer or consumer of goods than as a subject who produces and consumes in order to live}, then economic freedom loses its necessary relationship to the human person and ends up by alienating and oppressing him.
  \attrib{\textit{Centessimus Annus} 39, Pope John Paul II 1991, emphasis added}
\end{quote}

\begin{quote}
  Marxism criticized capitalist bourgeois societies, blaming them for the commercialization and alienation of human existence. This rebuke is of course based on a mistaken and inadequate idea of alienation, derived solely from the sphere of relationships of production and ownership, that is, giving them a materialistic foundation and moreover denying the legitimacy and positive value of market relationships even in their own sphere. Marxism thus ends up by affirming that only in a collective society can alienation be eliminated. However, the historical experience of socialist countries has sadly demonstrated that collectivism does not do away with alienation but rather increases it, adding to it a lack of basic necessities and economic inefficiency.

  The historical experience of the West, for its part, shows that even if the Marxist analysis and its foundation of alienation are false, nevertheless alienation — and the loss of the authentic meaning of life — is a reality in Western societies too. This happens in consumerism, when people are ensnared in a web of false and superficial gratifications rather than being helped to experience their personhood in an authentic and concrete way. \textbf{Alienation is found also in work, when it is organized} so as to ensure maximum returns and profits \textbf{with no concern whether the worker, through his own labour, grows or diminishes as a person}, either through increased sharing in a genuinely supportive community or through increased isolation in a maze of relationships marked by destructive competitiveness and estrangement, in which he is considered only a means and not an end.

  The concept of alienation needs to be led back to the Christian vision of reality, by recognizing in alienation a reversal of means and ends. \textbf{When man does not recognize in himself and in others the value and grandeur of the human person, he effectively deprives himself of the possibility of benefitting from his humanity and of entering into that relationship of solidarity and communion with others for which God created him.} Indeed, it is through the free gift of self that man truly finds himself. This gift is made possible by the human person's essential ``capacity for transcendence". Man cannot give himself to a purely human plan for reality, to an abstract ideal or to a false utopia. As a person, he can give himself to another person or to other persons, and ultimately to God, who is the author of his being and who alone can fully accept his gift. A man is alienated if he refuses to transcend himself and to live the experience of self-giving and of the formation of an authentic human community oriented towards his final destiny, which is God. A society is alienated if its forms of social organization, production and consumption make it more difficult to offer this gift of self and to establish this solidarity between people.
  \attrib{\textit{Centessimus Annus} 41, Pope John Paul II 1991, emphases added}
\end{quote}

So what are ways in which \textit{technologies} alienate people?

Clearly, the placement of a tool between a man and his work invokes a distancing, just as clothes prevent intimacy and an in-person meeting is more valued than a phone call. Man is \textit{sensual} and experiences the world through sensory experience.

It is true, though, that the tool does not create an impassable barrier, but perhaps one that should be let down from time to time in order for one to be reminded of the reality of the subject matter - just as friends stay friends even when separated by distance and only able to write or call, their relationship is enhanced by the times they have shared together in the flesh.

But, of course, not everything that man does is doable without tools - he cannot work iron without a fire, hammer, and tongs. But the blacksmith feels a much more tactile relationship to his workpiece than the CNC machine operator who must interact with it through dials and instruments. This is a truth that must be experienced and lived - and then it is known to be true. If this physical separation is true, then periodic abstinence from modern tools may allow one to retain connection to the workpiece.

\subsection{Consumption}

Nature abhors a vacuum. 

\hfill

\hfill

\hfill

\hfill

\subsection{Complexity}

Complexity is closely related to, but not exactly, alienation. While alienation occurs when one is separated from a subject they could reasonably grasp, complexity prevents one from comprehending the scope of a subject. Consequently, a great many evils can take place in complex systems because they cannot be analyzed holistically. Even when they are understood, they are not readily remedied.

There is the simple form of complexity: equipment that is difficult to understand, discouraging appreciation, repair, and non-alienated interaction. This is the sort of complexity that E.F. Schumacher is speaking of when he writes,

\begin{quote}
  It is my experience that it is rather more difficult to capture directness and simplicity than to advance in the direction of ever more sophistication and complexity. \textbf{Any third-rate engineer or researcher can increase complexity; but it takes a certain flair of real insight to make things simple again. And this insight does not come easily to people who have allowed themselves to become alienated from real, productive work} and from the self-balancing system of nature, which never fails to recognise measure and limitation. Any activity which fails to recognise a self-limiting principle is of the devil.
  \attrib{\textit{Small is Beautiful} 154, Schumacher 1973}
\end{quote}

Complexity can also take the form of supply chains- razors that require special proprietary cartridges, phones requiring specific charging ports. Simpler solutions exist, but the introduction of a special solution which at least claims to offer a minute improvement creates a more complex landscape by its differentiation and incompatibility. It would be one thing for someone to make a better razor cartridge- if it retained compatibility with other razors.

The worst, though, are \textit{industrial complexes}, occurring when technological businesses become extremely intertwined into the political and social spheres. This can take many forms, from company towns where a community of people are heavily reliant on some larger system to live (e.g. `coaltowns'), or when defense contractors make political donations or bolster local economies in exchange for political favors for their business (i.e. the `Congressional-Military Industrial Complex'). In the U.S., the medical sector forms another sort of complex. It has multiple regulatory bodies that have intimate interplay with a handful of key corporations, a flow of money that shields consumers from understanding and feeling the true cost of things, and perverse incentives and conditions for providers to secure additional contracts rather than fix problems in full.

I do not mean to write about the legality of these practices or political rammifications- much has been written about this already. Complexity can take many forms from mere incompatibility to the predatory practice of 'embrace, extend, extinguish'. This latter practice allows for an outward appearance of reducing complexity with an end result that makes the entire landscape complex and harsh. To make things simple rather than complex is not a mere legal rule to be followed, it is a spirit that must be embraced.

Complexity is not a problem in itself, but it frustrates things. Complex systems do not want to be examined. They do not want to be grasped. They do not want to be overhauled. When one pushes against their walls they are met with the same manner of reply: don't rock the boat, that's just the way things are. It discourages investigation and curiousity, it encourages one to keep their head down, it promotes machine-like, siloed behaviour.

\subsection{Predestination}

Every tool has an intended use. But even moreso, every tool has certain things it is better at doing. A hammer is good for hitting objects, a wrench is good for turning nuts and bolts, a saw for cutting material. And within these are many varieties of each tool: different sized wrenches, wood saws versus metal saws, band saws, rigging axes versus ball pein hammers... the variety goes on.

The furnisher of a workshop dictates what a workshop can do. Those who create and furnish tools not only are opening possibilities for what can be made next, they are inviting new forms directly in. They are also ushering old forms out as there is only so much room in the world for them to be stored - or at least, to be used. And if the lives and work of craftsmen is shaped by the toolmakers that came before them, how much more are the lives of those who recieve from these craftsmen! C.S. Lewis came to the same conclusion more elegantly:

\begin{quote}
  What we call Man's power over Nature turns out to be a power exercised by some men over other men with Nature as its instrument.
  \attrib{\textit{The Abolition of Man}, C.S. Lewis, 1943}
\end{quote}

Every time we choose to develop one tool rather than another, we alter the lives of those who come after us. To write off that how to use the tools given to future generations is their problem, rather than our own, is no argument against sin. In the act of creating the means for the tool, we enable the sins of future generations. Understanding this, how boldly and humbly must an engineer approach their work!

Without bombs, wars could not be fought on the level of casualty they are. Without medicines, more would die to diseases. It is not as simple as to say `make medicines, not weapons', of course. Wars are fought for good reason, and more precise weaponry that minimizes casualties, especially of innocents, is laudable. And medicines often have unintended consequences. In the past century we have seen man given powers he could never have dreamed of, and now, he is charged with their proper use.

If the forefathers' vision is incomplete, this too is a form of alienation, since the wielder of the tool does not fully participate in how it is to be used, and becomes a slave to the tool rather than its master.

\subsection{Usury}

The modern world has a very misguided idea of what economics is, that it somehow has to do inherently with money. The term ``economics" comes from the greek word ``οίκονέμoμαι", which means ``household management". In this, the household still has a further aim: for the benefit of its members both materially and spiritually.

Some have confused the aim of economics to be that of acquiring money. Not so! In fact, in one's reading of Aristotle, we find this original meaning juxtaposed to other means of wealth-getting.

\begin{quote}
  There are two sorts of wealth-getting as I have said; one is part of household management, the other is retail trade: the former necessary and honourable, which that which consists in exchange is justly censured; for it is unnatural, and a mode by which men gain from one another. The most hated sort, and with the greatest reason, is usury, which makes a gain out of money itself, and not from the natural object of it. For money was to be used in exchange, but not to increase at interest. And this term interest [τοκος], which means the birth of money from money, is applied to the breeding of money because the offspring resembles the parent. Wherefore of all modes of getting wealth this is the most unnatural.
  \attrib{\textit{Politics} Bk.1 Ch.10, Aristotle}
\end{quote}

This self-begetting interest must always be checked lest it become usury. Like a cancer, what unchecked usury does is crowd out truly productive endeavors, encumbering them and leeching off their resources. A renowned Austrian economist remarked,

\begin{quote}
  We have a system that increasingly taxes work and subsidizes nonwork.
  \attrib{\textit{Milton Friedman}}
\end{quote}

And, of course, money ultimately must be exchanged for goods and services.

% TODO: better quote on usury. Friedman quote is disingenuous.

\begin{quote}
%  In the Carboniferous Epoch we were promised abundance for all,
%  By robbing selected Peter to pay for collective Paul;
  But, though we had plenty of money, there was nothing our money could buy,
  And the Gods of the Copybook Headings said: "If you don't work you die."
  \attrib{\textit{Gods of the Copybook Headings}, Rudyard Kipling, 1919}
\end{quote}

This focus on monetary behavior rather than productivity has led us to choose some very poor metrics. The measure of ``Gross Domestic Product" (GDP) is often cited by economists and politicians alike as a good measure of economic well-being. But let's consider for a moment what this measure is. GDP for a given country and year, is the market value of all final goods produced in that country and year. This excludes goods that are not brought to market, and services that are performed for oneself!

If one decides to cook vegetables grown in their own garden and eat them rather than go out to a restauraunt, they have harmed the GDP, and by that measure, made the economy worse. If GDP is really good, then don't you dare do your own dishes. Now, this is a ridiculous prospect. Anyone with any sort of sense knows that to be frugal and self-reliant is a good, not an evil. Yet this foolish metric is used as justification for many economic policies, and it prys our eyes away from fruitful works and towards alluring monetary numbers.

% https://lukesmith.xyz/articles/why-its-bad-to-have-high-gdp

\iffalse
\subsection{Gnosticism}

Seeing these disconnecting patterns, the other heresy has become apparent in our times as well: the \textit{gnostic} heresy: that material things are flawed, evil, or are in some sense inauthentic or a distraction. These heretics have been throroughly rebuked by such saints as Irenaeus.
\fi

\subsection{Gluttony}

The developed world is technologically \textit{gluttonous}. This term has connotations with food, the dominant source of energy for the ancients, but there is no reason we cannot apply it to the ever-growing consumption of disposable goods, electricity, and the like. In fact, since now our predominant source of energy (which powers our predominant source of computational power) is not from food, it makes good sense to apply the virtue of \textit{temperance} to these created and refined resources and goods.

A typical environmentalist response to the growing demand for electricity, primarily is that we should seek the same levels of consumption with less inputs (i.e. increased efficiency) and forms of energy generation which do not pollute as much.

This is not a problem of a technical nature, but a manifestation of glut. We must ask ourselves: is our consumption (at least in the non-destitute of the developed world) to a point of nourishment, or a point of obesity?

% TODO: find some sources on obesity!

\section{The Ideal}

The current technological landscape lacks a vision. But technology should have ends.

\subsection{Virtue}

If the last section has made you despair, despair no more. The decaying state of affairs will be fertile compost for virtue to be planted. In the present age there is a longing to rediscover discarded tradition - and for us westerns and near-easterns, this is a Christian one. We must heed the essential teachings of our blessed Lord that the real value of a society is its moral nature. Its development of technology is a manifestation which reinforces this nature. As Leo XIII writes in \textit{Rerum Novarum},

\begin{quote}
  From contemplation of this divine Model [of Jesus Christ], it is more easy to understand that \textbf{the true worth and nobility of man lie in his moral qualities}, that is, \textbf{in virtue}.
  \attrib{\textit{Rerum Novarum} 24, Pope Leo XIII 1891, emphases added}
\end{quote}

If you cannot swallow this fundamental truth, no true wisdom will make any sense or be of any use.

Once swallowed and digested, things become clear. We recognize that our forefathers had virtue, and we desire to learn their understanding, apply it to our own life, and pass it on to our children. The solution to our groanings and pains is not in the modern religions of `innovation' and `scientific thinking'. It is in the past.

\begin{quote}
  All this lyrical stuff about entering the Aquarian Age and reaching a new level of consciousness and taking the next step in evolution is nonsense. Much of it is a sort of delusion of grandeur, the kind of thing you hear from people in the loony bin. What I'm struggling to do is help recapture something our ancestors had. If we can just regain the consciousness the West had before the Cartesian Revolution, which I call the Second Fall of Man, then we'll be getting somewhere.

  \attrib{\href{https://www.religion-online.org/article/small-is-beautiful-and-so-is-rome-surprising-faith-of-e-f-schumacher/}{\textit{Small is Beautiful, and So is Rome}}, Schumacher 1977}
\end{quote}

Or as Leo XIII tells us again to keep our eyes on the source of life and font of immortality,

\begin{quote}
  If human society is to be healed now, in no other way can it be healed save by a return to Christian life and Christian institutions. When a society is perishing, the wholesome advice to give to those who would restore it is to call it to the principles from which it sprang; for the purpose and perfection of an association is to aim at and to attain that for which it is formed, and its efforts should be put in motion and inspired by the end and object which originally gave it being. Hence, to fall away from its primal constitution implies disease; to go back to it, recovery.
  \attrib{\textit{RN} 27, Leo XIII 1891}
\end{quote}


\subsection{Wisdom}

If we want this wisdom, we must be willing to be educated. While some might say `holistic' education is important, the reality is more severe: \textit{integrated} education is important, along with a recognition that all technologies are built \textit{from} the created firmament, and ought to exist \textit{for} created life.

\begin{quote}
What do I miss, as a human being, if I have never heard of the Second Law of Thermodynamics? The answer is: Nothing. And what do I miss by not knowing Shakespeare? Unless I get my understanding from another source, I simply miss my life.
\attrib{\textit{Small is Beautiful} 87, Schumacher 1973}
\end{quote}

Or, as the stoics might say, ``the unexamined life is not worth living". The Christian would simply point to the essential teaching uttered over and over that a life not lived through Jesus Christ is no life at all: all others perish. If we build wonderful technologies that reduce material sufferring, but we do it at the cost of even one less soul desiring our Lord, it is for naught.

\iffalse
Indeed to harken back to Eden,

\begin{quote}
  Sorrow is knowledge; they who know the most

  Must mourn the deepest o'er the fatal truth,

  The Tree of Knowledge is not that of Life.
  \attrib{\textit{Manfred}, George Gordon Byron 1817}
\end{quote}
\fi

\begin{quote}
  Education cannot help us along as it accords no place to metaphysics. Whether the subjects taught are subjects of science or of the humanities, if the teaching does not lead to a clarification of metaphysics, that is to say, of our fundamental convictions, it cannot educate a man, and consequently, is of no value to society.
  \attrib{\textit{Small is Beautiful} 93, Schumacher 1973}
\end{quote}

This might suggest that we need to have endless `philosophy of <<x>>' courses- this is absolutely not so! It merely means that the learned skills must constantly be pointing back to underlying truths, and this is the essential purpose, not an ancillary benefit. If we had good technical courses like this, there would be scant need for these `philosophy of <<x>>' courses: one would simply take the course and be enriched by it. As Joel Barstad puts it in his \href{https://byzantinela.com/cappadocian-house-proposal/}{\textit{Cappadocian House Proposal}},

\begin{quote}
  I added an insistent thirst to overcome certain conventional and traditional oppositions, among them the separation of intelligence from sanctity, of study from worship, of the liberal arts from the arts of subsistence, of the speculative from the practical and creative. How could I hope to know the Word without bowing before Him in worship and them giving him a birth in the materiality of my life?
  \attrib{\textit{Cappadocian House Proposal} 3, Joel Barstad 2017}
\end{quote}

% This will have a secondary benefit which many teachers currently acknowledge, in that knowledge will be linked together thus being easier to remember and recall. 

\subsection{Temperance}

Just as good, nourishing food is hard to indulge in, so too would well-developed technology. I do not mean phones with timers that prevent us from checking email for too long, I mean that utilizing the technology should satiate.

\hfill

\hfill

\hfill



\subsection{Diligence}

The manual, human aspect of work must not be forgotten. There are a great many ills that are curable by labor. We have already discussed the prominence of gyms, but how much better would the virtue of diligence be instilled if it were provided alongside others required for hard manual labor?

\begin{quote}
  Man matures through work that inspires him to do difficult good.
  \attrib{Pope John Paul II}
\end{quote}

\hfill

\hfill

\hfill

\hfill

\subsection{Patience}
\hfill

\hfill

\hfill

\hfill

\hfill

\hfill

\hfill

\subsection{Humility}

We technologists must not presuppose that every invention we put forth will be used wisely.

\hfill

\hfill

\hfill

\hfill

\subsection{Charity}

We technologists often face the reverse trouble, that every invention we put forth will be misused and wasted by laypersons because they lack sufficient knowledge. This is not 

\hfill

\hfill

\hfill

\hfill

\subsection{Subsidiarity}

\begin{quote}
  Neither must it be supposed that the solicitude of the Church is so preoccupied with the spiritual concerns of her children as to neglect their temporal and earthly interests. Her desire is that the poor, for example, should rise above poverty and wretchedness, and better their condition in life; and for this she makes a strong endeavor.
  \attrib{\textit{RN} 28, Leo XIII 1891}
\end{quote}

Even in this, Leo XIII uses very nuanced language. It would be one thing to say `the poor should not be wretched'. It is another to say that the ``poor should... better their condition."  It is quite clear that the betterment of their condition would be best derived from their own works, not to be imposed upon them externally (even if this would be their own wish).

\subsection{Fraternity}

Friendliness towards God and all his creation is the goal. If anything gets in the way of this, it must be cut off. If the pursuit of increased efficiency causes us to look down upon others as unproductive and thusly less worthy of our attention, how can we justify it?

\begin{quote}
  But the Church, with Jesus Christ as her Master and Guide, aims higher still. She lays down precepts yet more perfect, and tries to bind class to class in friendliness and good feeling.
  \attrib{\textit{Rerum Novarum} 21, Leo XIII 1891}
\end{quote}

Let us be clear here, though: the goal is not mere kindness and compassion towards others, but friendliness. The Communists' atheism stemmed from a strong sense that religion kept masters from sympathizing with their serfs, but overlooked the right relation in the other direction. We know of course to turn the other cheek and paint as much of the picture properly as possible. If someone paints the sky green we should not paint the grass blue in response. Far from being the `opiate of the masses',

\begin{quote}
  There is no intermediary more powerful than religion in drawing the rich an the working class together, by reminding each of its duties to the other, and especially of the obligations of justice.
  \attrib{\textit{Rerum Novarum} 19, Leo XIII 1891}
\end{quote}

After all, what sort of opium leaves one with the sense that they must shoulder responsibility? What opium reminds one of their sins, and tells them they are worthy of hell? Clearly, true religion is no opium: it is that virtue which binds man to man in friendly responsibility and creates that which opium seeks to destroy: fraternity.

In a fraternistic view, all should have a voice and role as to how a technology is conceived and utilized.

When considering a technology that will displace livelihoods, those whose livelihoods are affected should be consulted to ensure that the new technology meets the full needs (as often certain customers' requirements and needs are not fully understood by the technologists, but are by those whose feet are on the ground). Care should also be made that these livelihoods are not completely abjected and a new suitable line of work is found. It would be unfair, though, to outlaw new and emergent technology on the basis that it interrupts these livelihoods. To continue a job knowing that a machine \textit{could} replace you, implying that the work you provide is of a lesser nature than a mere machine, is a worse fate than to simply be displaced from a line of work and need to find another one. People should be treated with charity that desires their wellbeing, but not with coddling infantilism that denies their agency.

\section{A Prescriptive Cure}

We must now turn our hearts on how we ought to then live - what sort of penances we should undertake to right the ills which we have wrought. These ideas here do not make up all possibilities, rather, they are some inspiring jumping off points.

\subsection{Ascesis}

Return to pre-industrial levels of energy consumption (a sort of \textit{technological asceticism}) may not be necessary, but reconsideration of our lives to see that we are not gluttonous in our usage of energy and technology is. Many have remarked on the need for and benefits of \textit{fasting} from technology. Like a fast from food, such fasting would allow us to bring our passions and desires for technology in line with what is truly necessary for our subsistence, and to make clear how these things place stumbling blocks for the wellness of our souls. 

Fasting is like a palate cleanser, which is why fasting is such powerful remedy. It removes the poor taste for sin we had and allows our inmost appetite for virtue to show forth.

For one unfamiliar with the practice of fasting, \href{http://rutgersnb.occministries.org/wp-content/uploads/2015/07/St.-Basil-the-Great%E2%80%99s-First-Homily-on-Fasting.pdf}{St. Basil the Great's First Homily on Fasting} is an excellent introduction to this powerful practice. 

\begin{quote}
  You would surely agree that the pilot of a merchant ship is better able to safely guide it to port if it is not fully loaded, when it is in excellent condition and light. The ship completely loaded down is sunk by a minor swell in the waters. But the boat that has a captain smart enough to toss overboard the extra weight will ride high above even surging waves.
  That’s like people in burdened down bodies. A person gets absorbed with filling up, getting weighed down until finally falling into ill health. But those who are well-equipped, light, and truly nourished, avoid the prospect of serious disease. They are like the boat in stormy weather that goes right over a dangerous rock.
  \attrib{\textit{On Fasting} 4, St. Basil}
\end{quote}

\iffalse
\begin{quote}
  Wine wasn’t in paradise; there was not yet any slaughtering of animals, not yet any eating of meat. After the flood there was wine. After the flood, “you will eat all kinds of things, like you eat vegetables that grow from the ground.” When perfection was despaired, then the enjoyment of those things was allowed.
  Now the wine is an example of inexperience, as Noah was ignorant of the use of wine. For it had not yet come into use in life, neither been known in human custom. Since he had neither seen another do it, nor tried it himself, he was unguardedly hurt by it. “For Noah planted a vineyard, and he drank from the fruit, and he got drunk.” He wasn’t out-of-control drunk, he just wasn’t aware of the potent thing he was consuming.
  \attrib{\textit{On Fasting} 5, St. Basil}
\end{quote}

\begin{quote}
  What did Esau throw away, and so was made a slave of his brother? Didn’t he sell his rights as first-born for a single meal? By contrast, wasn’t it with fasting and prayer that Hannah was favored to become the mother of Samuel?
  \attrib{\textit{On Fasting} 6, St. Basil}
\end{quote}
\fi

\subsection{Intermediate Technology}
\hfill

\hfill

\hfill

\hfill

\hfill

\hfill

\hfill

\subsection{Right to Repair}
\hfill

\hfill

\hfill

\hfill

\hfill

\hfill

\hfill

\subsection{Recreation}
\hfill

\hfill

\hfill

\hfill

\hfill

\hfill

\hfill

\subsection{Federation}
\hfill

\hfill

\hfill

\hfill

\hfill

\hfill

\hfill

\end{document}
